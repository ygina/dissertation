%-------------------------------------------------------------------------------
\section{Conclusion}
%-------------------------------------------------------------------------------

We performed an emulation measurement study on the impact of
connection-splitting PEPs on sustained throughput in the context of recent
developments such as BBR and QUIC. We found that TCP BBR benefits more from
splitting today than when it was first released, and its current version
benefits in settings where TCP CUBIC does not. QUIC congestion-control
implementations exhibit substantial variability both within QUIC and with
Linux TCP.

In the short term, we urge researchers to refer to congestion-control schemes
by algorithm/implementation/version, not just ``BBR'' or even ``QUIC BBRv1''.
We urge the community to create regression and acceptance tests---possibly
including our heatmaps with ``permissible zones''---for a scheme to call
itself an implementation of the XYZv1 algorithm.

For connection-splitting to truly be relevant again, there must be clear
scenarios where PEPs offer significant benefits but end-to-end solutions fall
short. This paper is a first step, but real-world studies are needed. Next,
even though the problem is old, the solution need not be to ``just do the same
things.'' We hope this paper motivates the community to pursue
protocol-agnostic approaches to in-network assistance and destigmatize the
controversial nature of PEPs.
