%-------------------------------------------------------------------------------
\begin{abstract}
%-------------------------------------------------------------------------------

In the 1990s, many networks deployed performance-enhancing proxies
(PEPs) that transparently split TCP connections to aid performance,
especially over lossy, long-delay paths. Two recent
developments have cast doubts on their relevance: the
BBR congestion-control algorithm, which de-emphasizes loss as a
congestion signal, and the QUIC transport protocol, which prevents
transparent connection-splitting yet empirically matches or exceeds
TCP's performance in wide deployment, using the same congestion control.


% In the 1990s, many networks deployed performance-enhancing proxies
% (PEPs) that transparently split TCP connections to aid performance,
% especially over long-delay paths with packet loss. Two recent
% developments have called PEPs' supposed benefits into question: the
% BBR congestion-control scheme, which de-emphasizes packet loss as a
% congestion signal, and the QUIC transport protocol, which prevents
% transparent connection-splitting but is said to achieve similar or
% better empirical performance than TCP in wide deployment, using the
% same congestion-control schemes, without benefiting from TCP PEPs in
% the field.

In light of this, are PEPs obsolete? This paper presents a range of
emulation measurements indicating: ``probably not.'' While BBR's
original 2016 version didn't benefit markedly from connection-splitting, more
recent versions of BBR \textit{do} and, in some cases, even \textit{more} so
than earlier ``loss-based'' congestion-control algorithms.  We also find that
QUIC implementations of the ``same'' congestion-control algorithms vary
dramatically and further differ from those of Linux TCP---frustrating
head-to-head comparisons.
Notwithstanding their controversial nature, our results suggest
that PEPs remain relevant to Internet performance for the foreseeable future.

% In light of this, are PEPs obsolete? This paper presents a range of
% emulation measurements indicating: ``probably not.'' While BBR's
% original 2016 version didn't benefit markedly from
% connection-splitting, more-recent versions of BBR behave differently,
% and can benefit a great deal from PEPs: in some cases, greater than
% earlier ``loss-based'' congestion-control schemes.  We also find that
% QUIC implementations of the ``same'' congestion-control schemes vary
% dramatically and further differ from those of Linux TCP---frustrating
% attempts to draw robust conclusions from head-to-head
% comparisons.
% Notwithstanding their controversial nature, our results suggest
% that PEPs remain relevant to Internet performance for the foreseeable future.

\end{abstract}
