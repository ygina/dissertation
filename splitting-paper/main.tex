%%%%%%%%%%%%%%%%%%%%%%%%%%%%%%%%%%%%%%%%%%%%%%%%%%%%%%%%%%%%%%%%%%%%%%%%%%%%%%%%
% Template for USENIX papers.
%
% History:
%
% - TEMPLATE for Usenix papers, specifically to meet requirements of
%   USENIX '05. originally a template for producing IEEE-format
%   articles using LaTeX. written by Matthew Ward, CS Department,
%   Worcester Polytechnic Institute. adapted by David Beazley for his
%   excellent SWIG paper in Proceedings, Tcl 96. turned into a
%   smartass generic template by De Clarke, with thanks to both the
%   above pioneers. Use at your own risk. Complaints to /dev/null.
%   Make it two column with no page numbering, default is 10 point.
%
% - Munged by Fred Douglis <douglis@research.att.com> 10/97 to
%   separate the .sty file from the LaTeX source template, so that
%   people can more easily include the .sty file into an existing
%   document. Also changed to more closely follow the style guidelines
%   as represented by the Word sample file.
%
% - Note that since 2010, USENIX does not require endnotes. If you
%   want foot of page notes, don't include the endnotes package in the
%   usepackage command, below.
% - This version uses the latex2e styles, not the very ancient 2.09
%   stuff.
%
% - Updated July 2018: Text block size changed from 6.5" to 7"
%
% - Updated Dec 2018 for ATC'19:
%
%   * Revised text to pass HotCRP's auto-formatting check, with
%     hotcrp.settings.submission_form.body_font_size=10pt, and
%     hotcrp.settings.submission_form.line_height=12pt
%
%   * Switched from \endnote-s to \footnote-s to match Usenix's policy.
%
%   * \section* => \begin{abstract} ... \end{abstract}
%
%   * Make template self-contained in terms of bibtex entires, to allow
%     this file to be compiled. (And changing refs style to 'plain'.)
%
%   * Make template self-contained in terms of figures, to
%     allow this file to be compiled.
%
%   * Added packages for hyperref, embedding fonts, and improving
%     appearance.
%
%   * Removed outdated text.
%
%%%%%%%%%%%%%%%%%%%%%%%%%%%%%%%%%%%%%%%%%%%%%%%%%%%%%%%%%%%%%%%%%%%%%%%%%%%%%%%%

\documentclass[letterpaper,dvipsnames,twocolumn,10pt]{article}
\usepackage{usenix}

% to be able to draw some self-contained figs
\usepackage{nopageno}
\usepackage{tikz}
\usepackage{cite}[sort]
\usepackage{amsmath}
\usepackage[labelfont=bf,font=small]{caption}

% \usepackage{titlesec}
\usepackage[compact]{titlesec}
\usepackage{booktabs}
\usepackage{enumitem}

% figures
\usepackage{subcaption}
\usepackage{float}
\usepackage{cleveref}
\usepackage{amssymb}
\usepackage{tcolorbox}

% cleveref
\makeatletter
\newcommand{\nospaceS}{\S\@gobble}
\makeatother
\Crefname{section}{\nospaceS}{\nospaceS}
\Crefname{figure}{Fig.}{Figs.}
% \Crefname{table}{Tab.}{Tabs.}
\Crefname{listing}{Listing}{Listings}
\crefalias{lstlisting}{listing}

% Listings
\usepackage[scaled=0.85]{beramono}
\usepackage{listings}
\usepackage{hyperref}
\usepackage{xcolor}

% hyperlinks
\hypersetup{
    colorlinks=true,
    urlcolor=blue,
    breaklinks=true
}
\urlstyle{same}

\newfloat{lstfloat}{htbp}{lop}
\floatname{lstfloat}{Listing}
\def\lstfloatautorefname{Listing} % needed for hyperref/auroref

\newcommand\code[1]{\lstinline$#1$}

\lstdefinelanguage{PythonDiff} {
 keywords={for, in},
 keywordstyle=\color{BrickRed}\bfseries,
 ndkeywords={transfer},
 ndkeywordstyle=\color{NavyBlue}\bfseries,
 basicstyle=\small\ttfamily,
 identifierstyle=\color{black},
 sensitive=false,
 comment=[l]{+},
 commentstyle=\color{ForestGreen}\ttfamily\bfseries,
 showstringspaces=false,
 string=[s]{\#}{\.},
 stringstyle=\color{gray}\ttfamily,
}

\lstdefinelanguage{Python} {
 keywords={class, def, return, instanceof, if},
 keywordstyle=\color{BrickRed}\bfseries,
 ndkeywords={DataFrameSplit, NdArraySplit, SplitType, CPU, GPU, @oa, @oa_alloc},
 ndkeywordstyle=\color{NavyBlue}\bfseries,
 basicstyle=\small\ttfamily,
 identifierstyle=\color{black},
 sensitive=false,
 comment=[l]{\#},
 commentstyle=\color{gray}\ttfamily\bfseries,
 showstringspaces=false,
 string=[s]{"}{"},
 stringstyle=\color{ForestGreen}\ttfamily,
}

\lstset{language=Python}

%-------------------------------------------------------------------------------
\begin{document}
%-------------------------------------------------------------------------------

\newcommand{\keith}[1]{\textcolor{red}{{\sf (KW: #1)}}}
\newcommand{\thea}[1]{\textcolor{red}{{\sf (TR: #1)}}}
\newcommand{\gina}[1]{\textcolor{red}{\sf {(GY: #1)}}}

%don't want date printed
\date{}

% make title bold and 14 pt font (Latex default is non-bold, 16 pt)
\title{\Large \bf Internet Connection Splitting: What's Old is New Again}

%for single author (just remove % characters)
\author{
{\rm Gina Yuan}\\
Stanford University
\and
{\rm Thea Rossman}\\
Stanford University
\and
{\rm Keith Winstein}\\
Stanford University
} % end author

\maketitle

%-------------------------------------------------------------------------------
\begin{abstract}
%-------------------------------------------------------------------------------

This paper describes and evaluates new techniques for
network-originated retransmissions for end-to-end transport
connections, yielding performance benefits for encrypted transport
protocols in lossy settings.  We use set-reconciliation techniques
based on the Rateless IBLT to let the receiver efficiently acknowledge
encrypted packets to a network function, without modifying the sender
or the underlying wire format. With these tools, endpoints can receive
network-originated retransmissions appropriate for them, without an
additional layer of encapsulation, tunnel, or link-layer
retransmissions.

\end{abstract}

%-------------------------------------------------------------------------------
\section{Introduction}
%-------------------------------------------------------------------------------    

The relationship between Internet transport protocols and lossy
wireless networks has long been fraught. Three decades ago, the
seminal Snoop work~\cite{balakrishnan1995snoop} addressed an issue
that emerged with the rise of wireless Ethernet (Wi-Fi): TCP's
end-to-end reliability works poorly over network segments that
regularly experience non-congestive packet loss. This is for two big
reasons: it's wasteful to require frequent end-to-end retransmissions
over a whole network path to address packet loss on one wireless link,
and because most congestion-control schemes interpret packet loss as a
sign of congestion and will slow down in response---a mistake when the
loss wasn't caused by congestion. From its point of view in 1995, the
Snoop paper outlined three plausible solutions:
\begin{itemize}[topsep=0pt,itemsep=0pt]
\item \textbf{Splitting TCP connections}~\cite{bakreitcp1994, bb95}. A
  proxy at the wireless base station transparently interposes on TCP
  connections that cross it, creating two concatenated connections:
  one from ``fixed host'' to proxy, and from proxy to the ``mobile
  host.'' This lets retransmission occur over the lossy segment only,
  and prevents the fixed host's congestion control from seeing
  non-congestive wireless losses.

\item \textbf{Link-level acknowledgment and
  retransmission}~\cite{palplus}. Wireless NICs add their own
  encapsulation (with their own sequence numbers or packet
  identifiers) to each packet, reply to incoming packets with
  link-layer acknowledgments, detect apparent losses from the absence
  of an ACK, and retransmit lost packets before the end-to-end
  transport protocol concludes they were lost.

\item \textbf{The Snoop approach}, where an on-path network function
  transparently interprets TCP acknowledgments in-flight, detects
  losses, retransmits packets over the lossy link only, and
  temporarily hides loss from the other host to avoid a duplicate
  retransmission.
\end{itemize}

In the intervening years, the first two approaches were widely
deployed. At the boundary between the wired Internet and a lossy
wide-area subpath (e.g.~a satellite or cellular network), network
operators used connection-splitting performance-enhancing proxies
(PEPs) to accelerate TCP connections crossing their networks. For
lossy \emph{local}-area subpaths, Wi-Fi and cellular networks use
link-layer sequence numbers, selective ``block'' acknowledgments,
link-layer retransmissions, and receiving NICs that sometimes hold
back received packets to wait for retransmissions of earlier packets
so that hosts receive packets in the original order to avoid
triggering a transport-layer end-to-end retransmission~\cite{rfc3366,
  rfc8985, 802.1ac, 5Greorder}. Because of the practical success of
the first two approaches, Snoop's ``network-originated
retransmissions'' didn't seem to be needed in practice.

Until, this paper proposes, maybe now. Post-TCP transport protocols,
especially QUIC~\cite{rfc9000}, encrypt and authenticate their
packets. This puts some operators in a bind, because they can no longer
``split'' these connections.
But what about link-layer ACKs and retransmissions?
This is a usable strategy on low-latency wireless links when one party
(or standards body) controls both sides, and losses can be patched
before the end-to-end protocol detects them. But when the lossy
subpath has higher latency (as for some satellite or experimental
networks), losses can't be patched before the transport protocol
detects them, and the head-of-line blocking necessary to put
packets back in order at the receiver can be ruinous to real-time
applications. In any event, such techniques can't be deployed
unilaterally by a network operator who doesn't control the receiver or
the encapsulation format.

What else could address this? Traffic sources could deploy
congestion-control and loss-detection schemes that are less sensitive
to loss and reordering~\cite{rfc8985}, but this is also out of a
network operator's control. Yet another approach could involve
Sidekick proxies~\cite{yuan2024sidekick}---but these provide
in-network \emph{acknowledgments} of encrypted transport protocols
(for hosts to interpret and perhaps trigger retransmission), not
in-network \emph{retransmission} based on the encrypted traffic of
hosts.

Given all this, what really happens today, at least some
of the time, is that satellite and other nontraditional network
operators block QUIC traffic, forcing hosts to fall back to TCP that
can be split as before. We think it may be possible to do better by drawing on ``Snoop-style''
ideas. In this paper, we describe and evaluate \Sys\footnote{For
``packet rateless retransmission''---and because the \Sys proxy keeps
a small cache of recent packets-in-flight for possible
retransmission.}: a technique for network-originated retransmissions
for encrypted end-to-end transport connections, yielding performance
benefits in lossy settings. We adapt set-reconciliation techniques
based on the Rateless IBLT~\cite{yang2024practical} to let the
receiver efficiently acknowledge encrypted packets (in an ``eACK'') to a network
function, without modifying the sender or the underlying wire
format. Compared with Sidekick, which also involves
acknowledgments over encrypted sequence numbers, these eACKs
are faster to decode, making them appropriate for a deployment
where network functions are the ones interpreting acknowledgments over
encrypted sequence numbers from many connections in parallel and
triggering retransmissions as a result.

% \begin{figure}
%     \centering
%     \includegraphics[width=\linewidth,height=35mm]{figures/packrat-scope.pdf}
%     \caption{Just a suggestion from Michael, to include a diagram like this. The editable pptx file is in the figures folder.}
%     \label{fig:packrat-scope}
% \end{figure}

We integrate the \Sys protocol with several applications built on encrypted
transport protocols, and show that it can enable a variety of performance
enhancements in various settings with a lossy path segment near the data receiver,
compared to end-to-end loss recovery schemes:

\begin{enumerate}[noitemsep]
\item Application 1: Improved throughput for large file downloads using QUIC and HTTP/3.
\item Application 2: Reduced packet tail latency of a low-latency media stream using a simple repetition code.
\item Application 3: Reduced end-to-end retransmissions at the server of a reliable multicast RTP stream.
\end{enumerate}

\noindent Unlike link-layer retransmissions, the assistance from a \Sys can be
deployed anywhere along the network path and tailored to the performance
demands of each base connection. Compared to existing transport-layer
solutions, \Sys can be applied to arbitrary transport protocols without
ossification.

\paragraph{Summary of results.}
To summarize, we show that it is possible to provide per-connection in-network
retransmissions to encrypted transport protocols using the \Sys protocol along
an adjacent connection, without involving the data sender nor modifying the
wire format of the underlying connection. We make the following contributions:

\begin{itemize}[noitemsep]
    \item The \Sys protocol and a mechanism at the \textit{endpoint} to correct
     end-to-end reordering signals when there are in-network retransmissions
     (\Cref{sec:packrat-protocol}).
    \item An acknowledgment of encrypted packets based on the Rateless IBLT
     that is fast to decode (\Cref{sec:eack}).
    \item Optimizations to improve proxy
     scalability based on the co-location of the endpoint and eACK sender
     (\Cref{sec:eack:hints}).
    \item An implementation of the \Sys proxy and integrations with three
     applications that use encrypted transport (\Cref{sec:implementation}).
    \item An emulation study of the performance enhancements, as well as the CPU,
     memory, and link overheads of using the \Sys protocol (\Cref{sec:evaluation}).
    % \item Real-world experiments using an in-line commodity device behind a Wi-Fi AP as the \Sys.
\end{itemize}

\section{Background}
\label{sec:background}

We provide a brief history of loss recovery mechanisms in the network to
motivate the need for lightweight, non-ossifying, in-network retransmissions
for encrypted transport protocols.

\paragraph{End-to-end loss recovery.}

% End-to-end loss recovery has traditionally served two purposes: facilitating
% retransmissions in reliable connections and adapting congestion control.
In end-to-end loss recovery,
congestion control algorithms were traditionally designed to treat all loss as an indicator of
congestion~\cite{rfc5681tcp,rfc2001tcp}.
% congestion~\cite{rfc8985,rfc9000,rfc5681tcp,rfc2001tcp,ha2008cubic,cardwell2024bbrv3-ietf120}.
% only react upon congestive loss~\cite{2019ccacensus}.

With modern wireless networks and higher bit-error rates, recent developments
such as the BBR congestion control
% (Bottleneck Bandwidth and Round-trip propagation time)
algorithm have de-prioritized packet loss as a congestion
signal~\cite{cardwell2017bbr}. This change enables higher throughput but has
raised concerns regarding TCP fairness~\cite
{ware2019modeling,philip2024prudentia}. In response, the performance of newer
versions of BBR has regressed in lossy environments, suggesting a lasting trend
in the treatment of loss as a congestion signal regardless of its source~\cite
{atc-submission}.

% regarding fairness among competing flows, suggesting that packet loss---whether
% due to congestion or non-congestive factors---remains closely associated with
% congestion dynamics and needs to be considered by
% CCAs~\cite{ware2019modeling,philip2024prudentia}.

\paragraph{Link-layer loss recovery.}

Loss recovery at the link layer~\cite{3gpp5gstandard,le2022link,ieee80211e} can
mitigate some of the dramatic reactions of CCAs by hiding random loss from
endpoints.
% The local RTT is much shorter than the end-to-end RTT such that
% sporadic losses can be repaired much faster (milliseconds or less).
% Local
% retransmissions also minimize overhead as they are confined to the lossy path
% segment itself.
% Note link-layer retransmissions are not applicable to higher
% latency settings such as satellite networks.
Applications and transport protocols are oblivious to this
functionality, and there is a tradeoff to this reliability~\cite{klingler2018impact,kliazovich2012arqproxy}.
Retransmitting more often increases the chance
of success but increases the time needed, and may also introduce
jitter and reordering~\cite{leung2007overview}.
Low-latency applications may prefer to transmit new data instead, but
retransmitting less often means more loss.
There is no ideal configuration for all connections that share the link.

% Detecting loss and retransmitting packets at the link layer has great benefits.
% The local RTT is much shorter than the end-to-end RTT, such that sporadic
% losses can typically be repaired much faster. Such local retransmissions also
% minimize overhead, as they are confined to the lossy path segment itself. Link
% layer loss recovery, e.g. at Wi-Fi or cellular links, can hide random loss from
% the endpoints such that the dramatic reaction of CCAs is avoided; applications
% and transport protocols are oblivious to this
% functionality~\cite{ratnamwtcp1998,ieee80211e,kliazovich2012arqproxy,wong1997linklayerrtx}.
% Retransmissions by link layers are a common source of jitter, and,
% depending on the behavior of the link's receiver, they may contribute to
% reordering in the network~\cite{perf2006linklayerrtx,leung2007overview}.

% There is a trade-off between reliability and the delay that local
% retransmissions produce. Trying longer (i.e., more often) increases the time
% needed and the chance of success. Rather than trying to retransmit (and delay)
% old data, some applications may prefer transmitting newer data instead; thus,
% ideally, applications should be in control of this trade-off, but they are not.
% Various ideas to improve the communication between applications and wireless
% link layers have been put forward~\cite{kwbdgrr-tsvwg-net-collab-rqmts-04}, but
% with little success~\cite{rfc9049}. The new Standard Communication with Network
% Elements (SCONE) Working Group in the IETF now defines such signaling for QUIC,
% and Explicit Congestion Notification (ECN) is such a signal---but these are in
% the direction of the network offering information (``throughput advice'') to
% endpoints, not vice versa~\cite{rfc3168,brw-scone-analysis-00}.
% This is not completely true: the DSCP offers a signal. E.g., RFC 8325 describes
% how the DSCP should be mapped onto 802.11 networks. but this is a coarse-grain,
% unreliable signal (the field may be zeroed on the path), and I don't know if  a
% certain DSCP choice translates into limiting the number of retransmits. All in
% all, I think it's best to avoid this complication and ignore this.

\paragraph{Transport-specific mechanisms.}

Some transport-layer proxies split a connection into two or more segments
and optimize the connection for each segment. In addition to
TCP connection splitters~\cite{rfc3135,honda2011still,hayes2019mmwave},
selective forwarding units (SFUs) are
used for media streams where low-latency is critical, so retransmissions occur
closer to the data receiver~\cite{rfc7667,andre2018comparative}.

% the "proxyimportance" ref may sound like it's just about mmWave but it
%  contains 4G measurements, finding proxy deployment

With Snoop~\cite{balakrishnan1995snoop}, it is clear
that complexities emerge in how the proxy interprets
and modifies plaintext sequence numbers on the wire, which also ossifies TCP.
This paper aims to explore how such a proxy can send helpful in-network
retransmissions to a variety of encrypted protocols while preserving
end-to-end semantics, without access to such information.


These types of approaches make unwarranted assumptions about
transport headers ossify the protocol~\cite{papastergiou2017deossifying}.
By making reliability guarantees to the endpoint, the connection-splitting
proxies also fate share with the connection.


% Another disadvantage is that these proxies fate share with th
% Any type of proxy assistance that cannot ultimately fall back to end-to-end
% mechanisms mandates that the proxy fate share with the connection.

% However, this requires
% credentialing the proxy and would not work for encrypted protocols or
% applications that don't pay to host a CDN.

\paragraph{Protocol-agnostic transport assistance.}

Forward error correction can be used in lossy environments, but this
introduces significant overheads as entire packet contents (and not just
sequence numbers) must be encoded during transmission~\cite
{rfc9265}. FEC also requires participation from both endpoints.

Mechanisms that encapsulate packets
% I like the Kramer MASQUE paper b/c it talks about applying MASQUE as a PEP
% The RFC is the introduction of MASQUE, but seems like some people cite the 
% original draft too.
offer potential for reliability without decrypting packet contents, but their
primary focus tends to be security. VPNs introduce a costly
encryption layer.
MASQUE tunnels protocols over HTTP/3, and if used for local retransmissions
has the potential for complex nested loss recovery and congestion control
interactions~\cite{rfc9298masque,schinazi-masque-proxy-05,kramer2021masquepep}.
It is intended to function per-application, and the best known implementation
from Apple functions like a VPN~\cite{icloud-private-relay}.
These proxies may also not be on-path, which harms performance.

% The MASQUE standards allow configurations with local retransmissions, e.g. when
% implementing a MASQUE tunnel with HTTP/2 over TCP, but the specifications
% present this as a disadvantage---which it is indeed, since MASQUE tunnels are
% not defined to be chosen per application, and some applications may not desire
% the added delay from local retransmits at all. Instead, the best known
% implementation, Apple Private Relay, functions like a VPN, carrying all traffic
% of a user through MASQUE tunnels. Another disadvantage of MASQUE configurations
% with local retransmissions is that TCP, or QUIC, as a tunnel protocol, is a
% full-fledged transport implementation, including additional functions such as
% congestion control (i.e., carrying QUIC over TCP or QUIC produces nested
% congestion control loops).

% advises against using HTTP/3 in
% ``reliable'' mode due to the potential for complex nested loss recovery and
% congestion control interactions.
% \thea{Can't find citations for this recommendation in RFC.}


% While this does not exist to our knowledge, what about a tunneling approach
% To generically enhance transport layer reliability, one approach involves
% encapsulating the transport protocol within a reliable transport protocol at a
% proxy. This method allows for protocol separation without requiring
% credentialed access to packet contents, effectively integrating aspects of
% link-layer retransmissions and transport-layer connection splitting.

The Sidekick proxy provides protocol-agnostic assistance by sending
``quACKs'' of encrypted packets to an endpoint, but cannot retransmit
packets itself~\cite{yuan2024sidekick}. This limits its usefulness
to when the loss is near the data \textit{sender}.
However, client traffic (which is near the lossy last-mile) is typically heavier
in the downlink direction. In this paper, we similarly leverage set
reconciliation in eACKs for referring to encrypted packets,
but to send network-originated retransmissions.

\begin{figure}[h]
  \centering
  \begin{tcolorbox}[colback=yellow!20, width=\linewidth, sharp corners]
    \textbf{Split Throughput Heuristic:} We can estimate the long-lived ``split
     throughput'' of a connection by measuring the long-lived ``end-to-end
     throughput'' on each segment of the split path and taking the minimum.
  \end{tcolorbox}
  \caption{A heuristic for evaluating split behavior from end-to-end behavior in
   an emulated network.}
  \label{fig:splitting:heuristic}
\end{figure}

%-------------------------------------------------------------------------------
\section{Measurement Methodology}
\label{sec:methodology}
%-------------------------------------------------------------------------------

\begin{figure}
    \centering
    \includegraphics[width=\linewidth]{splitting-paper/figures/mininet-topology.png}
    \caption{Two-segment network topology in \texttt{mininet}. The
     middle node splits the path into two segments with different properties.}
    \label{fig:mininet}
\end{figure}

\begin{table}[t!]
  \centering
\footnotesize
\begin{tabular}{ l l l l l }
  \toprule
    \textbf{CCA} & \textbf{Implementation}\\
    \midrule
    BBRv3 & Linux TCP v6.4.0+ \texttt{google-bbr/v3} fork\\
    BBRv2 & Linux TCP v6.4.0+ \texttt{google-bbr/v2alpha} fork\\
    BBRv1 & Linux TCP v5.15.0-122-generic \\
    CUBIC & Linux TCP v5.15.0-122-generic \\
    % TCP & BBRv1 & Ubuntu 22.04.2 with Linux kernel 6.1, 6.8; Ubuntu 18.04.1 with Linux kernel\\
    % & & 4.15, 4.16, 4.17, 4.18, 5.0, 5.4, 5.19; Ubuntu 18.04.1 with Linux kernel v4.9,\\
    % & & 4.10, 4.11, 4.12, 4.14 (Docker Ubuntu 16.04 for build)\\
    BBRv3 & Google \texttt{quiche} v131.0.6728.1 \\
    BBRv1 & Google \texttt{quiche} v131.0.6728.1 \\
    CUBIC & Google \texttt{quiche} v131.0.6728.1 \\
    BBRv2 & Cloudflare \texttt{quiche} v0.14 \\
    BBRv1 & Cloudflare \texttt{quiche} v0.14 \\
    CUBIC & Cloudflare \texttt{quiche} v0.14 \\
    BBRv2 & IETF \texttt{picoquic} \texttt{29c7c53} \\
    BBRv1 & IETF \texttt{picoquic} \texttt{29c7c53} \\
    CUBIC & IETF \texttt{picoquic} \texttt{29c7c53} \\
\bottomrule
\end{tabular}
  \caption{\small \label{tab:cca-implementations} The congestion control schemes and
   transport protocol implementations we evaluate in the measurement study.}
  \vspace{-0.4cm}
\end{table}


We want to evaluate different congestion control schemes in a variety of network
settings. To do this, we run emulation experiments in \texttt{mininet} with
simple HTTPS clients and servers to measure the throughputs of long-lived data
transfers. In this section,
we describe our emulated network configurations, HTTPS endpoints and
PEP, and other specifications.

\paragraph{Network configuration.}

We used two linear network topologies: a one-segment topology for caching
end-to-end measurements and a two-segment topology for evaluation with a
connection-splitting PEP (\Cref{fig:mininet}). Both have
a client and server node at each end. The two-segment topology additionally has
a router node in between. Each path segment has a bridging node to emulate network
properties on the link.

We parameterize each path segment in three dimensions: delay, bandwidth, and a
random loss rate. We configure the network properties on the bridging nodes'
egress interfaces, using \texttt{tc-netem} to set delay and random loss,
and \texttt{tc-htb} to set bandwidth. Additionally, we use \texttt{tc-qdisc} to
configure the queues to use RED\footnote{We apply RED and not droptail because it gives more
continuous feedback about loss for congestion control, and is commonly used in
core routers. We also do not need the multi-flow and low-delay properties of
other queue disciplines. We use RED in \texttt{adaptive harddrop} mode with a
maximum queuing delay of $\approx1$ BDP. Exact parameters are available in the code.},
which is the source of congestive loss.
Each link is symmetric in
the uplink and downlink directions. For some versions of BBR, we set an \texttt
{fq} qdisc on the host nodes' egress interfaces for pacing.

\paragraph{Host configuration.}

We create simple HTTPS wrappers around each transport protocol implementation to
evaluate the congestion control schemes in \Cref{tab:cca-implementations}. The TCP
endpoints use the Python \texttt{http} module. For the QUIC implementations, we
modify comparable client/server applications from each repository.
%to transfer some number of bytes from server to client.
With the exception of enabling TCP pacing where required by BBR, we use
``default values'' for tunable parameters, considering these to be part of each
implementation.

In TCP, we set the CCA by using a specific Linux kernel
version and loading the congestion control module. We use the default
kernel's implementations of CUBIC and BBRv1, and Google's kernel fork for
BBRv2 and BBRv3 (\Cref{tab:cca-implementations}).

In QUIC, we simply select the CCA as a command-line
argument to the user space implementation. We evaluate Google \texttt
{quiche}~\cite{google-quiche}, Cloudflare \texttt{quiche}~\cite
{quiche}, and \texttt{picoquic}~\cite{picoquic}. Google and Cloudflare \texttt
{quiche} are their production implementations of the same name,
and \texttt{picoquic} is a minimalist implementation based on the IETF spec.
All three include CUBIC and BBRv1 implementations, as well as some form of
BBRv2 or BBRv3, which is still undergoing standardization.

For the transparent, connection-splitting TCP PEP, we use PEPsal~\cite
{caini2006pepsal}. PEPsal intercepts the SYN packet during the three-way
handshake and forms separate TCP connections with each endpoint,
copying data between the two sockets. Note that
discussions of split QUIC are based only on the split throughput heuristic
and do not use an actual splitter.

\paragraph{Experiment specification.}
In each experiment, the HTTPS client requests a specific number of bytes to be
transferred from the server in the HTTPS payload. This number corresponds to the
amount of data that can be transmitted through the bottleneck link over a
$10$-second period. We have found this to be sufficiently large to reflect
sustained throughput.

In practice, we report the "single-stream goodput" calculated as the number of
application-layer bytes received (excluding HTTPS headers) divided by request
completion time (from when the client sends a request to when it receives a
complete response). Due to header overhead and request latency, this metric
will never equal the link rate.
% \thea{could add, may not be necessary: A more precise measurement would use the
%  same application for each CCA and invoke lower-level APIs, but with large data
%  transfers, minor differences in client/server implementations will not impact
%  observed trends. }
% \thea{say something about how long the flow is? + why this is limited but still
%  useful}.

\paragraph{Machine specification.} All experiments use CloudLab~\cite
 {duplyakin2019design} x86\_64 rs630 nodes in the Massachusetts cluster running
 Ubuntu 22.04.2. The nodes use the default Linux kernel v5.15.0-122-generic,
 except for TCP BBRv2 and BBRv3.%, in which they use the Google fork of v6.4.0+.

%-------------------------------------------------------------------------------
\section{Results}
\label{sec:results}
%-------------------------------------------------------------------------------

\begin{figure*}[t!]
    \centering
    \begin{subfigure}[b]{0.22\linewidth}
        \centering
        \includegraphics[width=\linewidth,trim={0 0 2cm 0.7cm},clip]
         {splitting-paper/figures/heatmaps/heatmap_tcp_cubic_10mbps.pdf}
        \captionsetup{skip=4pt}
        \caption{TCP CUBIC.}
        \label{fig:2d-heatmap:cubic-10}
    \end{subfigure}
    \begin{subfigure}[b]{0.22\linewidth}
        \centering
        \includegraphics[width=\linewidth,trim={0 0 2cm 0.7cm},clip]
         {splitting-paper/figures/heatmaps/heatmap_tcp_bbr1_10mbps.pdf}
        \captionsetup{skip=4pt}
        \caption{TCP BBRv1.}
        \label{fig:2d-heatmap:bbr1-10}
    \end{subfigure}
    \begin{subfigure}[b]{0.22\linewidth}
        \centering
        \includegraphics[width=\linewidth,trim={0 0 2cm 0.7cm},clip]
         {splitting-paper/figures/heatmaps/heatmap_tcp_bbr2_10mbps.pdf}
        \captionsetup{skip=4pt}
        \caption{TCP BBRv2.}
        \label{fig:2d-heatmap:bbr2-10}
    \end{subfigure}
    \begin{subfigure}[b]{0.22\linewidth}
        \centering
        \includegraphics[width=\linewidth,trim={0 0 2cm 0.7cm},clip]
         {splitting-paper/figures/heatmaps/heatmap_tcp_bbr3_10mbps.pdf}
        \captionsetup{skip=4pt}
        \caption{TCP BBRv3.}
        \label{fig:2d-heatmap:bbr3-10}
    \end{subfigure}
    \begin{subfigure}[b]{1cm}
        \includegraphics[width=1cm,trim={8cm 0 0 0},clip]
         {splitting-paper/figures/heatmaps/heatmap_tcp_bbr3_10mbps.pdf}
    \end{subfigure}
    \caption{The link rate utilization calculated as the ratio of achieved
     goodput to link rate for various TCP CCAs in an emulated network path, at
     various loss rates and one-way delays, at 10 Mbit/s. CUBIC is the
     most sensitive to loss and delay, while BBRv1 is the most aggressive and
     achieves the highest utilizations; BBRv2 and BBRv3 are more sensitive to
     loss than BBRv1 and utilization starts to suffer (left to right). CCAs
     tend to achieve lower utilizations, though higher absolute goodputs, as
     the link rate increases (\Cref{sec:appendix:heatmaps}). Median of $n=20$ trials.}
    \label{fig:2d-heatmap}
\end{figure*}


In our emulation measurement study, we aim to answer three questions to gain a
more comprehensive understanding of the relevance of connection-splitting with
modern networks:

\begin{enumerate}[noitemsep]
    \item Has the model-based BBR algorithm made the throughput enhancements of
     connection-splitting obsolete?
    \item Are there classes of network settings where BBR benefits significantly
     from splitting but CUBIC does not?
    % \thea{I am still not totally convinced by this claim as a "headline" and
    %  think the network path "classes" are more interesting.}
    \item How does CCA implementation impact end-to-end behavior, and therefore split
     behavior?
\end{enumerate}

\noindent To recap our measurement methodology, we have a method for
 measuring the throughput of TCP connections in emulation both with and without a
 transparent PEP. We also have the ability to estimate throughput both with and
 without a generic connection splitter, for both TCP and QUIC, based on
 knowledge of the network model of each path segment, and the measured end-to-end
 throughput on each segment.

For our analysis, we cache measurements for the end-to-end network parameters
in \Cref{tab:parameters} and the congestion control schemes in \Cref
{tab:cca-implementations}. We also run experiments in the two-segment network topology
to validate some of these predictions. Raw data for the end-to-end
behavior of each CCA implementation are available in \Cref{sec:appendix}.
Our emulation benchmarks are also publicly available on GitHub: \url{https://github.com/StanfordSNR/connection-splitting}.

\subsection{Finding: Splitting has become significantly more beneficial to TCP
 BBR since it was initially released in 2016.}

Has the model-based BBR algorithm made the throughput enhancements of
connection-splitting obsolete? We find this line of thought has some truth
with the initial release of TCP BBRv1 in 2016. But as the BBR algorithm has
evolved into BBRv2 in 2019 and now BBRv3 in 2023, it has also evolved to behave
more conventionally in the sense that it benefits from being split---just like
more traditional, loss-based congestion control schemes such as CUBIC.

% \begin{table}[t!]
  \centering
\begin{tabular}{ r l l }
  \toprule
    \textbf{Fig.} & \textbf{Segment 1} & \textbf{Segment 2} \\
    \midrule
    \ref{fig:bbr-over-time:class1} & 1ms, 20 Mbit/s, 4\% & 100ms, 20 Mbit/s, 0\% \\
    \ref{fig:bbr-over-time:class2} & 100ms, 20 Mbit/s, 2\% & 1ms, 20 Mbit/s, 2\%\\
    \ref{fig:bbr-over-time:class3} & 40ms, 40 Mbit/s, 2\% & 40ms, 40 Mbit/s, 2\%\\
\bottomrule
\end{tabular}
  \caption{\label{tab:network-setting-params} The propagation delay, link rate,
   and random loss for each path segment of the three scenarios in \Cref
   {fig:bbr-over-time}. \gina{Need to make consistent the use of link rate vs.
   bandwidth etc.}}
\end{table}


\Cref{fig:bbr-over-time} shows three different network settings in which BBRv2
and BBRv3 show significant throughput gains from connection-splitting.
 When BBRv1 was released, its throughput both with and
 without the connection-splitter were roughly the same, and nearly achieved the
 bottleneck link rate. With BBRv2 and BBRv3, the end-to-end throughput
 drastically deteriorated, behaving more like CUBIC than BBRv1. However, the
 split throughput remained relatively high, suggesting that BBRv3 today would
 still benefit from connection-splitting.

\paragraph{Analysis.}
To explore why CUBIC and BBRv3 benefit from splitting but BBRv1 does not, we
analyze end-to-end throughputs of each CCA and apply the split throughput
heuristic (\Cref{fig:heuristic}). As described in \Cref{sec:heuristic}, this
methodology allows us to study split settings by measuring the much smaller
parameter space of end-to-end connections.
We find that connection-splitting is likely to improve throughput on lossy
paths for CUBIC, BBRv2, and BBRv3 connections, but not for BBRv1.

\Cref{fig:2d-heatmap} visualizes our cached end-to-end measurements as link rate
 utilization heatmaps of loss vs. delay.
 Here is an example for how to interpret these graphs using \Cref
 {fig:2d-heatmap:bbr3-10}. The top-left cell represents a 100 ms 0\% segment
 with 0.82 utilization of the 10 Mbit/s link rate, and the bottom-right cell
 represents a 1 ms 4\% segment with 0.85 utilization of the 10 Mbit/s link
 rate. The predicted split utilization of a network path composed of these two
 segments is just 0.82, the minimum. The two cells compose to the top-right
 cell, which represents a 100 ms 4\% 10 Mbit/s segment with an end-to-end
 utilization of 0.49. Since $0.82>0.49$, we say that splitting has improved the
 throughput of this network path.

BBRv1 achieves high link rate utilization in all settings (\Cref
{fig:2d-heatmap:bbr1-10}), showing it has little to gain from splitting.
In fact, previous studies have shown that BBRv1 achieves $\approx\!85\%$
link utilization regardless of loss (same as our findings) before it reaches
a cliff point at around 20\% loss~\cite{cao2019use,cardwell2017bbr}.
This may be why it appears that splitting in lossy settings is now obsolete with BBR.
However, one should note that BBRv1's high throughput has long been attributed
to its aggressiveness and unfairness to legacy algorithms~\cite
{ware2019modeling,cao2019use}, which is what led to the changes in BBRv2 and
BBRv3.

While BBRv2 is a large departure from BBRv1, BBRv3 has been described as BBRv2
with bugfixes and performance tuning~\cite{cardwell2024bbrv3-ietf119}, which
supports why the two are so similar. We focus on BBRv3 since Google hopes to
now deprecate BBRv2~\cite{cardwell2024bbrv3-ietf119}.

% Unlike BBRv1, BBRv3 uses loss as a congestion signal. Indeed, BBRv3's end-to-end
% throughput suffers under high loss (\Cref{fig:2d-heatmap:bbr3-10}), though to a
% lesser extent than CUBIC (\Cref{fig:2d-heatmap:cubic-10}). Our results reflect
% existing findings of lower utilization in BBRv2 and BBRv3~\cite
% {datta2023replication,song2021understanding,zeynali2024promises}.
BBRv3 is more sensitive to loss than its previous versions (\Cref
{fig:2d-heatmap:bbr3-10}), and more similar to CUBIC (\Cref
{fig:2d-heatmap:cubic-10}) in that there exist scenarios where end-to-end
throughput suffers.
These reflect existing findings of lower utilization in BBRv2 and BBRv3~\cite
{datta2023replication,song2021understanding,zeynali2024promises}.
Based on the heuristic, it is clear that in some lossy networks
such as the ones empirically evaluated in \Cref{fig:bbr-over-time},
connection-splitting
can significantly increase the throughput of BBRv3 connections.
It is possible that the BBR algorithm continues to evolve in this direction
given that BBR's unfairness remains contentious today~\cite
{datta2023replication,zeynali2024promises}.

\paragraph{Summary.}

While BBR may not have benefited from splitting with the release of ``v1'' in 2016, BBRv2 and
now BBRv3 have evolved to behave more conventionally---similar to traditional,
loss-based CCAs such as CUBIC---in the sense that they \textit{do}. Even
so-called ``model-based'' congestion control algorithms seem to now react to
loss as a congestion signal, as the BBR algorithm continues to evolve.

\subsection{Finding: There exist classes of network paths where TCP BBRv3 would
 significantly benefit from splitting but TCP CUBIC would not.}

Are there new classes of network settings where BBR benefits significantly from
splitting but CUBIC does not? We want to understand the network settings in
which a congestion control scheme is not able to achieve practical bottleneck
link rate utilizations end-to-end, but is with a connection-splitter.

We find that while splitting benefits BBRv3 in all the same scenarios as CUBIC,
it also has the potential to benefit BBRv3 in many \textit{new} scenarios.
In addition to edge deployments with a lossy last-mile, BBRv3
also benefits from splitting in scenarios where there is loss on both path
segments, and when the connection-splitter is located farther from the edge.
We partition these scenarios into three classes of network settings:

\begin{enumerate}[label=\Roman*.,noitemsep]
\item Paths with asymmetric delay and a lossy last-mile,
\item Lossy paths with asymmetric delay,
\item Lossy paths with more symmetric delay.
\end{enumerate}

\noindent \Cref
 {fig:bbr-over-time:class1,fig:bbr-over-time:class2,fig:bbr-over-time:class3}
represent network settings in each class, respectively.
``Asymmetric'' refers to the delays on the two path segments.
These emulations empirically demonstrate that CUBIC only benefits in the first
class, while BBRv3 benefits in all three.
BBRv1 does not need splitting in any context.

\paragraph{Analysis.}
To identify which network paths benefit from connection-splitting and where
along the paths PEPs should be deployed, we apply the
heuristic (\Cref{fig:heuristic}).
For each CCA, we conduct an exhaustive search of the $15 \cdot 21 \cdot 25 =
7875$ combinations of settings within our parameter space (\Cref
{tab:parameters}), and efficiently predict the split and end-to-end throughputs.

We filter on the predicted throughputs for network settings where splitting improves end-to-end throughput by
at least $3\times$, and where the split connection utilizes at least half the
bottleneck link rate (\Cref{tab:network-path-analysis}). BBRv1 does not meet
these criteria is any scenarios, and the theoretical connection-splitter is
unable to improve the throughput of BBRv1 by even $50\%$. CUBIC and BBRv3 meet
these criteria in 942 and 188 scenarios, respectively. CUBIC benefits from
splitting in more scenarios because its end-to-end utilization is more
frequently low, so it more frequently has a large split improvement.

\begin{table}[t!]
  \centering
\begin{tabular}{ r l l l }
  \toprule
    \textbf{Filter} & \textbf{BBRv1} & \textbf{CUBIC} & \textbf{BBRv3} \\
    \midrule
    Initial & 7875 & 7875 & 7875 \\
    Split imprvmnt. $>3\times$ & 0 & 2231 & 234 \\
    Split utilization $>0.5$ & 0 & 942 & 188 \\
    \midrule
    Asymmetric, last-mile & 0 & 942 & 38 \\
    Asymmetric, lossy & 0 & 0 & 72 \\
    Symmetric, lossy & 0 & 0 & 78 \\
    \bottomrule
\end{tabular}
  \caption{\label{tab:network-path-analysis} An exhaustive search of network
   paths and their PEP locations that benefit from splitting for each CCA, and
   the number of filtered settings that belong to each class.}
  \vspace{-0.4cm}
\end{table}


Since the distribution of network paths in our parameter space does not reflect
any meaningful real world distribution, we are more interested in the \textit
{classes} of network settings that benefit from splitting. We realized
that \textit{all} of the relevant network settings for CUBIC can be clustered
into Class I, as network paths where one path segment has $1$ ms delay and
non-zero loss, and the other has $>1$ ms delay and $0\%$ loss. However, Class
I only accounts for $21\%$ of the relevant network settings for BBRv3. We
identify Class II, which is the same as I, except both path segments have
non-zero loss. Class III is the same as II, except both path segments have
$>1$ ms delay. We used the results to select three representative network
settings to empirically evaluate in \Cref{fig:bbr-over-time}.

Intuitively, we can understand why BBRv3 benefits more from splitting in lossy
scenarios than CUBIC based on how it reacts to loss and delay (\Cref
{fig:2d-heatmap:bbr3-10}). BBRv3's sensitivity to loss
and delay is more gradual than abrupt, so it is more likely to benefit when
splitting a lossy, high-delay network path in any way. In comparison, CUBIC's
throughput falls off a cliff for many of these segmentations.

\paragraph{Discussion.}
Do these results reflect where PEP deployments have been useful in the real
world? Connection-splitters have traditionally been found in satellite,
cellular, and Wi-Fi networks with a wireless link or rate policer~\cite
{edeline2019bottomup,honda2011still}. This resembles Class I, where the
network path consists of a lossy last-mile, and a reliable Internet path
segment. It makes sense then for PEPs to be traditionally located at the edge
to address the issues of loss-based schemes~\cite
{cloudsplitting2010,rfc3135,farkas2012splittcp}. We expect these PEPs to
similarly benefit BBRv3 in the same locations.

In Classes I and II, asymmetric delay can be severe in low-resource networks
in addition to wireless last-mile links;
Consider, for example, regions with no IXPs in which a significant proportion
of Internet traffic travels internationally.

For Classes II and III, satellite (and also wireless ad-hoc)
networks are known for
having lossy ``middle-miles''~\cite{kuhn2021quic-over-sat,border2022evaluating,pirovano2013new,cloudsplitting2010}.
This can be due to bad weather, fast-moving satellites, and
long-distance radio waves, etc.
Since CUBIC does not trivially benefit from splitting in these scenarios, the
traditional solution has been to split the connection at multiple points and to
use an FEC-based or other proprietary protocol in the satellite backhaul~\cite
{cloudsplitting2010,border2022evaluating,rfc3135}.
Our results suggest such an invasive solution may not be necessary for BBRv3.

How could this inform the deployment of connection-splitting PEPs?
With the caveat that futher exploration is required to understand how the
heuristic extrapolates to the real world, one idea is to
determine where to deploy PEPs along an existing network path for maximum
benefit especially as congestion-control schemes evolve.
Another idea is given the location of a PEP, determine which connections going
through the PEP would most benefit based on knowledge of each connection's
network path.
We leave network operators to decide how best to model their networks and apply
the heuristic to evaluate potential PEP deployments.

% \gina{Expand on the impact of these findings especially for new types of GEO and
%  LEO satellite networks. I think cellular network operators will be skeptical of the
%  relevance of these findings (they don't really believe in loss), but satellite
%  network operators may be more sympathetic.}

\paragraph{Summary.}
While TCP CUBIC only benefits from connection-splitting when the PEP is located
at the lossy last-mile, TCP BBRv3 can also benefit when there is loss on both
sides of the PEP and when there are longer propagation delays. This suggests
that TCP connections using BBRv3 should benefit from splitters at the same
locations as for CUBIC, and also that traditional methods used to
address loss in the middle-mile could use simple connection-splitting instead.
We believe this method of analyzing useful network paths and PEP placements for
connection-splitting can be extended to model new types of networks, especially
in space.

% \subsection{Finding: BBR's end-to-end behavior depends on implementation, and
%  evaluations of connection splitting should likewise take implementation into
%  account.}
\subsection{Finding: QUIC implementations of the same congestion control schemes
 vary significantly, and further differ from Linux's TCP implementations.}

\begin{figure*}[t!]
    \centering
    \begin{subfigure}[b]{0.22\linewidth}
        \centering
        \includegraphics[width=\linewidth,trim={0 0 2cm 0.7cm},clip]
        {splitting/figures/heatmaps/heatmap_tcp_bbr3_10mbps.pdf}
        \captionsetup{skip=4pt}
        \caption{TCP, BBRv3}
        \label{fig:splitting:quic:tcp-bbr3}
    \end{subfigure}
    \begin{subfigure}[b]{0.22\linewidth}
        \centering
        \includegraphics[width=\linewidth,trim={0 0 2cm 0.7cm},clip]
        {splitting/figures/heatmaps/heatmap_quic_bbr3_10mbps.pdf}
        \captionsetup{skip=4pt}
        \caption{Google \texttt{quiche}, BBRv3}
        \label{fig:splitting:quic:google-bbr3}
    \end{subfigure}
    \begin{subfigure}[b]{0.22\linewidth}
        \centering
        \includegraphics[width=\linewidth,trim={0 0 2cm 0.7cm},clip]
        {splitting/figures/heatmaps/heatmap_quiche_bbr2_10mbps.pdf}
        \captionsetup{skip=4pt}
        \caption{Cloudflare \texttt{quiche}, BBRv2}
        \label{fig:splitting:quic:cloudflare-bbr2}
    \end{subfigure}
    \begin{subfigure}[b]{0.22\linewidth}
        \centering
        \includegraphics[width=\linewidth,trim={0 0 2cm 0.7cm},clip]
        {splitting/figures/heatmaps/heatmap_picoquic_bbr3_10mbps.pdf}
        \captionsetup{skip=4pt}
        \caption{\texttt{picoquic}, BBRv3}
        \label{fig:splitting:quic:picoquic-bbr3}
    \end{subfigure}
    \begin{subfigure}[b]{1cm}
        \includegraphics[width=1cm,trim={8cm 0 0 0},clip]
        {splitting/figures/heatmaps/heatmap_tcp_bbr3_10mbps.pdf}
        \vspace*{0.2cm}
    \end{subfigure}
    
    \begin{subfigure}[b]{0.22\linewidth}
        \centering
        \includegraphics[width=\linewidth,trim={0 0 2cm 0.7cm},clip]
        {splitting/figures/heatmaps/heatmap_tcp_bbr1_10mbps.pdf}
        \captionsetup{skip=4pt}
        \caption{TCP, BBRv1}
        \label{fig:splitting:quic:tcp-bbr1}
    \end{subfigure}
    \begin{subfigure}[b]{0.22\linewidth}
        \centering
        \includegraphics[width=\linewidth,trim={0 0 2cm 0.7cm},clip]
        {splitting/figures/heatmaps/heatmap_quic_bbr1_10mbps.pdf}
        \captionsetup{skip=4pt}
        \caption{Google \texttt{quiche}, BBRv1}
        \label{fig:splitting:quic:google-bbr1}
    \end{subfigure}
    \begin{subfigure}[b]{0.22\linewidth}
        \centering
        \includegraphics[width=\linewidth,trim={0 0 2cm 0.7cm},clip]
        {splitting/figures/heatmaps/heatmap_quiche_bbr1_10mbps.pdf}
        \captionsetup{skip=4pt}
        \caption{Cloudflare \texttt{quiche}, BBRv1}
        \label{fig:splitting:quic:cloudflare-bbr1}
    \end{subfigure}
    \begin{subfigure}[b]{0.22\linewidth}
        \centering
        \includegraphics[width=\linewidth,trim={0 0 2cm 0.7cm},clip]
        {splitting/figures/heatmaps/heatmap_picoquic_bbr1_10mbps.pdf}
        \captionsetup{skip=4pt}
        \caption{\texttt{picoquic}, BBRv1}
        \label{fig:splitting:quic:picoquic-bbr1}
    \end{subfigure}
    \begin{subfigure}[b]{1cm}
        \includegraphics[width=1cm,trim={8cm 0 0 0},clip]
        {splitting/figures/heatmaps/heatmap_tcp_bbr1_10mbps.pdf}
        \vspace*{0.2cm}
    \end{subfigure}

    \begin{subfigure}[b]{0.22\linewidth}
        \centering
        \includegraphics[width=\linewidth,trim={0 0 2cm 0.7cm},clip]
        {splitting/figures/heatmaps/heatmap_tcp_cubic_10mbps.pdf}
        \captionsetup{skip=4pt}
        \caption{TCP, CUBIC}
        \label{fig:splitting:quic:tcp-cubic}
    \end{subfigure}
    \begin{subfigure}[b]{0.22\linewidth}
        \centering
        \includegraphics[width=\linewidth,trim={0 0 2cm 0.7cm},clip]
        {splitting/figures/heatmaps/heatmap_quic_cubic_10mbps.pdf}
        \captionsetup{skip=4pt}
        \caption{Google \texttt{quiche}, CUBIC}
        \label{fig:splitting:quic:google-cubic}
    \end{subfigure}
    \begin{subfigure}[b]{0.22\linewidth}
        \centering
        \includegraphics[width=\linewidth,trim={0 0 2cm 0.7cm},clip]
        {splitting/figures/heatmaps/heatmap_quiche_cubic_10mbps.pdf}
        \captionsetup{skip=4pt}
        \caption{Cloudflare \texttt{quiche}, CUBIC}
        \label{fig:splitting:quic:cloudflare-cubic}
    \end{subfigure}
    \begin{subfigure}[b]{0.22\linewidth}
        \centering
        \includegraphics[width=\linewidth,trim={0 0 2cm 0.7cm},clip]
        {splitting/figures/heatmaps/heatmap_picoquic_cubic_10mbps.pdf}
        \captionsetup{skip=4pt}
        \caption{\texttt{picoquic}, CUBIC}
        \label{fig:splitting:quic:picoquic-cubic}
    \end{subfigure}
    \begin{subfigure}[b]{1cm}
        \includegraphics[width=1cm,trim={8cm 0 0 0},clip]
        {splitting/figures/heatmaps/heatmap_tcp_cubic_10mbps.pdf}
        \vspace*{0.2cm}
    \end{subfigure}

    \caption{Heatmaps for three QUIC implementations of BBRv3 (or BBRv2), BBRv1,
     and CUBIC showing link rate utilization calculated as the ratio of
     achieved goodput to link rate, compared to Linux TCP. The heatmaps are
     shown at various loss rates and one-way delays with a fixed link rate of
     10 Mbit/s. User-space QUIC is not CPU-limited, achieving high utilizations
     at 1 ms delay and 0\% loss. The QUIC implementations are Google \texttt
     {quiche}, Cloudflare \texttt{quiche}, and a minimalist implementation
     based on the IETF spec called \texttt{picoquic}. Median of $n=20$
     trials.}
    \label{fig:splitting:quic}
\end{figure*}


How does CCA implementation impact end-to-end behavior, and therefore split
behavior? Might claims about ``split throughput'' depend not just on the CCA,
but also the \textit{implementation} and/or the transport protocol
on top of it?

For our initial study, we compared end-to-end throughput for four open-source
implementations each of CUBIC, BBRv1, and BBRv2/3; one using TCP and three
using QUIC. We find that the end-to-end behavior of each CCA varies by
implementation in both baseline throughput and sensitivity to loss and delay.
To evaluate the split behavior of QUIC, instead of creating a custom and
explicit connection-splitting PEP for each QUIC implementation, we apply the
split throughput heuristic and argue that these implementations
will likewise respond differently to connection-splitting PEPs.

% We think this is important to answer with the rise of different QUIC
% implementations~\cite{marx2020same}, as it remains contentious whether
% encrypted protocols should receive in-network assistance~\cite
% {yuan2024sidekick}.
% This makes it frustrating to directly compare QUIC and TCP.

\Cref{fig:quic} visualizes the end-to-end behavior of these schemes.
We highlight that the Cloudflare and \texttt
 {picoquic} CUBIC implementations are less sensitive to loss than Google or
 TCP; the former may benefit from splitting in more classes of network settings
 than the latter. Additionally, their BBR implementations exhibit non-uniform
 behavior, suggesting a non-uniform response to
 connection-splitting. These variations indicate that the benefits of
 in-network assistance should be considered along with not just the CCA but its
 specific implementations.

% We believe it is important to consider CCA \textit{implementation}, along with
% CCA choice, application-level behavior~\cite{philip2024prudentia}, transport
% protocol mechanisms~\cite{marx2020same}.

\paragraph{Analysis.}

The Google QUIC (\Cref{fig:quic:google-bbr1,fig:quic:google-bbr3,fig:quic:google-cubic})
and TCP (\Cref{fig:quic:tcp-bbr1,fig:quic:tcp-bbr3,fig:quic:tcp-cubic})
implementations are most similar to each other for each CCA.
This is reasonable if we take the community-based CUBIC implementation in
Linux to be the standard, and considering that Google contributed to the Linux
BBR implementations. In general, Google QUIC achieves slightly higher
utilization than Linux TCP.

The Cloudflare QUIC BBR implementations (\Cref{fig:quic:cloudflare-bbr1,fig:quic:cloudflare-bbr2})
exhibit profoundly different behavior from Linux TCP in baseline performance.
Note that Cloudflare uses BBRv1 for TCP but their use of BBR in QUIC
is experimental. Anecdotally, a
Cloudflare employee has expressed difficulty making
their BBR implementation performant, having to reverse engineer
the Linux kernel~\cite{cardwell2024bbrv3-ietf119-qna}. Given the wide adoption
of BBRv1 for TCP at many CDNs~\cite{ware2024ccanalyzer}, we expect it to
be desirable yet challenging for these same companies to correctly incorporate
BBRv3 into their QUIC stacks in the coming years.
% Maybe cite narayan2018restructuring

The \texttt{picoquic} BBR implementations (\Cref{fig:quic:picoquic-bbr1,fig:quic:picoquic-bbr3})
are more similar to Linux TCP, although its BBRv3 implementation seems to have
a contradicting reaction to delay.
We note that \texttt{picoquic} is intended for experimental use in the
IETF~\cite{picoquic}. It is important then to understand its congestion control
behavior if it is to be used to evaluate
IETF proposals. We believe its differences from Linux TCP warrant further
exploration, but perhaps also that the ongoing standardization efforts of BBR
in the IETF~\cite{cardwell2024bbr-ietf-draft} indicate
that there is no monolith yet of ``the BBR algorithm.''

The Cloudflare QUIC and \texttt{picoquic} implementations of CUBIC
(\Cref{fig:quic:cloudflare-cubic,fig:quic:picoquic-cubic}) interestingly
both exhibit a more gradual degradation in response to loss and delay than Linux TCP
(\Cref{fig:quic:tcp-cubic}). We
find this harder to explain, given that TCP CUBIC has been around since 2006,
and perhaps can be attributed to transport protocol mechanisms in QUIC.
Nevertheless, this indicates that it is important to understand the
behavior of a CCA in the context of its entire
implementation.

\begin{figure}[t!]
    \centering
    \begin{subfigure}[b]{0.49\linewidth}
        \centering
        \includegraphics[width=\linewidth]
         {splitting/figures/network_path_analysis_10_40_1_80_2_2.pdf}
        \captionsetup{skip=0pt}
        \caption{Some QUIC CUBIC implementations can benefit in new network
         classes where TCP CUBIC could not.}
        \label{fig:splitting:quic-predictions:cubic}
    \end{subfigure}
    \begin{subfigure}[b]{0.49\linewidth}
        \centering
        \includegraphics[width=\linewidth]
         {splitting/figures/network_path_analysis_40_50_20_1_0_4.pdf}
        \captionsetup{skip=0pt}
        \caption{BBRv3 implementations have non-uniform end-to-end
         behavior and no clear resulting split behavior.}
        \label{fig:splitting:quic-predictions:bbr3}
    \end{subfigure}
    \caption{Predicted bottleneck link rate utilizations calculated from the
     predicted end-to-end and split throughputs of the TCP and QUIC
     implementations, on two different network path segments. End-to-end
     behavior of each CCA varies significantly by implementation.}
    \label{fig:splitting:quic-predictions}
\end{figure}


\Cref{fig:quic-predictions} applies the heuristic to show how split behavior
can vary for CCA implementations with different end-to-end behaviors.
Some QUIC CUBIC implementations
benefit in new network settings where TCP CUBIC does not, while the various
QUIC BBRv3 implementations exhibit non-uniform end-to-end behavior and thus no
clear trend in split behavior.

% On the other hand, Cloudflare's implementations exhibit profoundly different
% behavior. Cloudflare's quic-CUBIC implementation appears more aggressive than
% those in Linux or Chromium. As expected, quic-BBRv1 and v2 are less responsive
% to loss than CUBIC, but goodputs trend \textit{lower} than in quic-CUBIC. This
% may reflect that Cloudflare's quic-BBR implementations are relatively new \cite
% {} and, as far as we are aware, not yet deployed in production \cite{}. \thea
% {Could add observation of the BBR code possibly introducing some additional
% loss, but we confirmed no full send buffers or loss on interfaces.}

\paragraph{Discussion.}

Why do we believe we can extrapolate split behavior from the end-to-end behavior
of QUIC? Previous studies explore the effects of QUIC's transport protocol
mechanisms in such a case~\cite{kosek2022quicpep,thomas2019google}. They find
that the effects of zero-RTT connection establishment with regards to
long-lived throughput to be minimal, and stream multiplexing to be mutually
beneficial in both scenarios. Further studies can clarify the interactions
between CCAs and transport mechanisms.

How do we know that the difference in behavior is due to the congestion control
implementation and not other transport protocol mechanisms? Well, we
don't, and there is known to be significant variance in the features supported
by different QUIC implementations~\cite{marx2020same}.
However, we find it intuitive that congestion control would be a major factor in
the measured sustained goodput of a bulk file transfer.

The divergence in QUIC implementations is not so dissimilar from that of TCP at
a similar state of evolution~\cite{allman1999effective}, but there are still
some fundamental differences. With QUIC implementations in userspace, there is
a larger diversity of implementations that are easier to tune for a specific
application metric, as opposed to correctness or fairness from a
congestion-control point of view. These algorithms are also highly
parameterized, with no standard nor test suite, so it's not surprising that the
implementations differ.

If there is reason to believe
that QUIC is losing out on throughput by not connection-splitting, then there
will continue to be research on how to achieve the same benefits without
ossification~\cite{kosek2023secure,yuan2024sidekick,kramer2021masquepep,yuan2022sidecar}.
The heuristic helps us understand the theoretical achievable
throughput with a simple connection-splitter, or even by combining multiple
end-to-end congestion control schemes.
In addition to research, this could motivate privacy-minded proposals in the Internet
standards community to also view themselves as potential deployment opportunities
for private PEPs~\cite{kosek2021masque,sattler2022towards,rfc9297,rfc9298}.
% For example, MASQUE is an explicit
% connection-splitter~\cite{kosek2021masque}, and PACUBIC is a path-aware
% congestion control algorithm that uses signals provided by network
% intermediaries in the Sidekick protocol~\cite{yuan2024sidekick}.
% While we do not think QUIC will ever have transparent connection-splitting
% proxies due to ossification, these proposals demonstrate that other methods of
% in-network assistance are possible.

% The analysis makes two assumptions: that congestion control is the major factor
% in explaining differences in end-to-end behavior, and that the congestion
% control and not other transport protocol mechanisms are the most important
% factors in extrapolating split performance.

% \begin{figure}[t!]
    \centering
    \includegraphics[width=\linewidth,height=100pt]{example-image-a}
    \caption{BBRv1 performance on 13 Linux kernels between 2016 and 2024.}
    \label{fig:bbr1-over-time}
\end{figure}


Another application of analyzing CCA implementations is to analyze the Linux TCP
implementations of BBR \textit{within} a major version over time. We did
this analysis with BBRv1 on 13 Linux kernels between 2016 and 2024, but found
no significant variance. However, given the dynamic nature of BBR and the
Linux networking stack, it
is possible there will still be changes to the split behavior of TCP in the future.

% \thea{Not sure where this belongs?} Recognizing that there have been changes to
%  both BBRv1 and the Linux networking stack \cite{} since 2016, we repeated
%  BBRv1 experiments in 13 Linux kernels between 2016 and 2024. We confirm that
%  BBRv1 does not meaningfully benefit from connection-splitting in any release
%  or network emulation. There are subtle differences in earlier versions
%  (e.g., higher standard deviation across multiple trials), but the irrelevance
%  of connection splitting is the same. can be used to evaluate the cca in the
%  past (and in the future). Plot the splittability of the congestion control
%  algorithms over time and how different iterations of it look.

\paragraph{Summary.}

We believe it is important to understand the end-to-end behavior of a congestion
control scheme in the context of its entire implementation.
Our results suggest that BBR is challenging to implement, and that even CUBIC
implementations can vary based on context. Whether the prevailing wisdom is
that a specific CCA or transport protocol has made in-network assistance
undesirable, these results suggest that it is valuable to consistently
re-evaluate these claims.

%-------------------------------------------------------------------------------
\section{Accuracy Analysis}
\label{sec:accuracy}
%-------------------------------------------------------------------------------

\begin{figure*}[t]
    \centering
    \begin{subfigure}[b]{0.325\linewidth}
        \centering
        \includegraphics[width=\linewidth,trim={0 0 25.8cm 0},clip]
         {figures/accuracy/accuracy_e2e_without_loss.pdf}
        \includegraphics[width=\linewidth,trim={0 0 25.8cm 0},clip]
         {figures/accuracy/accuracy_e2e_with_loss.pdf}
        \captionsetup{skip=4pt}
        \caption{End-to-end throughput accuracy.}
        \label{fig:accuracy:e2e}
    \end{subfigure}
    \begin{subfigure}[b]{0.645\linewidth}
        \centering
        \includegraphics[width=\linewidth,trim={0 0 13.4cm 0},clip]
         {figures/accuracy/accuracy_split_without_loss.pdf}
        \includegraphics[width=\linewidth,trim={0 0 13.4cm 0},clip]
         {figures/accuracy/accuracy_split_with_loss.pdf}
        \captionsetup{skip=4pt}
        \caption{Split throughput accuracy.}
        \label{fig:accuracy:split}
    \end{subfigure}

    \caption{Heatmaps of the measured and predicted BBRv3 throughputs for
     various splits of delay, bandwidth, and loss, both without (top) and with
     (bottom) loss.
     The \textit{end-to-end throughput} predictions (not pictured) are the same for all
     cells at 0.86 utilization without loss and 0.58 utilization with loss,
     because they represent the same network path, so end-to-end
     prediction errors are roughly uniform.
     The \textit{split throughput} predictions err slightly on the side of
     overestimation, but they accurately reflect trends in higher or lower throughputs for
     measurements on different splits of the same network path. Median of $n=40$ trials.}
    \label{fig:accuracy}
    \vspace{-0.3cm}
\end{figure*}


Our analysis of connection-splitting for TCP and its extensions to QUIC rely on
the accuracy of the split throughput heuristic. In this section, we are
primarily concerned with how accurate our predictions, which are based on
measurements from a one-segment topology, are for measurements from a
two-segment topology in emulation, without and with a connection-splitting TCP
PEP:
\begin{itemize}[noitemsep]
\item Does the \texttt{compose} function accurately represent the combined
 network path in the end-to-end throughput?
\item Does the split throughput heuristic accurately predict split throughput?
\end{itemize}

\noindent Most importantly, we find the heuristic to be able to usefully
 predict \textit{trends}.
 In terms of absolute predictions, we find the \texttt{pred\_e2e\_throughput()}
 and \texttt{pred\_split\_throughput()} functions to be correct within a
 reasonable tolerance, with a slight tendency to overestimate.
 % the \texttt{pred\_e2e\_throughput()} function enables
 % us to predict emulated end-to-end throughput within $7$\% and the \texttt
 % {pred\_split\_throughput()} function within $14$\%.

Orthogonally, we do not evaluate the accuracy to which emulation studies reflect
the real world with multi-flow settings and more complex network properties,
nor how the accuracy would extrapolate to QUIC connections with custom
connection splitters. It may be interesting to explore how to incorporate
such factors into the network model and heuristic.

\paragraph{Methodology.} Recall that for a given network path composed of two
 path segments, we can obtain both the predicted end-to-end and split
 throughputs, and the ground truth throughputs in an emulated network with and
 without a TCP PEP. Then for a network setting, we can compute the accuracy as
 the percent error in predicted vs. measured throughput.

We perform an empirical accuracy analysis of BBRv3 for two end-to-end network paths with
identical bandwidth and delay, both without (0\%) and with (4\%) loss. We test
various splits for the bandwidth, delay, and loss and analyze the accuracy
trends. In particular, we select delay splits such that the end-to-end delay is
80 ms, bandwidth splits such that the bottleneck bandwidth is 10 Mbit/s, and loss
splits such that the total loss is either 0\% or 4\%. We parameterize the network
path segments to use the cached measurements from \Cref
{tab:parameters}.

\subsection{End-to-End Throughput Accuracy}

Each of our splits composes to the same end-to-end network path, so we predict
the same end-to-end throughput for each. Our experimental
results in \Cref{fig:accuracy:e2e} show that the measured end-to-end
throughputs are also roughly uniform, especially without loss, indicating that
our method of composing network path segments in emulation represents the same
end-to-end network path.

\subsection{Split Throughput Accuracy}

The split throughput predictions accurately reflect trends in loss, delay, and
bandwidth (\Cref{fig:accuracy:split}). For example, split throughput is
generally higher when the high-bandwidth link is paired with high delay
(the yellow-est cells). It is also lower when the lossy link is paired with
high delay (columns 1 and 2) or low bandwidth (columns 3-5).

The split throughput predictions tend to slightly overestimate, but we think
the level of error is small enough to be helpful for informing real PEP
deployment. The maximum error is $\pm14\%$, and on average $\pm4\%$. This
dwarfs the measured gains in some situations, and rules out a large gain in
others.

One factor that may lead to overestimation of split throughput is the queue
behavior and the burstiness of the sender. With small queues and bursty
sending, we would expect the far path segment from the data sender to sometimes
be limited by the send buffer. This could subsequently affect how the far
connection probes for and utilizes available link rate capacity.

Another factor is the proximity of the bottleneck link to the sender. While
our heuristic does not account for this, the real split throughputs for
symmetric pairs of network paths (e.g., the left three and right three columns
in \Cref{fig:accuracy:split}), show that we slightly overestimate
when the low-bandwidth bottleneck link is far from the sender.
Based on our reasoning about queues, this would suggest that the far path
segment, already the bottleneck, is even further under-saturated.

Overall, we find our end-to-end and split throughput predictions to usefully
reflect relative trends and absolute throughput within a
reasonable tolerance. They are not intended to make claims about exact achievable
throughputs nor about the immediate utility of PEPs in the real world,
but simply to reason about how connection-splitting may impact
long-lived throughput in a simplified network model.

% \subsection{Possible Sources of Error}

% \gina{TODO: Simple experiment exploring the effects of queue discipline?}

% \paragraph{Modeling the network queue.} The bottleneck throughputs of each path
%  segment are taken assuming that the data sender of each TCP connection has
%  enough data to continuously saturate the connection when data needs to be
%  sent. Thus the throughput at the far path segment might be lower than
%  predicted when the queue isn't saturated. This could be if the queue is small
%  and drops packets when there are bursts.

% Reasons for overestimation would be, the effective RTT increases if there is
% buildup in the queue. Some algorithms aim to have some packets in the queue at
% any time. Bandwidth differences affect burstiness without pacing.

% \begin{figure}
%     \centering
%     \includegraphics[width=\linewidth,height=100pt]{figures/figure11a.png}
%     \includegraphics[width=\linewidth,height=100pt]{figures/figure11b.png}
%     \caption{Effect of different queue disciplines on BBRv3's end-to-end and
%      split throughput in a specific network setting. Or pick whichever one
%      showed differences. Sometimes short queues have an effect. \gina{I want to
%      run this experiment again I think it could be totally different with the
%      new topology, these results did not totally make sense anyway.}}
%     \label{fig:qdiscs}
% \end{figure}

% \paragraph{Queueing discipline.} We analyze the effect of different queueing
%  configurations on the heuristic as a possible explanation for these errors
%  (\Cref{fig:qdiscs}). Consider several different queue disciplines: RED, PIE,
%  droptail with a large buffer of 1 BDP, droptail with a small buffer of
%  $0.1\cdot$BDP. We also consider traffic policers that don't have a queue at
%  all.

% \subsection{Limitations}

% We can also add different representations of loss and how to compose them.

% We recognize that real split TCP deployments may use even more path segments,
% and the network model does not capture many of the nuances of real networks,
% and that queues may be configured with other behavior such as drop-tail at
% proxies.

% TCP is also not limited to bulk data transfers, and is used in a variety of
% settings where it is valuable to have reliable, ordered delivery. TCP PEPs can
% be useful in other ways like faster retransmission for low jitter. We hope the
% understanding of how these network conditions and this setup on congestion
% control behavior can help the reader extrapolate the effects of split TCP PEPs
% in their own deployments.

% %-------------------------------------------------------------------------------
\section{Discussion}
%-------------------------------------------------------------------------------
Another is thinking about future implications for encrypted transport protocols.
As transport protocols like QUIC develop is it bad if we can't split their
connections given how much performance can benefit?

Fairness, discussion on while ISPs have deployed these there hasn't been much
analysis into its fairness. In particular, relative to end-to-end congestion
control algorithms.
It is known that connection-splitting PEPs improve performance for that
connection, but what is the implication of its performance on competing flows?
But how much bandwidth does it save by reducing retransmissions by buffering at
the proxy?
% %-------------------------------------------------------------------------------
\section{Related Work} \label{sec:related}
%-------------------------------------------------------------------------------

Other measurement studies characterize the existence of PEPs, mostly harmful
ones. It is challenging to measure the benefits in the wild due to the
transparent nature of the proxies and because of varying network conditions.
One could run an encrypted transport protocol alongside TCP, but that adds
confounding factors such as the implementation of the protocol. Even with this,
some TCP PEP deployments, such as in satellite networks, will ban unknown
protocols completely due to their known performance issues. This emulation
study provides a characterization of the real world.

While PEPs have been used to improve performance, their application to encrypted
transport protocols such as QUIC is an area of active research. Snoop, Zombie,
Sidekick. Applied to specific scenarios and none as generalizable as the
connection-splitting TCP PEP. PEP-DNA.
\chapter{Conclusion}
\section{Summary}
\section{Future Work}

\subsection{Real-World Studies}
\subsection{Deployment and Standardization}
\subsection{Set Reconciliation in Packet-Scale Settings}

\section{Concluding Remarks}
...
\section*{Acknowledgments}

% Keith
% NSF grants 2045714 and 2039070, DARPA contract HR001120C0107, and a Sloan
% Research Fellowship, Google, Huawei, VMware, Dropbox, Amazon, and Meta
% Platforms.
% Michael Welzl (early), and (later) Frode Kileng and the members of the IRTF
% Path-Aware Networking Research Group.

% Thea
% I'm on Stanford SoE fellowship. Do you know if there's anything to acknowledge
% there? (officially it's Charles M. Pigott and NVIDIA-TSMC fellowships)

% Matei
% This research was supported in part by affiliate members and other supporters of
% the Stanford DAWN project, including Meta, Google, and VMware, as well as Cisco
% and SAP. Any opinions, findings, and conclusions or recommendations expressed
% in this material are those of the authors and do not necessarily reflect the
% views of the sponsors.

We thank Michael Welzl, our shepherd Jon Crowcroft, and the anonymous USENIX
ATC '25 reviewers for their invaluable feedback, Frode Kileng for a comment
that partly inspired this work~\cite{frode}, and members of IRTF PANRG and
Stanford SNR for other helpful discussions. This work was supported in part by
NSF grants 2045714 and 2039070, DARPA contract HR001120C0107, a Stanford School
of Engineering Fellowship, a Sloan Research Fellowship, affiliate members and
other supporters of Stanford DAWN, including Meta, Google, and VMware, as well
as Cisco and SAP, and Huawei, Dropbox, Amazon, and the Mozilla Foundation. Any
opinions, findings, and conclusions or recommendations expressed in this
material are those of the authors and do not necessarily reflect the views of
the sponsors.


%-------------------------------------------------------------------------------
\bibliographystyle{plain}
\bibliography{atc25}

\newpage
\appendix
\section{Appendix: Intuitive analysis of path-aware CUBIC}
\label{sec:sidekick:appendix}

\begin{figure*}[ht]
\centering
\includegraphics[width=0.8\linewidth]{sidekick/figures/cwnd_legend.pdf}\\
\begin{subfigure}{0.32\linewidth}
  \includegraphics[width=\linewidth]{sidekick/figures/cwnd_split_loss0p.pdf}
  \caption{Split CUBIC, 0\% loss.}
  \label{fig:time-cwnd:split-loss0p}
\end{subfigure}
\begin{subfigure}{0.32\linewidth}
  \includegraphics[width=\linewidth]{sidekick/figures/cwnd_pacubic_loss0p.pdf}
  \caption{PACUBIC, 0\% loss.}
  \label{fig:time-cwnd:pacubic-loss0p}
\end{subfigure}
\begin{subfigure}{0.32\linewidth}
  \includegraphics[width=\linewidth]{sidekick/figures/cwnd_cubic_loss0p.pdf}
  \caption{CUBIC, 0\% loss.}
  \label{fig:time-cwnd:cubic-loss0p}
\end{subfigure}
\begin{subfigure}{0.32\linewidth}
  \includegraphics[width=\linewidth]{sidekick/figures/cwnd_split_loss1p.pdf}
  \caption{Split CUBIC, 1\% loss.}
  \label{fig:time-cwnd:split-loss1p}
\end{subfigure}
\begin{subfigure}{0.32\linewidth}
  \includegraphics[width=\linewidth]{sidekick/figures/cwnd_pacubic_loss1p.pdf}
  \caption{PACUBIC, 1\% loss.}
  \label{fig:time-cwnd:pacubic-loss1p}
\end{subfigure}
\begin{subfigure}{0.32\linewidth}
  \includegraphics[width=\linewidth]{sidekick/figures/cwnd_cubic_loss1p.pdf}
  \caption{CUBIC, 1\% loss.}
  \label{fig:time-cwnd:cubic-loss1p}
\end{subfigure}
\caption{Congestion window of a long-running upload in Scenario \#2
(\Cref{tab:sidekick:experimental-scenarios}) with $0\%$ and $1\%$ loss on the
near path segment. The cwnd is measured at the data sender,
except for split CUBIC whose split connection also has a cwnd at the proxy.
PACUBIC reacts to every congestion event while keeping the cwnd high.
CUBIC performs poorly when there is loss on the near path segment.
CUBIC and PACUBIC are implemented in QUIC, while split CUBIC is implemented
in TCP using a PEP.
}
\label{fig:time-cwnd}
\end{figure*}

Here, we dive deeper into the intuition behind the PACUBIC constants
(\Cref{sec:sidekick:design:sender}), including how they were derived and why
the PACUBIC algorithm achieves similar congestion behavior to the CUBIC
algorithm in a split connection---we call this behavior ``split CUBIC''.

Consider the same network topology as \Cref{fig:sidekick:overview} in which a
data sender uploads a large file to a data receiver, with help from a Sidekick
proxy in the middle of the connection. The near path segment connects the sender
to the proxy, and the far path segment connects the proxy to the receiver.
The near segment is low-delay with varying random loss, and the far segment is
high-delay with no random loss. The far segment is the bottleneck link in terms
of bandwidth.
The actual link parameters are the same as in Scenario \#2 of
\Cref{tab:sidekick:experimental-scenarios}.

We first discuss how split CUBIC would behave in this setting to conceptually
motivate PACUBIC. Consider the congestion windows of each half of the split
connection, one taken at the data sender and one at the proxy
(\Cref{fig:time-cwnd:split-loss0p,fig:time-cwnd:split-loss1p}). The far path
segment experiences only congestive loss, leading the window at the proxy to
fluctuate around the segment's BDP regardless of the loss on the near path
segment. The window at the data sender independently determines whether the
packets that reach the proxy will be able to fully utilize the window set at
the far path segment. The data sender is able to achieve this at low random
loss rates, but becomes the bottleneck as loss rates increase
(\Cref{fig:sidekick:fairness-line}).

While split CUBIC has two windows, PACUBIC only has one
window representing the in-flight bytes of the end-to-end connection.
PACUBIC considers loss detected from both quACKs and end-to-end ACKs.
Conceptually, we want an algorithm that would enable PACUBIC's single
congestion window to match the sum of CUBIC's two congestion windows, or
the total number of in-flight bytes.
% That is the motivation behind adjusting the window proportionally to the
% number of in-flight bytes on each path segment, depending on where the loss
% occurred.

With no random loss on the near path segment, PACUBIC
(\Cref{fig:time-cwnd:pacubic-loss0p}) behaves the same as normal CUBIC
(\Cref{fig:time-cwnd:cubic-loss0p}). The congestion window is entirely governed
by end-to-end ACKs since the far path segment is the bottleneck link. Note that
while the sender may be able to deduce that a loss occurred on the far path
segment by combining info from the quACK with the end-to-end ACK, PACUBIC
conservatively treats the loss as occurring anywhere on the path.

With some random loss on the near path segment, PACUBIC grows and reduces cwnd
based on where the last congestion event occurs
(\Cref{fig:time-cwnd:pacubic-loss1p}). Note that if the congestion window $cwnd$
represents the bytes in-flight in the end-to-end connection, then $r \cdot cwnd$
represents the proportion of bytes in-flight on the near path segment. At a
high level, if the data sender discovers loss on the near path segment via the
quACK, it holds the $(1-r)\cdot cwnd$ portion of the ``far window'' constant
while applying the CUBIC algorithm to the remaining $r \cdot w_{max}$ of the
``near window,'' representing the bottleneck link.

Mathematically, instead of reducing $w_{max}$, the window size just before the
last reduction, by $(1-\beta^*) \cdot w_{max}$, PACUBIC reduces it by only
$[1 - (1-r(1-\beta^*))] \cdot w_{max} = r(1-\beta^*) \cdot w_{max}$.
That is $r$ times the original reduction, a \emph{smaller} amount.
We use the RTT ratio $r$ (near path segment to end-to-end)
% indicating the RTT ratio of near path segment vs.~far path segment)
as a proxy for the ratio of the number of in-flight bytes.

Similarly, instead of using a cubic growth function with scaling factor $C^*$
and inflection point $K = K^* = \sqrt[3]{w_{max}(1-\beta^*)/C^*}$,
we use a larger scaling factor $C = C^*/r^3$
and thus a shorter inflection point
\[
K = \sqrt[3]{\frac{w_{max}(1-\beta)}{C}}
= \sqrt[3]{\frac{r\cdot w_{max}(1-\beta^*)}{C^* / r^3}}
= r^{4/3} \cdot K^*.
\]
The shorter inflection point leads the congestion window to \emph{grow more
quickly} since the sender also reacts to feedback about loss more quickly over
the low-delay link.
% Since similarly proportional to $w_{max}$, so with $C'$, we both reduce the
% inflection point $K$ and the growth rate $C$ of the function proportionally to
% the number of bytes in-flight on the near path segment.

At times, there can be loss detected both in quACKs and in end-to-end ACKs.
The end-to-end ACKs have a greater effect since they reduce the congestion
window by a larger proportion, until the remaining path segment with loss is the
bottleneck link. In this scenario with loss, the bottleneck link at equilibrium
is the near path segment.
At this point, the quACK primarily determines the congestion window updates. If
the far path segment were to become the bottleneck again, the data sender would
detect a congestion event via the end-to-end ACK.

PACUBIC has several limitations. Although it beats end-to-end CUBIC, it still
performs worse than split CUBIC, especially at high loss rates
(\Cref{fig:sidekick:fairness-line}). Also, it doesn't consider loss on the far
path segment any differently than original CUBIC, unlike split CUBIC which
treats the two split connections independently. PACUBIC emulates the congestion
control behavior and fairness of split CUBIC fairly well as a heuristic, but
would benefit from an analysis in a wider variety of network scenarios. It
would also benefit from a side-by-side fairness comparison against other
congestion control algorithms that perform well in the same scenarios. We'd
like the primary takeaway of PACUBIC to be that knowing where loss occurs can
cleverly inform congestion control.


%%%%%%%%%%%%%%%%%%%%%%%%%%%%%%%%%%%%%%%%%%%%%%%%%%%%%%%%%%%%%%%%%%%%%%%%%%%%%%%%
\end{document}
%%%%%%%%%%%%%%%%%%%%%%%%%%%%%%%%%%%%%%%%%%%%%%%%%%%%%%%%%%%%%%%%%%%%%%%%%%%%%%%%

%%  LocalWords:  endnotes includegraphics fread ptr nobj noindent
%%  LocalWords:  pdflatex acks
