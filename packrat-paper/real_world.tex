\subsection{Raspberry Pi Experiments}

\subsection{Raspberry Pi Experiments}

\subsection{Raspberry Pi Experiments}

\subsection{Raspberry Pi Experiments}

\input{figures_tex/real_world}

\Cref{fig:real-world} shows that \Sys protocols are robust in a real-world
environment even in the direction when the \Sys connection is near the
data \textit{receiver}. In the HTTP download application, \Sys increases the
long-lived throughput of the file download from XX MBit/s to XX Mbit/s---a
XX\% improvement. In the media application, \Sys reduces the 99th percentile
de-jitter latency XX\%, from XX to XX ms.

When thinking about real world scenarios that might benefit from \Sys, in this
case those with lossy path segments, one might reason that loss has all but
been eliminated in cellular and Wi-Fi networks by link-layer retransmissions.
However, \Cref{fig:real-world} demonstrates that there are tradeoffs with
configuring hardware retransmits, particularly in the media application.

In low-latency transport protocols such as media transfers, it can often be
desirable to drop frames instead of waiting for their retransmission as a
tradeoff between playback delay and smoothness. As we increase the number of
Wi-Fi retransmissions in the access point, the media application
actually...does something undesirable. In general, since link-layer
retransmissions are unilaterally applied to all connections that cross that
network path segment, it takes control out of co-existing applications to
determine the reliability guarantees that apply to them most.


\Cref{fig:real-world} shows that \Sys protocols are robust in a real-world
environment even in the direction when the \Sys connection is near the
data \textit{receiver}. In the HTTP download application, \Sys increases the
long-lived throughput of the file download from XX MBit/s to XX Mbit/s---a
XX\% improvement. In the media application, \Sys reduces the 99th percentile
de-jitter latency XX\%, from XX to XX ms.

When thinking about real world scenarios that might benefit from \Sys, in this
case those with lossy path segments, one might reason that loss has all but
been eliminated in cellular and Wi-Fi networks by link-layer retransmissions.
However, \Cref{fig:real-world} demonstrates that there are tradeoffs with
configuring hardware retransmits, particularly in the media application.

In low-latency transport protocols such as media transfers, it can often be
desirable to drop frames instead of waiting for their retransmission as a
tradeoff between playback delay and smoothness. As we increase the number of
Wi-Fi retransmissions in the access point, the media application
actually...does something undesirable. In general, since link-layer
retransmissions are unilaterally applied to all connections that cross that
network path segment, it takes control out of co-existing applications to
determine the reliability guarantees that apply to them most.


\Cref{fig:real-world} shows that \Sys protocols are robust in a real-world
environment even in the direction when the \Sys connection is near the
data \textit{receiver}. In the HTTP download application, \Sys increases the
long-lived throughput of the file download from XX MBit/s to XX Mbit/s---a
XX\% improvement. In the media application, \Sys reduces the 99th percentile
de-jitter latency XX\%, from XX to XX ms.

When thinking about real world scenarios that might benefit from \Sys, in this
case those with lossy path segments, one might reason that loss has all but
been eliminated in cellular and Wi-Fi networks by link-layer retransmissions.
However, \Cref{fig:real-world} demonstrates that there are tradeoffs with
configuring hardware retransmits, particularly in the media application.

In low-latency transport protocols such as media transfers, it can often be
desirable to drop frames instead of waiting for their retransmission as a
tradeoff between playback delay and smoothness. As we increase the number of
Wi-Fi retransmissions in the access point, the media application
actually...does something undesirable. In general, since link-layer
retransmissions are unilaterally applied to all connections that cross that
network path segment, it takes control out of co-existing applications to
determine the reliability guarantees that apply to them most.


\Cref{fig:real-world} shows that \Sys protocols are robust in a real-world
environment even in the direction when the \Sys connection is near the
data \textit{receiver}. In the HTTP download application, \Sys increases the
long-lived throughput of the file download from XX MBit/s to XX Mbit/s---a
XX\% improvement. In the media application, \Sys reduces the 99th percentile
de-jitter latency XX\%, from XX to XX ms.

When thinking about real world scenarios that might benefit from \Sys, in this
case those with lossy path segments, one might reason that loss has all but
been eliminated in cellular and Wi-Fi networks by link-layer retransmissions.
However, \Cref{fig:real-world} demonstrates that there are tradeoffs with
configuring hardware retransmits, particularly in the media application.

In low-latency transport protocols such as media transfers, it can often be
desirable to drop frames instead of waiting for their retransmission as a
tradeoff between playback delay and smoothness. As we increase the number of
Wi-Fi retransmissions in the access point, the media application
actually...does something undesirable. In general, since link-layer
retransmissions are unilaterally applied to all connections that cross that
network path segment, it takes control out of co-existing applications to
determine the reliability guarantees that apply to them most.
