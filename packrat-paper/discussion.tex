\section{Discussion}

Our emulations model non-congestive loss as random, independent loss,
which may not accurately reflect the real world. While non-congestive loss
does exists, its various behaviors are not well-characterized. This work
would benefit from a measurement study on the patterns of non-congestive loss
to better understand the scope of the problem and its impact.

We base this work on the idea that proxies and link-layer approaches are
deployed for a reason, and understanding the impact that encrypted transport
protocols have in these real-world network paths would better motivate the need
for protocol-agnostic proxies at all. This is also true on paths where
reliable connectivity is not a solved problem. Also,
in-network retransmissions intuitively reduce load in the network and it may be
interesting to explore how these retransmissions benefit not just
applications but the network itself.

The question for any research system is how can it be deployed and useful in
the real world. The simplest deployment today could look like a simple in-line
\Sys proxy between a Wi-Fi router and Internet modem. Public hotspots and
dense urban apartments make good candidates for smaller-scale lossy path
segments.
% Another deployment is in satellite networks where TCP connection-splitting
% proxies are typically deployed. Here, the performance of QUIC is a documented,
% unsolved problem~\cite{kosek2022quicpep,kuhn2021quic-over-sat}.
The client integration could be a browser extension for QUIC or a simple
modified media client.
The \Sys protocol has the advantage that only one endpoint in addition to the
proxy needs to speak the protocol.

In this paper, we explored an alternative construction of the quACK that is
optimized for computational efficiency at a small granularity. An extension to
this is finding other applications of the quACK in the setting of network
packet analysis. We may also be able to further improve the link overheads of
the IBLT quACK at millisecond timescales using, e.g., interactive quACKing
between the proxy and endpoint.

% We have mainly worked around the non-determinism of the IBLT quACK by
% sending $4\!\times$ the number of symbols as the packets the client thinks it
% is missing. The theoretical constant overhead of IBLTs is only $1.35\!\times$
% and perhaps, e.g., interactive quACKing could improve the signaling in the \Sys
% protocol. Another extension is quACKing from the proxy to the client to
% indicate which packets are possibly available for retransmission, or applying
% the computationally efficient quACK to other practical packet-scale problems.

% In the context of QUIC and connection-splitting specifically, can this
% combine with the Sidekick protocol to match the performance
% of split TCP connection settings bi-directionally?

% scalability

% Related: real-world evaluations in service provider networks. what does real traffic
% look like and how does it impact the proxy? (e.g. lots of short connections,
% connections per page load, connections per client, "elephant" flows, etc.)

% what should congestion control look like for connection splitters?
% what is fair for in-network retransmissions
% same as picoquic connection splitter as a baseline
