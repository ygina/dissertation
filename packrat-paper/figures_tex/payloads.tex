\newtcolorbox{protopayload}[2][]{
  colback=yellow!20,
  colframe=black,
  boxrule=0.5pt,
  arc=1pt,
  fontupper=\ttfamily,
  width=\linewidth,
  fonttitle=\small\bfseries,
  top=-1.5mm,
  bottom=-1.5mm,
  left=2mm,
  right=2mm,
  title={#2},
  #1
}

\begin{figure}[t]
    % \centering
    % Client payloads
    \begin{subfigure}[b]{0.48\linewidth}
        \begin{protopayload}{\texttt{Init}}
            \begin{lstlisting}[language=Rust,basicstyle=\footnotesize]
epoch: u32;
base_conn: [u8; 12];
eack_ty: u8;
num_symbols: u8;
id_offset: u16;
mem_bytes: u32;
            \end{lstlisting}
        \end{protopayload}
        \begin{protopayload}{\texttt{quACK}}
            \begin{lstlisting}[language=Rust,basicstyle=\footnotesize]
count: u32;
last_identifier: u32;
code: Vec<Symbol>;
            \end{lstlisting}
        \end{protopayload}
        \caption{Client payloads.}
        \label{fig:payloads:client}
    \end{subfigure}
    \hfill
    % Proxy payloads
    \begin{subfigure}[b]{0.48\linewidth}
        \begin{protopayload}{\texttt{InitACK}}
            \begin{lstlisting}[language=Rust,basicstyle=\footnotesize]
epoch: u32;
udp_port: u16;
errno: u32;
            \end{lstlisting}
        \end{protopayload}
        \begin{protopayload}{\texttt{Reset}}
            \begin{lstlisting}[language=Rust,basicstyle=\footnotesize]
epoch: u32;
errno: u32;
            \end{lstlisting}
        \end{protopayload}
        \begin{protopayload}{\texttt{Retransmit}}
            \begin{lstlisting}[language=Rust,basicstyle=\footnotesize]
udp_payload: Vec<u8>;
            \end{lstlisting}
        \end{protopayload}
        \caption{Proxy payloads.}
        \label{fig:payloads:proxy}
    \end{subfigure}
  % Caption
  \caption{Packrat protocol messages to initialize the connection and generate
   retransmissions. There are two types of quACKS: the IBLT quACK
   we introduce in \Cref{sec:eack:iblt} and an equivalent implementation of the
   power sum quACK from \cite{yuan2024sidekick}. The \texttt{Symbol}
   uses 5 and 4 bytes in each type, respectively.}
  \label{fig:payloads}
\end{figure}
