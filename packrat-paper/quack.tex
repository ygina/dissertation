%-------------------------------------------------------------------------------
\section{Scalability of the Packrat Proxy}
\label{sec:eack}
%-------------------------------------------------------------------------------
An on-path proxy handles many tens to hundreds of thousands of concurrent connections at once.
Any per-packet overhead incurred by the \Sys protocol at the proxy must be
extremely small. The \Sys proxy encodes every packet in the base connection and, unlike
the Sidekick proxy, also decodes every acknowledgment
($91\!\times$ more expensive than encoding in \cite{yuan2024sidekick}).
We want to explore whether a different type of acknowledgment of encrypted packets
can be more suitable in the setting of in-network
retransmissions for its computational efficiency.

\subsection{eACKing with Client Hints}
\label{sec:eack:hints}

We describe two optimizations at the client for dynamically adjusting the eACK
based on knowledge of the base connection to send smaller and fewer eACKs
overall.
We evaluate the impact of these optimizations in \Cref{sec:evaluation:link-overheads}.

\paragraph{Selective eACKing.} If the client does not expect it needs a
 retransmission, there is not much point in sending an eACK. In NACK schemes,
 where the client detects loss and asks for a retransmission, it can choose to
 selectively eACK only when it would otherwise send a NACK. Note that the client can omit
 regular eACKs without the cache exploding in size because the proxy
 optimistically evicts packets, and these are likely to be received or the
 client would have NACKed.\\
