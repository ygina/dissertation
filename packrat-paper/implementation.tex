%-------------------------------------------------------------------------------
\section{Implementation}
\label{sec:implementation}
%-------------------------------------------------------------------------------

We now describe our implementation of the \Sys protocol, which includes
the \texttt{eack} library (\Cref{sec:eack:microbenchmarks}),
a \texttt{packrat} library used for three client integrations,
and a \Sys proxy. We also describe the implementations of the
applications and baselines we use to evaluate \Sys.

\subsection{Applications}

We evaluate the \Sys protocol in three applications with different performance
metrics to explore the versatility of in-network retransmissions:
a high-throughput HTTP/3 file download, a low-latency media stream with forward
error correction, and a reliable multicast stream whose server has limited
capacity to handle end-to-end unicast retransmissions.

\paragraph{HTTP/3 file download.}

The HTTP/3 benchmark uses the default client and server in the Picoquic QUIC
implementation~\cite{picoquic}. Picoquic is an implementation of QUIC in C
based on the IETF standard used for experimentation on extensions to QUIC in
the IETF, with simplicity and correctness as its goals. Both endpoints use
CUBIC for congestion control.

% We modified the server to pre-cache data in memory and for the client to be able
% to query variable data sizes via an HTTP/GET request.

The goodput is measured by dividing the connection time by the requested data
size, excluding headers.
The connection time is measured as when the client sends the first
byte of its request to when it receives all requested bytes.
We use 25 MB which is large enough such that the goodput is stable.

\paragraph{Low-latency media with FEC.}

We implement an endpoint for streaming low-latency audio data from the server
to the client, using a simple repetition code for forward error correction.
The server (a) sends a packet every 20 ms
(b) containing audio from the last 40 ms (so there's an overlap in each
packet). Each audio packet contains 480 bytes of data, representing an audio
stream at 96 kbit/s.

The client begins playback with 40 ms in its buffer, stalling if frames arrive
too late and fast-forwarding if behind the target buffer size.
If there is a gap in the
playback buffer, the client sends a NACK with the sequence number of the
missing frame. The client retransmits NACKs, up to one per RTT, until it has
received the missing frame. On receiving a NACK, the server immediately
retransmits.

The one-way latency is measured from the time the data is produced in the real
world to when it is encoded, transmitted, and available in the client's
buffer. For example, a frame containing 40 ms of data sent over a network path
with 150 ms delay has a minimum one-way latency of 190 ms.

\paragraph{Reliable IP multicast stream.}

The IP multicast application is similar to the low-latency media application
except the server streams data to a multicast IP address. The server does not
use forward error correction, and each packet contains 240 bytes of data. Multiple clients
subscribe to the multicast IP address, and if a client is missing data,
it sends an end-to-end NACK to the
multicast server to receive a unicast retransmission.
The primary metric is the number of end-to-end retransmissions requested by
the clients. That is, we measure the number of packets that the server
must (unicast) retransmit, which captures both server and network load.

\subsection{Baselines}

The primary baseline is just the end-to-end protocol, which is the default for
encrypted transport protocols that cannot receive transport-layer assistance
from proxies. 

We additionally aim to compare \Sys to link-layer loss recovery approaches,
as discussed in \Cref{sec:background}.
To mimic these approaches,
we implement a reliable tunnel that
encapsulates and retransmits packets
similar to Wi-Fi. The tunnel sender encapsulates IP datagrams in a header with
a plaintext sequence number, and the tunnel receiver replies with a 64-bit
Block ACK~\cite{ieee80211e}. The sender retransmits gaps in the acknowledgment up
to $7$ times, the default limit in Wi-Fi, and takes care to avoid duplicates. We
also implement an option for the receiver to order packets before releasing
them to the application, behavior that we believe is unspecified in the Wi-Fi
standard but can dramatically impact application performance.
We refer to these variants as ordered and unordered tunnels.

Finally, we benchmark our HTTP/3 application against a ``true'' split
connection. We implement a \texttt{picoquic} connection splitter that decrypts
and re-encrypts packets in two separate QUIC connections from one endpoint to
another. We emphasize that these connection-splitters do not actually exist. To
be deployed, the proxy would need to be credentialed with access to underlying
sequence numbers, the antithesis of the transport purists who encrypted these
headers in the first place~\cite{duke2023rfc}. Since TCP splitters \textit
{are} commonly deployed~\cite{honda2011still,rfc3135}, our goal is to
understand how much QUIC is ``losing out'' by not splitting its connection, and
how closely \Sys can help QUIC achieve the same performance benefits without
ossification.

\subsection{Client Integrations}

\begin{lstfloat}[t]
\begin{lstlisting}[language=Rust]
struct Packrat {
    /// Serialize a Packrat message to send
    fn send_init(params: ...) -> SendBuf;
    fn send_eack(); -> SendBuf;
    fn send_eack_rateless(m: usize); -> SendBuf;
    /// Handle the incoming message and return the
    /// payload if it is a `Retransmit` message.
    fn recv_payload(buf: RecvBuf) -> &[u8]
    /// Manage eACK state
    fn is_ready() -> bool;
    fn insert(id: u32) -> bool;
    fn reset();
}
\end{lstlisting}
\captionof{lstlisting}{Pseudocode interface for the \texttt{packrat} library.
 The library consolidates shared functionality at the client for sending
 eACKs. The client is responsible for reading and writing to the actual Packrat
 socket.}
\label{lst:quacker-interface}
\end{lstfloat}

Clients participate in the \Sys protocol by knowingly opting in to proxy
assistance and sending eACKs. Clients share much of the same functionality,
such as initializing the \Sys connection, maintaining a cumulative eACK of
received packets, and determining when to eACK based on a set frequency. We
implement a library for this shared functionality (\Cref{lst:quacker-interface})
and integrate it into our three clients.
The library uses $\approx\!700$ lines of Rust and includes C bindings.
% quacker$ cloc .

The library can serialize and deserialize messages in the \Sys connection, but
the client is responsible for reading and writing to the actual socket. Each
application uses a packet loop to interact with a socket for the base
connection. We incorporate the \Sys socket into the same loop, which allows us to
consider hints from the application for rateless and selective eACKing
(\Cref{sec:eack:hints}). We also incorporate a reorder delay in the base
connection (\Cref{sec:packet-protocol:problem}).
% Each integration uses a few hundred lines of code.


% Separately, we also implemented a ``sniffing quacker'' that sniffs the interface
% on the client associated with this base connection as opposed to being on-path.
% We find that this mostly works although we can't apply the optimizations of
% selective eACKing. Also, although we did not observe significant issues, there
% aren't guarantees about the synchronization of sniffing and base packets which
% makes correctness more difficult to reason about. However, it is a quick way to
% evaluate the potential benefits of a eACKer for a generic protocol. We do not
% evaluate the sniffing eACKer in this section.

\subsection{Proxy Implementation}
\label{sec:implementation:proxy}

% \thea{If time - I know this is just implementation, but wondering if there is
% anything we could/should do to make this section more interesting?
% Reference any design challenges? Remind people of cool design things from S3-4?}

We implement the \Sys proxy as a network bridge that uses raw sockets to read
and write packets between two interfaces. The proxy needs to
be on-path and intercept the packets, as opposed to sniffing them, so
its \texttt{InitACK} and \texttt{Reset} messages can be ordered along with the
data packets.

The proxy inspects each packet for magic discovery packets, those that match the
4-tuples of the base connections it is helping, and also eACKs.
It handles each packet as described in \Cref
{sec:packrat-protocol:proxy-behavior}. We use our \texttt{eack} library to
encode and decode eACKs. The proxy use $\approx\!1000$ lines of Rust.

% bin/sidekick.rs
% cache/base.rs
% cache/mod.rs
% sidekick/base.rs
% sidekick/mod.rs
% stream.rs

\paragraph{Multicast proxy.}

We also implement an extension to the \Sys proxy for IP multicast, or more
generally, multiple clients that share a base data stream. The proxy maintains
a fixed-size cache for the multicast 4-tuple with optimistic eviction only. For
each client, the proxy maintains a eACK and a \textit{virtual cache}. The
virtual cache contains (1) a global index in the base cache of the client's
first unacknowledged packet and (2) pairs of global indexes that indicate which
packet to insert and where to insert it because it was retransmitted to the
client. This allows the proxy to maintain state proportional only to the number
of outstanding retransmissions per client.

% \paragraph{Packet ordering.}

% The proxy needs to be on-path and order all incoming and outgoing packets in
% something like a packet loop, at least for a single connection. This is unlike
% when the sidekick connection is near the data \textit{sender}, where the proxy
% can simply sniff packets off the path.

% In our implementation, we configure the network interfaces to not forward
% packets, and then we write a program that intercepts packets from raw sockets
% for two interfaces, respectively, and forwards incoming packets from one
% interface to the other. This involves a data copy which inherently adds
% overhead, but the overhead is not fundamental if we were to use DPDK or some
% other kernel pointers.

% Serialization is important, for example, because say we process a eACK out of
% sync of the data path, we may process a eACK from the future if the proxy
% hasn't encoded packets it has already sent because it hasn't had a chance to
% sniff them. The DiscoverAck and Reset packets are both serialization points in
% a connection. This allows sidekick connections to be established or reset at
% any point in the base connection, and it is important that there is not too
% much reordering.

% \paragraph{Sidekick functionality.}

% The sidekick functionality in the proxy leverages the eACK library from \cite
% {yuan2024sidekick}, along with an extension for the IBLT eACK. It encodes each
% packet in the base connection, and then decodes eACKs, retransmitting any
% packets determined to be missing. If it hits a memory limit when encoding
% packets and adding them to the cache, it will reset the connection and
% communicate this limit to the client to configure its eACK frequency. If it
% hits a limit due to the congestion window, it will drop the packet from the
% data sender and not forward it to the data receiver.

% In addition, for shared streams such as in the IP multicast application, the
% proxy maintains a fixed-size cache for the base connection with optimistic
% eviction. It also maintains a virtual buffer of packets for each client as
% opposed to a single buffer that duplicates all the packet data. The virtual
% buffer consists of a 32-bit index for the next index in the base buffer to
% receive, as well as a ring buffer of base index and virtual index pairs at
% which to insert retransmitted packets.

% \paragraph{Cache drop policy.}

% The proxy has similar memory constraints to a connection-splitting TCP PEP per
% base connection, and we won't try to make it better than that. Since our caches
% are re-implemented, we have a similar policy that newly added packets exceeding
% the capacity are simply not added until the packets already in the cache are
% evicted.

% Similar to the TCP PEP, in addition to OS memory limits we also need some sense
% of a congestion control window. This allows us to achieve the same fairness as
% a split connection of the same congestion control algorithm. However, since the
% behavior differs based on CCA and even implementation of the CCA, we simply
% select one and call it a day.
