%-------------------------------------------------------------------------------
\begin{abstract}
%-------------------------------------------------------------------------------

This paper describes and evaluates new techniques for
network-originated retransmissions for end-to-end transport
connections, yielding performance benefits for encrypted transport
protocols in lossy settings.  We use set-reconciliation techniques
based on the Rateless IBLT to let the receiver efficiently acknowledge
encrypted packets to a network function, without modifying the sender
or the underlying wire format. With these tools, endpoints can receive
network-originated retransmissions appropriate for them, without an
additional layer of encapsulation, tunnel, or link-layer
retransmissions.

% Lossy network paths can significantly harm application performance. Existing
% approaches such as link-layer retransmissions and connection-splitting either
% cannot be tuned to the connections that share the link, or are otherwise
% infeasible without ossification or credentialing the proxy. In this paper, we
% show how in-network retransmissions via a sidekick connection can enable a
% variety of performance enhancements---higher throughput, lower latency, and
% lower link overheads---for a variety of \textit{encrypted} transport protocols
% in these settings. We utilize set reconciliation techniques based on the quACK
% to refer to encrypted packets, without involving the data sender, and leave the
% underlying protocol unchanged on the wire and opaque to the
% performance-enhancing proxy.

% These in-network retransmissions require an on-path proxy located near the
% data \textit{receiver}, and scalability becomes challenging when the more
% complex decoding and retransmission logic is shifted from the endpoint to the
% proxy. To address this issue, we present an alternate construction of the quACK
% using the IBLT data structure, improving the efficiency of encoding and
% decoding operations. Combined with the insight that the quACK sender is
% co-located at the endpoint with the application ACK sender, we can send
% significantly fewer and smaller quACKs overall, minimizing the extra load
% introduced by the sidekick connection to the network.

\end{abstract}
