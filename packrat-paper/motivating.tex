%-------------------------------------------------------------------------------
\section{Motivating Scenario}
\label{sec:motivating}
%-------------------------------------------------------------------------------

Say you're on a moving train, again. You are audio calling a friend but this
time \textit{you} are the one experiencing poor connection quality. Your friend
keeps freezing and you don't know what they are saying as many packets are
getting dropped.

Fortunately, the train access point keeps a small buffer of encrypted packets
for your connection. Your application recognizing it didn't receive a packet
from your friend, and quACKnowledges the packets that were received to the
access point. Note that you can't directly send a negative acknowledgment since
the AP does not have access to packet sequence numbers. The AP retransmits most
of the packets that were lost, significantly reducing your dejitter buffer
latency.

Not soon after, you receive the same packet again directly from your friend.
When you sent that quACK, your application still relied on the end-to-end
mechanism to send a NACK. This is a spurious retransmission.

In the case when you are downloading a large file, these spurious
retransmissions clog up the network, making congestion even worse. In addition,
since the same end-to-end signals about loss make it back to the sender, you
don't get the split throughput advantages of a split connection. Or worse, the
data sender detects all these spurious retransmissions and you end up in a land
of poorly defined behavior because most transport protocol developers don't
heavily test this codepath.

It is super important to receive sidekick assistance when downloading data near
a lossy last-mile path segment, but it is not that trivial to just add a packet
cache and immediately retransmit missing packets.