\section{Background}
\label{sec:background}

We provide a brief history of loss recovery mechanisms in the network to
motivate the need for lightweight, non-ossifying, in-network retransmissions
for encrypted transport protocols.

\paragraph{End-to-end loss recovery.}

% End-to-end loss recovery has traditionally served two purposes: facilitating
% retransmissions in reliable connections and adapting congestion control.
In end-to-end loss recovery,
congestion control algorithms were traditionally designed to treat all loss as an indicator of
congestion~\cite{rfc5681tcp,rfc2001tcp}.
% congestion~\cite{rfc8985,rfc9000,rfc5681tcp,rfc2001tcp,ha2008cubic,cardwell2024bbrv3-ietf120}.
% only react upon congestive loss~\cite{2019ccacensus}.

With modern wireless networks and higher bit-error rates, recent developments
such as the BBR congestion control
% (Bottleneck Bandwidth and Round-trip propagation time)
algorithm have de-prioritized packet loss as a congestion
signal~\cite{cardwell2017bbr}. This change enables higher throughput but has
raised concerns regarding TCP fairness~\cite
{ware2019modeling,philip2024prudentia}. In response, the performance of newer
versions of BBR has regressed in lossy environments, suggesting a lasting trend
in the treatment of loss as a congestion signal regardless of its source~\cite
{atc-submission}.

% regarding fairness among competing flows, suggesting that packet loss---whether
% due to congestion or non-congestive factors---remains closely associated with
% congestion dynamics and needs to be considered by
% CCAs~\cite{ware2019modeling,philip2024prudentia}.

\paragraph{Link-layer loss recovery.}

Loss recovery at the link layer~\cite{3gpp5gstandard,le2022link,ieee80211e} can
mitigate some of the dramatic reactions of CCAs by hiding random loss from
endpoints.
% The local RTT is much shorter than the end-to-end RTT such that
% sporadic losses can be repaired much faster (milliseconds or less).
% Local
% retransmissions also minimize overhead as they are confined to the lossy path
% segment itself.
% Note link-layer retransmissions are not applicable to higher
% latency settings such as satellite networks.
Applications and transport protocols are oblivious to this
functionality, and there is a tradeoff to this reliability~\cite{klingler2018impact,kliazovich2012arqproxy}.
Retransmitting more often increases the chance
of success but increases the time needed, and may also introduce
jitter and reordering~\cite{leung2007overview}.
Low-latency applications may prefer to transmit new data instead, but
retransmitting less often means more loss.
There is no ideal configuration for all connections that share the link.

% Detecting loss and retransmitting packets at the link layer has great benefits.
% The local RTT is much shorter than the end-to-end RTT, such that sporadic
% losses can typically be repaired much faster. Such local retransmissions also
% minimize overhead, as they are confined to the lossy path segment itself. Link
% layer loss recovery, e.g. at Wi-Fi or cellular links, can hide random loss from
% the endpoints such that the dramatic reaction of CCAs is avoided; applications
% and transport protocols are oblivious to this
% functionality~\cite{ratnamwtcp1998,ieee80211e,kliazovich2012arqproxy,wong1997linklayerrtx}.
% Retransmissions by link layers are a common source of jitter, and,
% depending on the behavior of the link's receiver, they may contribute to
% reordering in the network~\cite{perf2006linklayerrtx,leung2007overview}.

% There is a trade-off between reliability and the delay that local
% retransmissions produce. Trying longer (i.e., more often) increases the time
% needed and the chance of success. Rather than trying to retransmit (and delay)
% old data, some applications may prefer transmitting newer data instead; thus,
% ideally, applications should be in control of this trade-off, but they are not.
% Various ideas to improve the communication between applications and wireless
% link layers have been put forward~\cite{kwbdgrr-tsvwg-net-collab-rqmts-04}, but
% with little success~\cite{rfc9049}. The new Standard Communication with Network
% Elements (SCONE) Working Group in the IETF now defines such signaling for QUIC,
% and Explicit Congestion Notification (ECN) is such a signal---but these are in
% the direction of the network offering information (``throughput advice'') to
% endpoints, not vice versa~\cite{rfc3168,brw-scone-analysis-00}.
% This is not completely true: the DSCP offers a signal. E.g., RFC 8325 describes
% how the DSCP should be mapped onto 802.11 networks. but this is a coarse-grain,
% unreliable signal (the field may be zeroed on the path), and I don't know if  a
% certain DSCP choice translates into limiting the number of retransmits. All in
% all, I think it's best to avoid this complication and ignore this.

\paragraph{Transport-specific mechanisms.}

Some transport-layer proxies split a connection into two or more segments
and optimize the connection for each segment. In addition to
TCP connection splitters~\cite{rfc3135,honda2011still,hayes2019mmwave},
selective forwarding units (SFUs) are
used for media streams where low-latency is critical, so retransmissions occur
closer to the data receiver~\cite{rfc7667,andre2018comparative}.

% the "proxyimportance" ref may sound like it's just about mmWave but it
%  contains 4G measurements, finding proxy deployment

With Snoop~\cite{balakrishnan1995snoop}, it is clear
that complexities emerge in how the proxy interprets
and modifies plaintext sequence numbers on the wire, which also ossifies TCP.
This paper aims to explore how such a proxy can send helpful in-network
retransmissions to a variety of encrypted protocols while preserving
end-to-end semantics, without access to such information.


These types of approaches make unwarranted assumptions about
transport headers ossify the protocol~\cite{papastergiou2017deossifying}.
By making reliability guarantees to the endpoint, the connection-splitting
proxies also fate share with the connection.


% Another disadvantage is that these proxies fate share with th
% Any type of proxy assistance that cannot ultimately fall back to end-to-end
% mechanisms mandates that the proxy fate share with the connection.

% However, this requires
% credentialing the proxy and would not work for encrypted protocols or
% applications that don't pay to host a CDN.

\paragraph{Protocol-agnostic transport assistance.}

Forward error correction can be used in lossy environments, but this
introduces significant overheads as entire packet contents (and not just
sequence numbers) must be encoded during transmission~\cite
{rfc9265}. FEC also requires participation from both endpoints.

Mechanisms that encapsulate packets
% I like the Kramer MASQUE paper b/c it talks about applying MASQUE as a PEP
% The RFC is the introduction of MASQUE, but seems like some people cite the 
% original draft too.
offer potential for reliability without decrypting packet contents, but their
primary focus tends to be security. VPNs introduce a costly
encryption layer.
MASQUE tunnels protocols over HTTP/3, and if used for local retransmissions
has the potential for complex nested loss recovery and congestion control
interactions~\cite{rfc9298masque,schinazi-masque-proxy-05,kramer2021masquepep}.
It is intended to function per-application, and the best known implementation
from Apple functions like a VPN~\cite{icloud-private-relay}.
These proxies may also not be on-path, which harms performance.

% The MASQUE standards allow configurations with local retransmissions, e.g. when
% implementing a MASQUE tunnel with HTTP/2 over TCP, but the specifications
% present this as a disadvantage---which it is indeed, since MASQUE tunnels are
% not defined to be chosen per application, and some applications may not desire
% the added delay from local retransmits at all. Instead, the best known
% implementation, Apple Private Relay, functions like a VPN, carrying all traffic
% of a user through MASQUE tunnels. Another disadvantage of MASQUE configurations
% with local retransmissions is that TCP, or QUIC, as a tunnel protocol, is a
% full-fledged transport implementation, including additional functions such as
% congestion control (i.e., carrying QUIC over TCP or QUIC produces nested
% congestion control loops).

% advises against using HTTP/3 in
% ``reliable'' mode due to the potential for complex nested loss recovery and
% congestion control interactions.
% \thea{Can't find citations for this recommendation in RFC.}


% While this does not exist to our knowledge, what about a tunneling approach
% To generically enhance transport layer reliability, one approach involves
% encapsulating the transport protocol within a reliable transport protocol at a
% proxy. This method allows for protocol separation without requiring
% credentialed access to packet contents, effectively integrating aspects of
% link-layer retransmissions and transport-layer connection splitting.

The Sidekick proxy provides protocol-agnostic assistance by sending
``quACKs'' of encrypted packets to an endpoint, but cannot retransmit
packets itself~\cite{yuan2024sidekick}. This limits its usefulness
to when the loss is near the data \textit{sender}.
However, client traffic (which is near the lossy last-mile) is typically heavier
in the downlink direction. In this paper, we similarly leverage set
reconciliation in eACKs for referring to encrypted packets,
but to send network-originated retransmissions.
