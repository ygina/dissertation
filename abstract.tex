\prefacesection{Abstract}

Post-TCP transport protocols such as QUIC now include end-to-end encryption at
the transport layer. This enhances security by making their packets opaque to
connection-splitting proxies and immune to ossification, but can harm
performance. In this dissertation, I will present the Sidekick approach to
in-network assistance for secure transport protocols where proxies and
endpoints send information on an adjacent connection about which encrypted
packets they have received. Sidekick protocols apply set reconciliation
techniques in a novel setting to efficiently refer to encrypted packets in
a \textbf{quACK}, without using plaintext sequence numbers. In some use cases
of the Sidekick protocol, Packrat proxies keep a small cache of packets for
possible in-network retransmissions of encrypted packets. This approach allows
secure transport protocols to achieve performance benefits similar to those of
traditional PEPs, but leaves the protocol unchanged on the wire and free to
evolve. Finally, I will present the split throughput heuristic for reasoning
about connection-splitting in the context of two recent developments:
the BBR congestion control algorithm and the QUIC transport protocol.
I use this heuristic in an emulation measurement study and discuss how
connection-splitting, despite the ossification it can induce, still offers
valuable performance benefits today.