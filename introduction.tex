\chapter{Introduction}

In the Internet's canonical model, transport is end-to-end and implemented only
in hosts. Traditionally, routers and other network components forwarded IP
datagrams without regard to their payloads or flow membership~\cite
{saltzer1984endtoend, clark1988darpa}; only hosts thought about connections,
reliable delivery, or flow-by-flow congestion control.

In practice, however, the best behavior for a transport protocol depends on the
particulars of the network path. An appropriate retransmission or
congestion-control scheme for a heavily-multiplexed wired network wouldn't be
ideal for paths that include a high-delay satellite link, Wi-Fi with bulk ACKs
and frequent reordering, or a cellular WWAN~\cite
{kuhn2021quic-over-sat,goyal2017abc}.

By the 1990s, many networks had broken from the canonical model by deploying
in-network TCP accelerators, also known as ``performance-enhancing proxies''
(PEPs)~\cite{rfc3135}. TCP PEPs can split an end-to-end connection into
multiple concatenated connections~\cite
{kapoor2005achieving,caini2006pepsal,davern2011httpep,farkas2012splittcp,hayes2019mmwave},
buffer and retransmit packets over a lossy link~\cite
{balakrishnan1995snoop,polese2017milliproxy}, virtualize congestion
control~\cite{cronkite2016vcc,he2016acdc,mihaly2012mobilePEP}, resegment the
byte stream, and enable forward error correction, explicit congestion
notification, or other segment-specific enhancements. Because TCP isn't
encrypted or authenticated, PEPs can achieve this transparently, without the
knowledge or cooperation of end hosts. Roughly 20--40\% of Internet paths cross
at least one TCP PEP~\cite{imc2011handley, edeline2019bottomup}.

While many flows benefit from PEPs, their use carries a cost: protocol
ossification~\cite{papastergiou2017deossifying, edeline2019bottomup}. When a
middlebox inserts itself in a connection and enforces its preconceptions about
the transport protocol, it can thwart the protocol's evolution, dropping
traffic that uses an upgraded version or new options. TCP PEPs have hindered or
complicated the deployment of many TCP improvements, such as ECN++, tcpcrypt,
TCP extended options, and multipath TCP~\cite
{mandalari2018ecnplusplus,imc2011handley,raiciu2012multipathtcp}.

In response to this ossification, and to an increased emphasis on privacy and
security, post-TCP transport protocols have been designed to be impervious to
meddling middleboxes, by encrypting and authenticating the transport header. We
call these newer transport protocols ``opaque.'' The most prevalent is
QUIC~\cite{rfc9000}, found in billions of installed Web browsers and millions
of servers~\cite{zirngibl2021quicdeployment}; other opaque transport protocols
are used in WebRTC/SRTP~\cite{rfc8834webrtc}, Zoom~\cite
{zoom}, BitTorrent~\cite{bittorrent}, and Mosh/SSP~\cite{winstein2012mosh}.

This opacity means that middleboxes can't interpose themselves on a connection
or understand the sequence numbers of packets in transit.  This prevents PEPs
from providing assistance, reducing---in some situations---the performance of
opaque transport protocols~\cite
{border2020quicsat-presentation,kuhn2021quic-over-sat,martin2022bbr-quic-sat,border2022evaluating,kosek2022quicpep}.
It's possible to co-design protocols and PEPs to preserve security and privacy
while permitting assistance from credentialed middleboxes~\cite
{ford2008logjam,sherry2015blindbox, dogar2012tapa,iyengar2009flow}, but
challenging to do so without tightly coupling these components, risking
ossification and fragility.
