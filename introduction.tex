\chapter{Introduction}
\label{sec:introduction}

This thesis will refer to many concepts about the Internet, both old and new.
It is important to understand why things in the Internet were designed the way
they were, so that we have the proper context for why they no longer fit with
how we use the Internet today. In the very beginning of the Internet in the
1960s, ARPANET was a small research network of four stationary hosts that
communicated over wired links. ARPANET introduced the concept of packet
switching, by delivering large amounts of data as small, individual packets.

In the Internet's canonical model of the 1970s, transport is end-to-end and
implemented only in hosts. Traditionally, routers and other network components
forwarded IP datagrams without regard to their payloads or flow
membership~\cite{saltzer1984endtoend, clark1988darpa}; only hosts thought about
connections, reliable delivery, or flow-by-flow congestion control, typically
using TCP. The first congestion control algorithms from the 1980s treated
packet loss as congestion, backing off the sending rate in fear of congestion
collapse.

The beginnings of the mainstream, commercial Internet were marked by the
World-Wide Web in the 1990s. From there, Wi-Fi expanded mobility for devices in
the home and workspace, and cellular networks along with the popularity of
Apple and Android phones expanded mobility everywhere else. Satellite networks,
which long provided Internet access to more remote and rural regions, are being
made more accessible today with low-Earth orbit satellites such as Starlink.
The combination of wireless networking and the expansion of international and
extraterrestrial links means our networks are more lossy and high-delay
compared to the beginnings of ARPANET.

The applications we use the Internet for also demand more from the network than
ever before. The first Internet applications included e-mail and web browsing,
and now we have expanded into high-bandwidth services such as streaming and
content sharing, as well as interactive low-latency services such as video
conferencing. The Internet today still uses TCP for transport, IP to route
packets, and ``loss-based'' congestion control algorithms. Have these concepts
survived the test of time for the networks and applications of the Internet
today?

Let's look at it in the context of QUIC, a relatively new transport protocol
invented at Google in 2012 and standardized in 2021. If you have used Chrome,
watched YouTube, or interacted with some other Google service, you have
probably used QUIC. QUIC in some ways can be thought of as an alternative to
TCP, offering ordering and reliability in the transport layer.

Now imagine you're on a moving train. Your laptop is connected to the router on
the train via a lossy, low-latency Wi-Fi link, and the router is connected to
the rest of the Internet via a more reliable, high-latency cellular path. You
upload a large file using QUIC and find it to be incredibly slow. Packets being
dropped on the wireless link require a full round-trip time to invoke a
retransmission, and the ``loss-based'' congestion control algorithm on your
laptop interprets the end-to-end signals as a lossy, high-latency path,
reducing the sending rate dramatically.

However, thanks to the deployment of performance-enhancing proxies (PEPs) in the
1990s, your file upload using TCP is actually very fast. These proxies exist in
the middle of the network, at satellite ground stations, cellular base
stations, or even the router on a moving train. The most common type of PEP is
the connection-splitting TCP PEP. This PEP splits an end-to-end connection into
two separate TCP connections. This allows the PEP to invoke retransmissions
from closer to the data sender, as well as tailor the congestion control
according to the properties of each path segment. TCP PEPs are still prevalent
in the network today. It is estimated that 20-40\% of Internet paths containing
a connection-splitting TCP PEP, even more in satellite and wireless networks.

These proxies cannot help QUIC because QUIC is what we call a ``secure''
transport protocol. Since TCP headers are unencrypted, an in-network proxy can
actively manipulate the transport layer, intercepting the initialization
packets and splitting the connection without knowledge of the endpoints. In
comparison, QUIC and many other post-TCP transport protocols now completely
encrypt their transport headers on the wire, preventing the PEP from helping
and harming the performance of the connection.

If proxies help performance by manipulating the transport layer, then why do
protocols choose to encrypt the transport layer? First, when TCP/IP was first
invented, ARPANET was just a small network of trusted research institutions,
where privacy and security were not design priorities. Transport protocols
these days prefer to hide unessential information from proxies using
encryption. Second, proxies have caused protocol ossification and prevented
protocols like TCP from evolving over time. To provide in-network assistance,
proxies have to make assumptions about what fields the transport layer should
or should not have, and alter or block traffic based on these fields, fixing
their functionality in place. Protocol ossification has prevented TCP from
deploying many proposed extensions and changes to the protocol over the last
few decades.

This thesis is about resolving the tension between performance and ossification
in secure transport protocols. \textit{I will present the
Sidekick approach to in-network assistance and show that it can yield many of
the same performance benefits of traditional PEPs, without protocol
ossification.} In this approach, proxies and endpoints send information on an
adjacent connection about which packets they have received. The information is
called a quACK, and it applies set reconciliation techniques in a novel setting
to efficiently refer to encrypted packets without plaintext sequence numbers.
Transport protocols are unmodified on the wire, leaving them free to evolve.

In the rest of this dissertation, I will start by presenting quACKs
(\Cref{sec:quack}), a tool for efficiently referring to a set of randomly
encrypted packets that a proxy or endpoint has received. Next, I will describe
how proxies use the Sidekick (\Cref{sec:sidekick}) protocol to send quACKs over
an adjacent connection so the endpoint can modify its behavior to emulate
traditional PEPs. Similarly, the endpoint sends quACKs to the proxy using the
Packrat (\Cref{sec:packrat}) protocol to receive in-network retransmissions,
also achieving a variety of performance benefits. Finally, I will take a step
and analyze why connection-splitting is still relevant today even with recent
developments such as BBR and QUIC (\Cref{sec:splitting}) and conclude
(\Cref{sec:conclusion}).


% \section{QuACKs: Referring to encrypted packets without sequence numbers}
% \label{sec:introduction:quacks}

% One of the main challenges of providing in-network assistance to secure
% transport protocols is how to usefully refer to packets when they do not have
% plaintext sequence numbers like TCP. In the first part of the dissertation,
% I will describe the quACK, a tool for efficiently referring to a set of
% randomly encrypted packets that a proxy or endpoint has received. The quACK
% applies set reconciliation techniques from more theoretical literature to
% communicate a small amount of information with efficient packet encoding and
% decoding times.

% \section{Sidekick: Secure in-network assistance for data senders}
% \label{sec:introduction:sidekick}

% Then, I will describe the Sidekick protocol for providing in-network assistance
% to data senders using secure transport protocols.
% The Sidekick approach is a fundamentally different type of architecture for
% in-network assistance. Rather than actively manipulating the connection on the
% wire by modifying and rewriting packets, the proxy simply observes the
% underlying connection, which we refer to as the base protocol. The proxy sends
% quACKs to the endpoint describing which packets it has received, and the
% endpoint makes path-aware decisions to enhance the performance of the
% connection while making the transport decisions end-to-end.

% \section{Packrat: Secure in-network retransmissions for data receivers}
% \label{sec:introduction:packrat}

% Next, I will present the Packrat protocol for helping secure transport protocols
% by sending in-network retransmissions to the endpoint. This scenario differs
% from that of the Sidekick protocol because the lossy path segment is near the
% data receiver instead of the data sender. This is challenging because the data
% sender dictates the terms of the data transfer, such as the sending rate and
% end-to-end retransmissions. However, there are also advantages in that the
% quACK sender is now co-located with the end-to-end ACK sender.

% \section{The future of connection-splitting with BBR and QUIC}
% \label{sec:introduction:heuristic}

% Taking a step back, recall that
% connection-splitting PEPs peaked in the late 1990s as a response to lossy,
% high-delay networks, and this was possible because TCP was unencrypted.
% Their utility today is complicated by post-TCP transport protocols such as QUIC
% that totally encrypt the transport layer on the wire. Sometimes what's old is
% new again, and sometimes what's old just needs to stay old.

% One of the core reasons QUIC was so slow in our train scenario was because of
% its congestion control algorithm's response to loss. In 2016, we had another
% advancement, also from Google, in the form of a new type of congestion control
% algorithm which is BBR. BBR was designed specifically for lossy, high-delay
% networks and is presumed to have much better performance. With options that are
% seemingly appealing for both ossification and performance, do recent
% developments such as BBR and QUIC truly make connection-splitting obsolete?

% We perform an emulation study of modern transport protocols and congestion
% control schemes both with and without connection-splitting PEPs, applying the
% \textit{split throughput heuristic}, and find that congestion control schemes do
% still benefit from connection-splitting today and the degree varies
% significantly by implementation. We present several takeaways regarding how to
% refer to end-to-end congestion control schemes and also hope this study
% motivates the need for protocol-agnostic forms of in-network assistance that
% emulate connection-splitting PEPs in the future.
