\section{Motivating scenarios}
\label{sec:sidekick:motivating}

\begin{figure}[t]
	\centering
	\includegraphics[width=0.8\linewidth]{sidekick/figures/sc_protocol.pdf}
\caption{The proxy generates quACKs, in-network acknowledgments, based on
the encrypted packets it observes in the base protocol. It quACKs to an end
host, the data sender, which sends or resends packets on the base protocol as a result.
Although we only show one side of the connection, the Sidekick could assist
either end host of a bidirectional flow.
}
\label{fig:sidekick:overview}
\end{figure}


We focus on three scenarios where end hosts benefit from in-network assistance.
In each one, a proxy provides feedback in the form of a quACK to an end
host: the data sender (\Cref{fig:sidekick:overview}). Recall that a quACK is a
``cumulative ACK + selective ACK'' over encrypted packet identifiers. The data
sender uses this feedback to influence its behavior on the base connection,
without altering the wire format.

To be clear: the Sidekick protocol is not tied to a specific base protocol
nor to how the end hosts use the quACK information. The base protocol does not
need to be reliable, nor to have unique datagrams---we implemented and evaluated
the same Sidekick protocol and the same proxy behavior across the different
scenarios in this paper.

\subsection{Low-latency media retransmissions}
\label{sec:sidekick:motivating:media}

Consider a train passenger using on-board Wi-Fi to have a low-latency audio
conversation, using WebRTC/SRTP~\cite{rfc8834webrtc}, with a friend. The
end-to-end network path contains a low-latency, high-loss ``near'' path
segment (the Wi-Fi hop) followed by a high-latency, low-loss ``far'' path
segment (the cellular and wired path over the Internet). The friend probably
suffers from poor connection quality, experiencing drops in the audio stream or
high de-jitter buffer delays from waiting for retransmitted packets to be
played in order (\Cref{fig:sidekick:real-world:media} in the real world,
\Cref{fig:sidekick:main-results:media} in emulation).

In the Sidekick approach, a PEP on the Wi-Fi access point sends quACKs to
the media client on the passenger's laptop, assisting the base connection.
The passenger's media client uses quACKs to retransmit packets sooner than they
would have using negative acknowledgments (NACKs) from the friend. The
end result is similar to the effect of PEPs such as
Snoop~\cite{balakrishnan1995snoop} and Milliproxy~\cite{polese2017milliproxy}
that leverage TCP's cleartext sequence numbers to trigger early retransmission
on lossy wireless paths.

\subsection{Emulating split congestion control in an HTTP/3 file upload}
\label{sec:sidekick:motivating:http}

Consider the same train passenger as before but uploading a large file over the
Internet with a reliable transport protocol. If the protocol were TCP, the
train could deploy a split TCP PEP at the access point. The split connection
allows quick detection and retransmission of dropped packets on the lossy Wi-Fi
segment, while opening up the congestion window on the high-latency cellular
segment.

However, secure transport protocols like QUIC can't benefit from (nor be harmed
by) connection-splitting PEPs. Without a PEP, QUIC relies on end-to-end
mechanisms over the entire path to detect losses, recover from them, and adjust
the congestion-control behavior. This leads to reduced upload speeds
(\Cref{fig:sidekick:real-world:pep-emulation} in the real world,
\Cref{fig:sidekick:main-results:pep-emulation} in emulation).

With help from the same Sidekick PEP, the QUIC client combines information from
quACKs and end-to-end ACKs to emulate the congestion-control behavior of a
split TCP connection (\Cref{sec:sidekick:pacubic}). The
application considers whether packets are lost on the near or far path
segments, and adjusts the congestion window accordingly while respecting the
opacity of the end-to-end base connection. The application also retransmits the
packet as soon as the loss has been detected.

The only guarantee the proxy makes to the client via the quACK is that it has
received some packets. To respect the end-to-end reliability contract with the
server, the client does not delete packets that may need to be transmitted
until it receives an end-to-end ACK from the server, even if the packet has
been quACKed by the proxy.

\subsection{Battery-saving ACK reduction}
\label{sec:sidekick:motivating:ack-reduction}

Now consider a battery-powered device downloading a large file from the
Internet. To reduce how often the device's
radio needs to wake up, saving energy, the base connection can reduce the
frequency of end-to-end ACKs the device sends to the data sender.
ACK reduction has also been shown to improve performance by reducing collisions
and contention over half-duplex links~\cite{custura2023reducing,li2020tack}.
The ACK frequency can be configured with a TCP kernel setting or proposed
QUIC extension~\cite{ietf-quic-ack-frequency-07}.

However, ACK reduction can also degrade
throughput~\cite{custura2023reducing,custura2020impact}
(\Cref{fig:sidekick:main-results:ack-reduction} in emulation).
The data sender receives more delayed feedback about loss, and has to carefully
pace packets to avoid bursts in the large delay between ACKs.
One proposal has the PEP acknowledge packets on behalf of the data
receiver~\cite{kliazovich2012arqproxy}, leveraging cleartext TCP sequence
numbers. This proposal does not apply to secure transport protocols without
these sequence numbers.

In this scenario, a Sidekick proxy at a Wi-Fi access point or cellular base station
quACKs to the data sender on behalf of the battery-powered device. The device still occasionally
wakes up its radio to send ACKs, but the data sender uses the more frequent quACKs
from the proxy to advance its flow-control and congestion-control windows.

The data sender respects the end-to-end reliability contract by only deleting packets
in response to end-to-end ACKs, but disregards the data receiver's flow control by using quACKs
to advance the flow-control window. If the data sender only used end-to-end ACKs to advance the
window, it would waste time waiting between ACKs to send packets with too small
a window, and need to pace sent packets on receiving a large ACK with too large
a window.
