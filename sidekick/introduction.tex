\section{Introduction}
\label{sec:sidekick:intro}

In the Internet's canonical model, transport is end-to-end and implemented only
in hosts. Traditionally, routers and other network components forwarded IP
datagrams without regard to their payloads or flow membership~\cite
{saltzer1984endtoend, clark1988darpa}; only hosts thought about connections,
reliable delivery, or flow-by-flow congestion control.

In practice, however, the best behavior for a transport protocol depends on the
particulars of the network path. An appropriate retransmission or
congestion-control scheme for a heavily-multiplexed wired network wouldn't be
ideal for paths that include a high-delay satellite link, Wi-Fi with bulk ACKs
and frequent reordering, or a cellular WWAN~\cite
{kuhn2021quic-over-sat,goyal2017abc}.

By the 1990s, many networks had broken from the canonical model by deploying
in-network TCP accelerators, also known as ``performance-enhancing proxies''
(PEPs)~\cite{rfc3135}. TCP PEPs can split an end-to-end connection into
multiple concatenated connections~\cite
{kapoor2005achieving,caini2006pepsal,davern2011httpep,farkas2012splittcp,hayes2019mmwave},
buffer and retransmit packets over a lossy link~\cite
{balakrishnan1995snoop,polese2017milliproxy}, virtualize congestion
control~\cite{cronkite2016vcc,he2016acdc,mihaly2012mobilePEP}, resegment the
byte stream, and enable forward error correction, explicit congestion
notification, or other segment-specific enhancements. Because TCP isn't
encrypted or authenticated, PEPs can achieve this transparently, without the
knowledge or cooperation of end hosts. Roughly 20--40\% of Internet paths cross
at least one TCP PEP~\cite{imc2011handley, edeline2019bottomup}.

While many flows benefit from PEPs, their use carries a cost: protocol
ossification~\cite{papastergiou2017deossifying, edeline2019bottomup}. When a
middlebox inserts itself in a connection and enforces its preconceptions about
the transport protocol, it can thwart the protocol's evolution, dropping
traffic that uses an upgraded version or new options. TCP PEPs have hindered or
complicated the deployment of many TCP improvements, such as ECN++, tcpcrypt,
TCP extended options, and multipath TCP~\cite
{mandalari2018ecnplusplus,imc2011handley,raiciu2012multipathtcp}.

In response to this ossification, and to an increased emphasis on privacy and
security, post-TCP transport protocols have been designed to be impervious to
meddling middleboxes, by encrypting and authenticating the transport header. We
refer to these newer transport protocols as ``secure'' transport protocols. The
most prevalent is QUIC~\cite{rfc9000}, found in billions of installed Web
browsers and millions of servers~\cite{zirngibl2021quicdeployment}; other secure
transport protocols are used in WebRTC/SRTP~\cite{rfc8834webrtc},
Zoom~\cite{zoom}, BitTorrent~\cite{bittorrent}, and
Mosh/SSP~\cite{winstein2012mosh}.

This encryption means that middleboxes can't interpose themselves on a
connection or understand the sequence numbers of packets in transit. This
prevents PEPs from providing assistance, reducing---in some situations---the
performance of secure transport protocols~\cite
{border2020quicsat-presentation,kuhn2021quic-over-sat,martin2022bbr-quic-sat,border2020evaluating,kosek2022quicpep}.
It's possible to co-design protocols and PEPs to preserve security and privacy
while permitting assistance from credentialed middleboxes~\cite
{ford2008logjam,sherry2015blindbox, dogar2012tapa,iyengar2009flow}, but
challenging to do so without tightly coupling these components, risking
ossification and fragility.

In this chapter, we propose a method for in-network assistance of secure transport
protocols that tries to resolve this tension. Our approach leaves the transport
protocol unchanged on the wire: an encrypted end-to-end connection between hosts,
opaque to middleboxes and free to evolve. No PEPs are credentialed to decrypt
the transport protocol's headers.

Instead, we propose a second protocol to be spoken on an adjacent connection
between an end host and a PEP. We call this the \emph{\bf Sidekick protocol},
and its contents are \emph{about} the packets of the underlying, or ``base,''
connection. Sidekick PEPs assist end hosts by reporting what they've observed
about the packets of the encrypted base connection, without coupling their
assistance to the details of the base protocol. End hosts use this information
to influence decisions about how and when to send or resend packets on the base
connection, approximating some of the performance benefits of traditional PEPs.
We first proposed a similar functional separation in \cite{yuan2022sidecar},
and presented a concrete realization of the idea and its nuanced
interactions with real transport protocols in \cite{yuan2024sidekick}.

One key technical challenge with this approach is how the Sidekick can
efficiently refer to ranges of packets in an encrypted base connection. These
packets appear random to the middlebox, and referring to a range of, e.g., 100
encrypted packets in the presence of loss and reordering is not as simple as
saying ``up to 100'' when there are cleartext sequence numbers. In \Cref
{sec:quack}, we presented and evaluated a mathematical tool called a \emph
{\bf quACK} that concisely represents a selective acknowledgment of randomly
encrypted packets. In this chapter, we leverage the power sum quACK,
based on the insight that we can model the problem as a system of power
sum polynomial equations if there is a practical bound on the maximum number of
``holes'' among the packets being ACKed.

A second challenge is how the end host should use information from a Sidekick
connection to obtain a performance benefit for its base connection. Since the
performance benefit comes from changing behavior at the end host rather than
the middlebox, transport protocols need to incorporate this information into
their existing algorithms for, e.g., loss detection and retransmission, which
have gotten increasingly complex over time. To explore this, we designed a
Sidekick protocol and integrated it into client implementations in three scenarios:
\begin{itemize}[noitemsep,topsep=2pt]
\item A low-latency audio stream over an Internet path that includes a Wi-Fi
  path segment (low latency with loss), followed by a WAN path segment (higher
  latency with low loss). Can the Sidekick PEP reduce the de-jitter buffer delay
  by triggering earlier retransmissions on loss?

\item An upload over the same path. Can a secure transport protocol like QUIC,
  aided by a Sidekick PEP at the point between these two path segments, match
  the throughput of TCP over a connection-splitting PEP?

\item A battery-powered receiver, downloading data from the Internet over Wi-Fi.
  If the Wi-Fi access point sends Sidekick quACKs on behalf of the receiver,
  can it reduce the number of times the receiver's radio needs to wake up
  to send an end-to-end ACK?
\end{itemize}

\smallskip

A third technical challenge is how knowledge about \emph{where}
loss occurs along a path should influence a congestion-control scheme.
The challenge in any such scheme is how to maximize the congestion window
while sharing the network fairly with competing flows.
We present a path-aware modification to the CUBIC congestion-control
algorithm~\cite{ha2008cubic}, which we call \mbox{\textbf{PACUBIC}},
that approximates the congestion-control behavior of a PEP-assisted split TCP
CUBIC connection while making its decisions entirely on the host.

\paragraph{Summary of results.}

We implemented the Sidekick protocol in a low-latency media client
based on the WebRTC standard, and an HTTP/3 client using the Cloudflare
implementation of QUIC~\cite{quiche} and the \texttt{libcurl}~\cite{libcurl}
implementation of HTTP/3. We evaluated the three scenarios in
real-world and emulation experiments.
In real-world experiments using an unmodified local Wi-Fi network to access our
nearest AWS datacenter, the Sidekick was able to trigger early retransmissions
to fill in gaps in the audio of a latency-sensitive audio stream, reducing the
receiver's de-jitter delay from 2.3~seconds to 204~ms---about a 91\% reduction
(\Cref{fig:sidekick:real-world}). The Sidekick was also able to improve the
speed of an HTTP/3 (QUIC) upload by about 50\%.

In emulation experiments of the ``battery-powered receiver'' scenario,
the Sidekick PEP was able to reduce the need for the receiver to send ACKs
by sending proxy acknowledgments on its behalf---ACKs the sender used
to advance its flow-control and congestion-control windows. The
receiver only needed to wake up its radio to send occasional
end-to-end ACKs, which the sender used to discard data from its
buffer (\Cref{fig:sidekick:main-results:ack-reduction}).

Also in an emulation experiment, we confirmed that PACUBIC's
performance approximates a split CUBIC connection (two TCP CUBIC
connections separated by a PEP), responding to loss events on the
different path segments similarly to how the individual CUBIC flows would
(\Cref{fig:sidekick:fairness-line}). The results indicate that the Sidekick
protocol's gains do not come at the
expense of congestion-control fairness relative to a split CUBIC connection.

\smallskip

The rest of this chapter discusses three motivating scenarios for the Sidekick
(\Cref{sec:sidekick:motivating}), describes the concrete Sidekick protocol we
built around power sum quACKs (\Cref{sec:sidekick:design}) and
its implementation in two base protocols (\Cref{sec:sidekick:implementation}),
and then evaluates the protocol in real-world (\Cref{sec:sidekick:real-world})
and emulation experiments (\Cref{sec:sidekick:emulation}).
