\chapter[Sidekick Protocols]{Sidekick Protocols: Secure in-network assistance for data senders}
\label{sec:sidekick}

This chapter describes \emph{Sidekick protocols}: an approach to in-network
assistance for secure transport protocols. In this approach, in-network
intermediaries help endpoints by sending information adjacent to the underlying
connection, which remains opaque and unmodified on the wire. Described in \Cref
{sec:quack}, this information is called a quACK, and the quACK is the key
technical tool that allows proxies to usefully refer to a set of packets that
they have received even when the transport headers are randomly encrypted. In
real-world and emulation-based evaluations, the Sidekick protocol improves
performance in several scenarios: early retransmission over lossy Wi-Fi paths,
proxy acknowledgments to save energy, and a path-aware congestion-control
mechanism we call \emph{PACUBIC}. The path-aware CUBIC congestion control
algorithm emulates the congestion response of a ``split'' connection from the
perspective of the data sender at the endpoint, without buffering packets at
the proxy.

\section{Introduction}
\label{sec:sidekick:intro}

In the Internet's canonical model, transport is end-to-end and implemented only
in hosts; only hosts think about connections, reliable delivery, or
flow-by-flow congestion control (\Cref{sec:introduction}).
In practice, however, the best behavior for a
transport protocol depends on the particulars of the network path. An
appropriate retransmission or congestion-control scheme for a
heavily-multiplexed wired network wouldn't be ideal for paths that include a
high-delay satellite link, Wi-Fi with bulk ACKs and frequent reordering, or a
cellular WWAN~\cite{kuhn2021quic-over-sat,goyal2017abc}.

By the 1990s, many networks had broken from the canonical model by deploying
in-network TCP accelerators, also known as ``performance-enhancing proxies''
(PEPs)~\cite{rfc3135}. TCP PEPs can split an end-to-end connection into
multiple concatenated connections~\cite
{kapoor2005achieving,caini2006pepsal,davern2011httpep,farkas2012splittcp,hayes2019mmwave},
buffer and retransmit packets over a lossy link~\cite
{balakrishnan1995snoop,polese2017milliproxy}, virtualize congestion
control~\cite{cronkite2016vcc,he2016acdc,mihaly2012mobilePEP}, resegment the
byte stream, and enable forward error correction, ECN,
or other segment-specific enhancements. Because TCP isn't
encrypted or authenticated, PEPs can achieve this without the
knowledge or cooperation of end hosts. Roughly 20--40\% of Internet paths cross
at least one TCP PEP~\cite{honda2011still, edeline2019bottomup}.

While many flows benefit from PEPs, their use carries a cost: protocol
ossification~\cite{papastergiou2017deossifying, edeline2019bottomup}.
% When a
% middlebox inserts itself in a connection and enforces its preconceptions about
% the transport protocol, it can thwart the protocol's evolution, dropping
% traffic that uses an upgraded version or new options. TCP PEPs have hindered or
% complicated the deployment of many TCP improvements, such as ECN++, tcpcrypt,
% TCP extended options, and Multipath TCP~\cite
% {mandalari2018ecnplusplus,honda2011still,raiciu2012multipathtcp}.
In response,
% to this ossification, and to an increased emphasis on privacy and security,
post-TCP transport protocols have been designed to be impervious to
meddling middleboxes by encrypting and authenticating the transport header. We
refer to these newer transport protocols as ``opaque'' transport protocols. The
most prevalent is QUIC~\cite{rfc9000}, found in billions of installed Web
browsers and millions of servers~\cite{zirngibl2021quicdeployment}; other secure
transport protocols are used in WebRTC/SRTP~\cite{rfc8834webrtc},
Zoom~\cite{zoom}, BitTorrent~\cite{bittorrent}, and
Mosh/SSP~\cite{winstein2012mosh}.

This opacity means that middleboxes can't interpose themselves on a
connection or understand the sequence numbers of packets in transit. This
prevents PEPs from providing assistance, in some situations reducing the
performance of opaque transport protocols~\cite
{border2020quicsat-presentation,kuhn2021quic-over-sat,martin2022suitability,border2020evaluating,kosek2022quicpep}.
It's possible to co-design protocols and PEPs to permit assistance from
credentialed middleboxes~\cite{ford2008logjam,sherry2015blindbox, dogar2012tapa,iyengar2009flow},
but challenging to do so without tightly coupling these components and risks
ossification and fragility. The resulting tension between the performance
enhancements of PEPs and protocol ossification makes it challenging to provide
in-network assistance to secure transport protocols.

In this chapter, we propose a
second protocol called the \emph{\bf Sidekick protocol} to be spoken on an
adjacent connection between an end host and a PEP. The contents of the adjacent
connection are \emph{about} the packets of the underlying ``base'' connection.
Sidekick proxies assist end hosts by reporting what they've observed about the
packets of the encrypted base connection, leaving the transport protocol
unchanged on the wire: an encrypted end-to-end connection between hosts, opaque
to middleboxes and free to evolve. End hosts use this information to influence
decisions about how and when to send packets on the base connection,
approximating some of the performance benefits of traditional PEPs.
No PEPs are credentialed to decrypt the transport protocol's headers.

% Instead, we propose a second protocol to be spoken on an adjacent connection
% between an end host and a PEP. We call this the \emph{\bf Sidekick protocol},
% and its contents are \emph{about} the packets of the underlying, or ``base,''
% connection.
% Sidekick PEPs assist end hosts by reporting what they've observed
% about the packets of the encrypted base connection, without coupling their
% assistance to the details of the base protocol. End hosts use this information
% to influence decisions about how and when to send or resend packets on the base
% connection, approximating some of the performance benefits of traditional PEPs.
% We first proposed a similar functional separation in \cite{yuan2022sidecar},
% and presented a concrete realization of the idea and its nuanced
% interactions with real transport protocols in \cite{yuan2024sidekick}.

One key technical challenge in this approach is how Sidekick proxies can
efficiently refer to a set of packets in an encrypted base connection when
these packets appear random to the middlebox. Referring to a range of, say, 100
encrypted packets in the presence of loss and reordering is not as simple as
saying ``up to 100'' in the case of packets with cleartext sequence numbers. To
solve this problem, we leverage the power sum quACK from \Cref{sec:quack}. The
{\bf quACK} is a mathematical tool that concisely represents a selective
acknowledgment of encrypted packets when there is a practical bound on the
maximum number of ``holes'' among the packets being ACKed.

A second challenge is how the end host should use information from a Sidekick
proxy to obtain a performance benefit for its base connection. Since the
performance benefit comes from changing behavior at the end host rather than
the middlebox, transport protocols need to incorporate this information into
their existing algorithms for, e.g., loss detection and retransmission, which
have gotten increasingly complex over time. To explore this, we designed a
Sidekick protocol and integrated it into client implementations with path-aware
modifications in three scenarios:
\begin{itemize}[noitemsep,topsep=2pt]
\item A low-latency audio stream over an Internet path that includes a lossy,
  low-latency Wi-Fi path segment, followed by a reliable, higher-latency WAN
  path segment. Can using the Sidekick protocol reduce the de-jitter buffer
  delay by triggering earlier retransmissions on loss?

\item An upload over the same path. Can a secure transport protocol like QUIC,
  aided by a Sidekick proxy at the point between these two path segments, match
  the throughput of TCP over a connection-splitting PEP?

\item A battery-powered receiver downloading data from the Internet over Wi-Fi.
  If the Wi-Fi access point sends quACKs over a Sidekick connection on behalf of
  the receiver,
  can it reduce the number of times the receiver's radio needs to wake up
  to send an end-to-end ACK?
\end{itemize}

\smallskip

A third technical challenge is how knowledge about \emph{where}
loss occurs along a path should influence a congestion-control scheme.
The challenge in any such scheme is how to maximize the congestion window
to achieve high throughput
while sharing the network fairly with competing flows.
We present a path-aware modification to the CUBIC congestion-control
algorithm~\cite{ha2008cubic}, which we call \mbox{\textbf{PACUBIC}},
that approximates the congestion-control behavior of a PEP-assisted split TCP
CUBIC connection while making its decisions entirely on the host.

\paragraph{Summary of results.}

We evaluated the three scenarios above, integrating the Sidekick protocol with a
media client based on the WebRTC standard and an HTTP/3 client using
\texttt{libcurl} and the Cloudflare implementation of QUIC~\cite{quiche}.
In real-world experiments using an unmodified local Wi-Fi network to access our
nearest AWS datacenter, the Sidekick proxy was able to trigger early retransmissions
to fill in gaps in the audio of a latency-sensitive audio stream, reducing the
receiver's de-jitter delay from 2.3~seconds to 204~ms---about a 91\% reduction
(\Cref{fig:sidekick:real-world}). The Sidekick proxy was also able to improve the
speed of an HTTP/3 (QUIC) upload by about 50\%.

In emulation experiments of the ``battery-powered receiver'' scenario,
the Sidekick proxy was able to reduce the need for the receiver to send ACKs
by sending proxy acknowledgments on its behalf---ACKs the sender used
to advance its flow-control and congestion-control windows. The
receiver only needed to wake up its radio to send occasional
end-to-end ACKs, which the sender used to discard data from its
buffer (\Cref{fig:sidekick:main-results:ack-reduction}).

Also in an emulation experiment, we confirmed that PACUBIC's achieved throughput
approximates that of a split CUBIC connection (two TCP CUBIC
connections separated by a PEP), responding to loss events proportionally to
the delay of the path segment on which the loss occurs
(\Cref{fig:sidekick:fairness-line}). The results indicate that the gains from
using the Sidekick protocol do not come at the
expense of congestion-control fairness relative to a split CUBIC connection.\\

\noindent
The rest of this chapter discusses the three motivating scenarios for Sidekick
protocols in more detail (\Cref{sec:sidekick:motivating}). We describe the
concrete Sidekick protocol and path-aware sender modifications we built around
power sum quACKs (\Cref{sec:sidekick:design}), including path-aware CUBIC
congestion control (\Cref{sec:sidekick:pacubic}) and its integrations with two
base protocols (\Cref{sec:sidekick:implementation}). We evaluate
(\Cref{sec:sidekick:methodology}) the use of the protocol in real-world
(\Cref{sec:sidekick:real-world}) and emulation experiments
(\Cref{sec:sidekick:emulation}). Finally, we discuss limitations to our approach
(\Cref{sec:sidekick:limitations}) and conclude (\Cref{sec:sidekick:summary}).

\section{Motivating scenarios}
\label{sec:sidekick:motivating}

\begin{figure}[t]
	\centering
	% \includegraphics[width=\linewidth]{figures/sc-legend.pdf}\\
	\includegraphics[width=\linewidth]{figures/sc_protocol.pdf}%
\caption{The proxy generates quACKs, in-network acknowledgments, based on
the opaque packets it observes in the base protocol. It quACKs to an end
host, the data sender, which sends or resends packets on the base protocol as a result.
Although we only show one side of the connection, the \sys could assist
either end host of a bidirectional flow.
\vspace{-0.4cm}
}
\label{fig:sc-protocols}
\end{figure}


We focus on three scenarios where end hosts benefit from in-network assistance.
In each one, a proxy server provides feedback in the form of a quACK to an end
host: the data sender (\Cref{fig:sc-protocols}). Recall that a quACK is a
``cumulative ACK + selective ACK'' over encrypted sequence numbers. The data
sender uses this feedback to influence its behavior on the base connection,
without altering the wire format.

To be clear: the Sidekick protocol is not tied to a specific base protocol
nor to how the end hosts use the quACK information. The base protocol does not
need to be reliable, nor to have unique datagrams---we implemented and evaluated
the same Sidekick protocol and the same proxy behavior across the different
scenarios in this paper.

\subsection{Low-latency media retransmissions}
\label{sec:sidekick:motivating:media}

Consider a train passenger using on-board Wi-Fi to have a low-latency audio
conversation, using WebRTC/SRTP~\cite{rfc8834webrtc}, with a friend. The
end-to-end network path contains a low-latency, high-loss ``near'' path
segment (the Wi-Fi hop) followed by a high-latency, low-loss ``far'' path
segment (the cellular and wired path over the Internet). The friend probably
suffers from poor connection quality, experiencing drops in the audio stream or
high de-jitter buffer delays from waiting for retransmitted packets to be
played in order (\Cref{fig:real-world:scenario1} in the real world,
\Cref{fig:media} in emulation).

In the Sidekick approach, a Sidekick on the Wi-Fi access point sends quACKs to the audio
application on the user's laptop, assisting the base connection's data sender.
The sender uses quACKs to retransmit packets sooner than they would have using
negative acknowledgments (NACKs) from the receiver. The end result is similar
to the effect of prior PEPs, such as Snoop~\cite{balakrishnan1995snoop} and
Milliproxy~\cite{polese2017milliproxy}, that leverage TCP's cleartext sequence
numbers to trigger early retransmission on lossy wireless paths.

\subsection{Emulating split congestion control in an HTTP/3 file upload}
\label{sec:sidekick:motivating:http}

Consider the same train passenger as before but uploading a large file over the
Internet with a reliable transport protocol. If the protocol were TCP, the
train could deploy a split TCP PEP at the access point. The split connection
allows quick detection and retransmission of dropped packets on the lossy Wi-Fi
segment, while opening up the congestion window on the high-latency cellular
segment.

However, secure transport protocols like QUIC can't benefit from (nor be harmed
by) connection-splitting PEPs. Without a PEP, QUIC relies on end-to-end
mechanisms over the entire path to detect losses, recover from them, and adjust
the congestion-control behavior. This leads to reduced upload speeds
(\Cref{fig:real-world:scenario2} in the real world, \Cref{fig:baseline-line} in
emulation).

With help from the same Sidekick PEP, the QUIC sender combines information from
quACKs and end-to-end ACKs to emulate the congestion-control behavior of a
split TCP connection (\Cref{sec:sidekick:sender}). The
application considers whether packets are lost on the near or far path
segments, and adjusts the congestion window accordingly while respecting the
opacity of the end-to-end base connection. The application also retransmits the
packet as soon as the loss has been detected.

The only guarantee the proxy makes to the sender via the quACK is that it has
received some packets. To respect the end-to-end reliability contract with the
receiver, the sender does not delete packets that may need to be transmitted
until it receives an ACK, even if the packet has been quACKed.

\subsection{Battery-saving ACK reduction}
\label{sec:sidekick:motivating:ack-reduction}

Now consider a battery-powered device downloading a large file from the
Internet. To reduce how often the receiver's
radio needs to wake up, saving energy, the base connection can reduce the
frequency of end-to-end ACKs the device sends.
ACK reduction has also been shown to improve performance by reducing collisions
and contention over half-duplex links~\cite{custura2023reducing,li2020tack}.
The ACK frequency can be configured with a TCP kernel setting or proposed
QUIC extension~\cite{ietf-quic-ack-frequency-07}.

However, ACK reduction can also degrade throughput~\cite{custura2023reducing,custura2020impact}
(\Cref{fig:ack-reduction} in emulation).
The sender receives more delayed feedback about loss, and has to carefully
pace packets to avoid bursts in the large delay between ACKs.
One proposal has the PEP acknowledge packets on behalf of the
receiver~\cite{kliazovich2012arqproxy}, leveraging cleartext TCP sequence
numbers, but it does not apply to secure transport protocols.

In this case, a Sidekick at the Wi-Fi access point (or a cellular base station)
quACKs to the sender on behalf of the receiver. The receiver still occasionally
wakes up its radio to send ACKs, but the sender uses the more frequent quACKs
to advance its flow-control and congestion-control windows.

The sender respects the end-to-end reliability contract by only deleting packets
in response to ACKs, but disregards the receiver's flow control by using quACKs
to advance the flow-control window. If the sender only used ACKs to advance the
window, it would waste time waiting between ACKs to send packets with too small
a window, and need to pace sent packets on receiving a large ACK with too large
a window.

\section{Design}
\label{sec:sidekick:design}

This section describes our design for a Sidekick protocol built around
quACKs. This includes the setup and configuration of a Sidekick
connection, how a data sender detects loss from a quACK, and a path-aware
modification to CUBIC called PACUBIC for congestion-controlled base
protocols.

%%%%%%%%%%%%%%%%%%%%%%%%%%%%%%%%%%%%%%%%%%%%%%%%%%%%%%%%%%%%%%%%%%%%%%%%%%%%%%%

\subsection{PEP discovery mechanism}
\label{sec:sidekick:design:discovery}

% In most settings, such as 4G/5G cellular networks,
PEPs have traditionally been deployed as transparent proxies, silently
interposing on end-to-end connections when they are on the network path.
Endpoints therefore need a way to detect
Sidekick proxies and inform them of where to send quACKs.  Because
of network address translation, all communication to the proxy must be
initiated by the endpoint or use the same IP addresses and port numbers of the
base connection.

Our design has endpoints signal Sidekick support by sending a
distinguished packet containing a 128-byte \texttt{sidekick-request} marker
along the base connection.
Such inline signaling could confuse hosts when only one endpoint supports Sidekick,
but this approach target protocols
such as QUIC that discard cryptographically unauthenticated data anyway.  It
would be cleaner to signal support through out-of-band UDP options~\cite
{ietf-tsvwg-udp-options-28}, which we may hope to do once standardized.
The proxy replies to a \texttt{sidekick-request} packet by sending a special packet from
the other endpoint's IP address and port number back to the original endpoint.
This packet
contains a \texttt{sidekick-reply} marker, an opaque session ID, and an IP
address and port number for communicating with the proxy.

% Sidekick connections can be configured explicitly or implicitly.
In systems that explicitly configure proxies, such as Apple's iCloud Private
Relay~\cite{icloud-private-relay} based on MASQUE~\cite
{kosek2021masque,kramer2021masquepep}, proxies can simply negotiate the Sidekick
connection during session establishment. However, any explicit proxy configuration
has the disadvantage that the proxy may not be on the most direct network path.
Explicit proxies also make mobility and handoffs more challenging.

\subsubsection{Security.}
A malicious third-party could execute a reflection amplification attack that
generates a large amount of traffic while hiding its source. This is
possible because the endpoint requests quACKs to a different port and (for some
carrier-grade NATs) IP address from the underlying session. To mitigate this,
each quACK can contain a quota, initially 1, of remaining quACKs the proxy
will send as well as an updated session ID\@.
The quota and session ID ensure only the endpoint can increase the quota or
otherwise reconfigure the session.

An adversarial PEP could send misleading information to the endpoint. Note that
only on-path PEPs can send credible information, since they refer to unique
packet identifiers.
To mitigate this, the endpoint can consider PEP feedback along with
end-to-end metrics to determine whether to keep using the PEP. The endpoint can
always opt out of the PEP and fall back to end-to-end mechanisms, and the PEP cannot actively manipulate traffic any
more than outside a Sidekick setting.

%%%%%%%%%%%%%%%%%%%%%%%%%%%%%%%%%%%%%%%%%%%%%%%%%%%%%%%%%%%%%%%%%%%%%%%%%%%%%%%

\subsection{Sidekick protocol messages}
\label{sec:sidekick:design:messages}

\begin{figure}[t]
    % \centering
    % Client payloads
    \begin{subfigure}[b]{0.48\linewidth}
        \begin{protopayload}{\texttt{Init}}
            \begin{lstlisting}[language=Rust]
epoch: u32;
base_conn: [u8; 12];
quack_ty: u8;
num_symbols: u8;
id_offset: u16;
quack_pkts: u16;
quack_ms: u16;
            \end{lstlisting}
        \end{protopayload}
        \begin{protopayload}{\texttt{Reset}}
            \begin{lstlisting}[language=Rust]
epoch: u32;
errno: u32;
            \end{lstlisting}
        \end{protopayload}
        \caption{Client payloads.}
        \label{fig:sidekick:payloads:client}
    \end{subfigure}
    \hfill
    % Proxy payloads
    \begin{subfigure}[b]{0.48\linewidth}
        \begin{protopayload}{\texttt{InitACK}}
            \begin{lstlisting}[language=Rust]
epoch: u32;
udp_port: u16;
errno: u32;
            \end{lstlisting}
        \end{protopayload}
        \begin{protopayload}{\texttt{QuACK}}
            \begin{lstlisting}[language=Rust]
count: u32;
last_element: u32;
code: Vec<Symbol>;
            \end{lstlisting}
        \end{protopayload}
        \caption{Proxy payloads.}
        \label{fig:sidekick:payloads:proxy}
    \end{subfigure}
  % Caption
  \caption{Sidekick protocol messages to configure and reset the Sidekick
  connection.}
  \label{fig:sidekick:payloads}
\end{figure}


Once the endpoint has an IP address and port for communicating with the proxy, it
can establish an adjacent Sidekick connection with the proxy from a different
local UDP port. The endpoint can then send various messages to the proxy to
configure the connection and reset bad state (\Cref{fig:sidekick:payloads:client}),
while receiving quACKs to decode and adjust the behavior of the base connection
(\Cref{fig:sidekick:payloads:proxy}).

\subsubsection{Configuration.}
In the \texttt{Init} message, the endpoint configures (i) which quACK construction
to use and the number of symbols, (ii) a byte offset into the packet payload at
which to compute the 4-byte identifier, and (iii) the interval at which the PEP
should send quACKs.
The proxy accepts or rejects these configurations with an \texttt{InitACK}
and, if accepted, immediately begins to send \texttt{QuACK} messages.

The number of symbols represents the upper bound on the number of missing
packets between quACKs, in practice the number of ``holes'' among the packets
that are selectively ACKed. This bound depends on the quACK interval, and
should be set based on how precise loss detection needs to be and other
link qualities. For example, the number of symbols should be larger to
detect congestive loss in the queue of a bottleneck link, or smaller to detect
transmission error on a lossy link.

The quACK interval is expressed in terms of time or number of packets,
 e.g., every $N$ milliseconds or every $N$ packets, as in a TCP delayed ACK.
The endpoint determines the desired interval based on its estimated
RTT of the base connection and its application objectives. For example, the
endpoint may want to receive quACKs more frequently for latency-sensitive
applications or lower-RTT paths.

\subsubsection{Resets.}
The Sidekick protocol allows the endpoint to tell the PEP to reinitialize the
quACK. The \texttt{Reset} message acts as a synchronization point in the base
connection in which both the endpoint and proxy disregard old packets and start
a new cumulative quACK. This is helpful if the quACK becomes invalid, such as
if the number of missing packets exceeds the maximum bound. It is always safe
to reset the quACK, or even to ignore the PEP entirely and fall back to the
base protocol's end-to-end mechanisms.

%%%%%%%%%%%%%%%%%%%%%%%%%%%%%%%%%%%%%%%%%%%%%%%%%%%%%%%%%%%%%%%%%%%%%%%%%%%%%%%
%%%%%%%%%%%%%%%%%%%%%%%%%%%%%%%%%%%%%%%%%%%%%%%%%%%%%%%%%%%%%%%%%%%%%%%%%%%%%%%
%%%%%%%%%%%%%%%%%%%%%%%%%%%%%%%%%%%%%%%%%%%%%%%%%%%%%%%%%%%%%%%%%%%%%%%%%%%%%%%

\subsection{Path-aware sender behavior enabled by the quACK}
\label{sec:sidekick:design:sender}

Now we discuss sender-side behaviors that are enabled by the
Sidekick protocol and which are helpful across several scenarios: detecting
packet loss from a decoded quACK, earlier retransmissions, and path-aware
congestion control to emulate the congestion response of a split TCP connection.

%%%%%%%%%%%%%%%%%%%%%%%%%%%%%%%%%%%%%%%%%%%%%%%%%%%%%%%%%%%%%%%%%%%%%%%%%%%%%%%

\subsubsection{Loss detection.}

The endpoint knows definitively which packets have been received by the proxy from
a decoded power sum quACK. Next, it must determine from the remaining packets which ones
have been dropped and which are still in-flight, including if there has been a
reordering of packets. In-flight packets are later classified as received or
dropped based on future quACKs.

When there is no reordering, the packets that are dropped are just the ``holes''
among the packets that are selectively ACKed by the quACK. In particular, these
are the holes when considering sent packets in the order they were sent up to
the last element received, which represents the last selective ACK. To identify
these dropped packets, the endpoint encodes $t$ cumulative power sums of its sent
packets up to the last element received. The difference between these power
sums and the power sums in the quACK represents the dropped packets. The endpoint
``removes'' the identifiers of dropped packets from its cumulative power sums,
ensuring that the only packets that contribute to the threshold limit are those
that went missing since decoding the last quACK.

To account for reordering in loss detection, our design has the Sidekick
protocol use an
algorithm similar to the 3-duplicate ACK rule in
TCP~\cite{rfc5681tcp,rfc2001tcp}. In TCP, if three or more duplicate ACKs are
received in a row, it is a strong indication that a segment has been lost. Our
design similarly considers a packet lost only if three or more packets sent
after the missing packet have been received. Other mechanisms could involve
timeouts for individual packets similar to the RACK-TLP loss detection
algorithm for TCP~\cite{rfc8985}.

\subsubsection{Earlier retransmissions.}

On detecting loss, a data sender can immediately retransmit missing packets.
Assuming reliability is a desired property of the base protocol, this allows
the data sender to retransmit the packet earlier than if it had waited for an
end-to-end ACK.

%%%%%%%%%%%%%%%%%%%%%%%%%%%%%%%%%%%%%%%%%%%%%%%%%%%%%%%%%%%%%%%%%%%%%%%%%%%%%%%

\subsubsection{Path-aware congestion control.}

The congestion response of a congestion-controlled base protocol to lost packets
that they retransmit due to quACKs determines whether the connection achieves
any throughput improvements. Consider what happens if the base protocol treats
loss detected from quACKs the same as from end-to-end ACKs. Since loss is a
binary signal in ``loss-based'' congestion control algorithms such as
CUBIC~\cite{ha2008cubic}, the connection would have the same, low throughput as
an end-to-end connection. If the base protocol had no congestion response at
all, it would not be fair to connections without Sidekick assistance, and in
the worst case, retransmissions from quACKs could cause congestion collapse.

Thus congestion-controlled base protocols must have some congestion response to
loss from quACKs to ensure friendliness with existing congestion control
schemes that do consider the loss. Our benchmark for a fair response is to
achieve the same throughput as TCP CUBIC in the presence of a
connection-splitting TCP PEP in the same settings. We refer to CUBIC with a
split connection as ``split CUBIC'', which has distinct behavior from
``end-to-end CUBIC'' without a connection-splitter and has notably higher
throughput. However, a path-aware congestion response can only change the
behavior at the endpoint, since packets are opaque to the proxy, in order to
emulate the split congestion-control behavior of split CUBIC. We describe
the algorithmic details of in \Cref{sec:sidekick:pacubic}.

\section{Path-aware CUBIC congestion control}
\label{sec:sidekick:pacubic}

Here, we propose PACUBIC, or path-aware CUBIC, an algorithm that emulates
``split CUBIC'' behavior. PACUBIC uses knowledge of where loss occurs along a
network path to improve connection
throughput compared to end-to-end CUBIC, while remaining fair to competing flows.

\subsection{Background: CUBIC}

Recall that CUBIC~\cite{ha2008cubic} reduces its congestion window by a
multiplicative decrease factor,
$\beta = \beta^* = 0.7$, when observing loss (a congestion event), and otherwise
increases its window based on a real-time dependent cubic function with scaling
factor $C=C^*=0.4$:
\[
cwnd = C(T-K)^3 + w_{max} \text{ where } K = \sqrt[3]{\frac{w_{max}(1-\beta)}{C}}.
\]

\noindent Here, $cwnd$ is the current congestion window,
$w_{max}$ is the window size just before the last reduction,
and $T$ is the time elapsed since the last window reduction.

\subsection{PACUBIC algorithm}

While a split CUBIC connection has \emph{two} congestion windows,
end-to-end PACUBIC only has \emph{one} window representing the in-flight bytes
of the end-to-end connection.
Conceptually, we want an algorithm that enables PACUBIC's single
congestion window to match the sum of the split connection's two congestion
windows.

PACUBIC effectively makes it so that we reduce and grow $cwnd$
proportionally to the number of in-flight bytes on the path segment
of where the last congestion event occurred.
Let $r$ be the estimated ratio of the RTT of the near path segment
(between the data sender and the proxy) to the RTT of the entire connection
(between end hosts).
We use $r$ as a proxy for the ratio of the number of in-flight bytes. If the
last congestion event came from a quACK, we use the same real-time dependent
cubic function but with the following constants:
\[
\beta = 1 - r(1-\beta^*)\text{ and }C = \frac{C^*}{r^3}.
\]
\noindent If the last congestion event came from an end-to-end ACK, then we use
the original $\beta$ and $C$.

While this algorithm resembles the congestion behavior of split CUBIC, it is
simply an approximation. PACUBIC does not know the exact number of bytes
in-flight on each path segment, and the sum of the two congestion windows is
simply a heuristic for an inherently different split connection. The main
takeaway is that knowing where loss occurs can inform congestion control. We
generally hope that quACKs can lead to the development of smarter, path-aware
congestion-control schemes.

\subsection{Intuitive correctness analysis}
\label{sec:sidekick:pacubic:analysis}

\begin{figure}[t]
\centering
\includegraphics[width=0.8\linewidth]{sidekick/figures/cwnd_legend.pdf}\\
\begin{subfigure}{0.32\linewidth}
  \includegraphics[width=\linewidth]{sidekick/figures/cwnd_split_loss0p.pdf}
  \caption{Split CUBIC, 0\% loss.}
  \label{fig:sidekick:pacubic:split-loss0p}
\end{subfigure}
\begin{subfigure}{0.32\linewidth}
  \includegraphics[width=\linewidth]{sidekick/figures/cwnd_pacubic_loss0p.pdf}
  \caption{PACUBIC, 0\% loss.}
  \label{fig:sidekick:pacubic:pacubic-loss0p}
\end{subfigure}
\begin{subfigure}{0.32\linewidth}
  \includegraphics[width=\linewidth]{sidekick/figures/cwnd_cubic_loss0p.pdf}
  \caption{CUBIC, 0\% loss.}
  \label{fig:sidekick:pacubic:cubic-loss0p}
\end{subfigure}
\begin{subfigure}{0.32\linewidth}
  \includegraphics[width=\linewidth]{sidekick/figures/cwnd_split_loss1p.pdf}
  \caption{Split CUBIC, 1\% loss.}
  \label{fig:sidekick:pacubic:split-loss1p}
\end{subfigure}
\begin{subfigure}{0.32\linewidth}
  \includegraphics[width=\linewidth]{sidekick/figures/cwnd_pacubic_loss1p.pdf}
  \caption{PACUBIC, 1\% loss.}
  \label{fig:sidekick:pacubic:pacubic-loss1p}
\end{subfigure}
\begin{subfigure}{0.32\linewidth}
  \includegraphics[width=\linewidth]{sidekick/figures/cwnd_cubic_loss1p.pdf}
  \caption{CUBIC, 1\% loss.}
  \label{fig:sidekick:pacubic:cubic-loss1p}
\end{subfigure}
\caption{Congestion window of a long-running upload in Scenario \#2
(\Cref{tab:sidekick:experimental-scenarios}) with $0\%$ and $1\%$ loss on the
near path segment. The cwnd is measured at the data sender,
except for split CUBIC whose split connection also has a cwnd at the proxy.
PACUBIC reacts to every congestion event while keeping the cwnd high.
CUBIC performs poorly when there is loss on the near path segment.
CUBIC and PACUBIC are implemented in QUIC, while split CUBIC is implemented
in TCP using a PEP.
}
\label{fig:sidekick:pacubic}
\end{figure}


Here, we dive deeper into the intuition behind the PACUBIC constants $\beta$
and $C$, including how they were derived and why
the PACUBIC algorithm achieves similar congestion behavior to split CUBIC.

\subsubsection{Analysis.}

Consider the same network topology as \Cref{fig:sidekick:overview} in which a
data sender uploads a large file to a data receiver, with help from a Sidekick
proxy in the middle of the connection. The near path segment connects the data sender
to the proxy, and the far path segment connects the proxy to the data receiver.
The near path segment is low-delay with varying random loss, and the far path segment is
high-delay with no random loss. The far path segment is the bottleneck link in terms
of bandwidth.
The actual link parameters are the same as in Scenario \#2 of
\Cref{tab:sidekick:experimental-scenarios}.

To conceptually motivate PACUBIC, let's first discuss how split CUBIC would
behave in this setting.
Consider the congestion windows of each half of the split
connection, one taken at the data sender and one at the proxy
(\Cref{fig:sidekick:pacubic:split-loss0p,fig:sidekick:pacubic:split-loss1p}). The far path
segment experiences only congestive loss, leading the window at the proxy to
fluctuate around the segment's bandwidth-delay product regardless of the loss on the near path
segment. The window at the data sender independently determines whether the
packets that reach the proxy will be able to fully utilize the window set at
the far path segment. We will see that the data sender is able to achieve this at low random
loss rates, but becomes the bottleneck as loss rates increase
(\Cref{fig:sidekick:fairness-line}).

While split CUBIC has two windows, PACUBIC only has one
window representing the in-flight bytes of the end-to-end connection.
PACUBIC considers loss detected from both quACKs and end-to-end ACKs.
Conceptually, we want an algorithm that would enable PACUBIC's single
congestion window to match the sum of CUBIC's two congestion windows, or
the total number of in-flight bytes.
% That is the motivation behind adjusting the window proportionally to the
% number of in-flight bytes on each path segment, depending on where the loss
% occurred.

With no random loss on the near path segment, PACUBIC
(\Cref{fig:sidekick:pacubic:pacubic-loss0p}) behaves the same as end-to-end CUBIC
(\Cref{fig:sidekick:pacubic:cubic-loss0p}). The congestion window is entirely governed
by end-to-end ACKs since the far path segment is the bottleneck link. Note that
while the sender may be able to deduce that a loss occurred on the far path
segment by combining info from the quACK with the end-to-end ACK, PACUBIC
conservatively treats the loss as occurring anywhere on the path.

With some random loss on the near path segment, PACUBIC grows and reduces $cwnd$
based on where the last congestion event occurs
(\Cref{fig:sidekick:pacubic:pacubic-loss1p}). Note that if the congestion window $cwnd$
represents the bytes in-flight in the end-to-end connection, then $r \cdot cwnd$
represents the proportion of bytes in-flight on the near path segment. At a
high level, if the data sender discovers loss on the near path segment via the
quACK, it holds the $(1-r)\cdot cwnd$ portion of the ``far window'' constant
while applying the CUBIC algorithm to the remaining $r \cdot w_{max}$ of the
``near window'', representing the bottleneck link.

Mathematically, instead of reducing $w_{max}$, the window size just before the
last reduction, by $(1-\beta^*) \cdot w_{max}$, PACUBIC reduces it by only
$[1 - (1-r(1-\beta^*))] \cdot w_{max} = r(1-\beta^*) \cdot w_{max}$.
That is $r$ times the original reduction, a \emph{smaller} amount.
We use the RTT ratio $r$ (near path segment to end-to-end)
% indicating the RTT ratio of near path segment vs.~far path segment)
as a proxy for the ratio of the number of in-flight bytes.

Similarly, instead of using a cubic growth function with scaling factor $C^*$
and inflection point $K = K^* = \sqrt[3]{w_{max}(1-\beta^*)/C^*}$,
we use a larger scaling factor $C = C^*/r^3$
and thus a shorter inflection point:
\[
K = \sqrt[3]{\frac{w_{max}(1-\beta)}{C}}
= \sqrt[3]{\frac{r\cdot w_{max}(1-\beta^*)}{C^* / r^3}}
= r^{4/3} \cdot K^*.
\]
The shorter inflection point leads the congestion window to \emph{grow more
quickly} since the sender also reacts to feedback about loss more quickly over
the low-delay link.
% Since similarly proportional to $w_{max}$, so with $C'$, we both reduce the
% inflection point $K$ and the growth rate $C$ of the function proportionally to
% the number of bytes in-flight on the near path segment.

At times, there can be loss detected both in quACKs and in end-to-end ACKs.
The end-to-end ACKs have a greater effect since they reduce the congestion
window by a larger proportion, until the remaining path segment with loss is the
bottleneck link. In this scenario with loss, the bottleneck link at equilibrium
is the near path segment.
At this point, the quACK primarily determines the congestion window updates. If
the far path segment were to become the bottleneck again, the data sender would
detect a congestion event via the end-to-end ACK.

\subsubsection{Limitations.}

PACUBIC has several limitations. Although it beats end-to-end CUBIC, we will see
that it still performs worse than split CUBIC, especially at high loss rates
(\Cref{fig:sidekick:fairness-line}). Also, it doesn't consider loss on the far
path segment any differently than end-to-end CUBIC, unlike split CUBIC which
treats the two split connections independently. PACUBIC emulates the congestion
control behavior and fairness of split CUBIC fairly well as a heuristic, but
would benefit from an analysis in a wider variety of network scenarios. It
would also benefit from a side-by-side fairness comparison against congestion
control algorithms such as BBR~\cite{cardwell2024bbr-ietf-draft} that perform
well in the same scenarios. We'd like a takeaway of PACUBIC to be that knowing
where loss occurs can cleverly inform congestion control.

\section{Implementation}
\label{sec:sidekick:implementation}

\begin{table}[ht]
  \centering
  \begin{tabular}{l r r}
    \hline
    \textbf{Module} & \textbf{Language} & \textbf{LOC} \\
    \hline
    Media server/client + integration & Rust & 478 \\
    \texttt{quiche} client integration & Rust & 1821 \\
    \texttt{libcurl} client integration & C & 1459 \\
    Sidekick proxy binary & Rust & 833 \\
    \hline
  \end{tabular}
  \caption{Lines of code in the Sidekick protocol implementation.
  }
  \label{tab:sidekick:lines-of-code}
\end{table}


We now describe our implementation of the Sidekick protocol for several
applications. We integrated Sidekick functionality with a simple media client
for low-latency streaming and an HTTP/3 (QUIC) client. We used the power sum
construction of the quACK from our quACK library in \Cref
{sec:quack:implementation}. The total implementation of the proxy and client
integrations used 4591 LOC (\Cref{tab:sidekick:lines-of-code}).

\subsection{End-to-end applications}
\label{sec:sidekick:implementation:applications}

The baselines we evaluated against were the performance of two secure transport
protocols without proxy assistance, and the fairness of a split CUBIC
connection.

\subsubsection{Low-latency media application.}
We implemented a simple server and client in Rust for streaming low-latency
media. The client sends a numbered packet containing 240 bytes of data every
20 milliseconds, representing an audio stream at 96 kbit/s.
The sequence number is encrypted on the wire.

The server receives packets. If it receives a nonconsecutive sequence number,
it sends a NACK back to the client that contains the sequence number of each
missing packet. The client's behavior on NACK is to retransmit the packet. The
server retransmits NACKs, up to one per RTT, until it has received the packet.

The server's application behavior is to store incoming packets in a buffer
and play them as soon as the next packet in the sequence is available. The
de-jitter buffer delay is the length of time between when the packet is stored
to when it can be played in-order. Some packets can be played immediately.

\subsubsection{HTTP/3 file upload application.}

We used the popular \texttt{libcurl}~\cite{libcurl} file transfer library as the
basis for our HTTP client, and an \texttt{nginx} webserver. The client makes an
HTTP POST request to the server. Both are patched with \texttt{quiche}~\cite
{quiche}, a production implementation of the QUIC protocol from Cloudflare, to
provide support for HTTP/3.

For our TCP baselines, we used the same file upload application with the
default HTTP/1.1 server and client. We used a split-connection
TCP PEP~\cite{caini2006pepsal} that intercepts the TCP
SYN packet in the three-way handshake, pretends to be the other side of that
connection, and initiates a new connection to the real endpoint.
Both clients use CUBIC congestion control.

\subsection{Client integrations with Sidekick}
\label{sec:sidekick:implementation:client-integrations}

In each application, we modified only the \emph{client} to speak the Sidekick
protocol and respond to in-network feedback. The server remained unchanged.
The modifications were in two parts: following the discovery mechanism to
establish bi-directional communication with the proxy, and using the information
in the quACK to modify transport layer behavior.

\begin{table*}[h]
  \centering
  % \renewcommand{\arraystretch}{0.000023}
  \small
  \begin{tabular}{lllll}
    \toprule
    & \bf Data Sender  & \bf Proxy $\leftrightarrow$ Data & \bf QuACK     & \\
    & \bf (Client) $\leftrightarrow$ Proxy & \bf Receiver (Server) & \bf Interval & \bf Threshold \\
    \midrule
    \#1 Low-latency media & $1$ms, $100$ Mbit/s, & $25$ms, $10$ Mbit/s, & $2$ pkts & $8$ \\
                          & $3.6\%$ loss         & $0\%$ loss           &          &     \\

    \#2 Connection-splitting & $1$ms, $100$ Mbit/s, & $25$ms, $10$ Mbit/s, & $30$ ms & $10$ \\
    PEP emulation            & $1.0\%$ loss         & $0\%$ loss           &         & \\

    \#3 ACK reduction & $25$ms, $10$ Mbit/s, & $1$ms, $100$ Mbit/s & $15$ ms & $50$ \\
                      & $0\%$ loss,          & $0\%$ loss          &         & \\
    \bottomrule
  \end{tabular}
  \caption{Experimental scenarios. Link 1 connects the data sender (client) to
  the proxy, while Link 2 connects the proxy to the data receiver (server).
  The quACK interval and threshold represent our Sidekick configuration.
  }
  \label{tab:experimental-scenarios}
\end{table*}


\subsubsection{Low-latency media client.}

The media client has two open UDP sockets: one for the base connection and one
for the Sidekick connection. When it receives a quACK, it detects lost packets
without reordering and immediately retransmits them. The protocol does not have
a congestion window nor a flow-control window. The client also sends reset and
configuration messages over the Sidekick connection.

\subsubsection{HTTP/3 file upload client.}

The HTTP/3 client similarly has an adjacent UDP socket for the Sidekick
connection on which it receives quACKs and sends reset and configuration
messages. The client passes the quACK to our modified \texttt{quiche} library,
which interprets the quACK and makes transport layer decisions. From the
client's perspective, \texttt{quiche} tells \texttt{libcurl} exactly what bytes
to send over the wire.

Our modified \texttt{quiche} library uses the quACK to inform the
retransmission behavior, congestion window, and flow-control window. The library
immediately retransmits lost \emph{frames} in a newly-numbered
packet, as opposed to the lost \emph{packet}, similar to QUIC's original
retransmission mechanism. We implement PACUBIC,
described in \Cref{sec:sidekick:design:sender}.
We also move the flow-control window (without forgetting packets in the
retransmission buffer), but only in the ACK reduction scenario, when the
congestion window is nearly representative of that of the Sidekick connection's
path segment.

\subsection{Sidekick proxy}
\label{sec:sidekick:implementation:proxy}

Our proxy sniffs incoming packets of a network interface using the
\texttt{recvfrom} system call on a raw socket.
It stores a hash table using Rust's standard library \texttt{HashMap} that maps
socket pairs to their respective quACKs, and incrementally updates the quACKs
for flows that have requested Sidekick assistance. It also sends quACKs at
their configured frequencies and listens for configuration messages.

\section{Evaluation methodology}
\label{sec:sidekick:methodology}

\begin{figure}[t]
\centering
\includegraphics[width=0.7\linewidth]{sidekick/figures/setup_real.pdf}
\caption{Real-world experimental setup.
}
\label{fig:sidekick:real-setup}
\end{figure}


We modeled the scenarios from \Cref{sec:sidekick:motivating}. We use the same
m4.xlarge AWS instance as before for the emulated experiments.

\subsubsection{Emulation experiments.}

We emulated a two-hop network topology (\Cref{fig:sidekick:overview}) in
mininet, configuring the link properties using \texttt{tc}.
In emulation, we represented
each link by a constant delay (with variability induced by the queue), a random
loss percentage, and a maximum bandwidth.
\Cref{tab:sidekick:experimental-scenarios} describes the parameters
for each link to model---e.g., lossy Wi-Fi or a high-latency cellular
path---as well as the metrics for success in that scenario.
Link 1 connects the data sender (client) to the proxy,
while Link 2 connects the proxy to the data receiver (server).
On the proxy, we either run a Sidekick,
a connection-splitting TCP PEP~\cite{caini2006pepsal}, or nothing at all.

\subsubsection{Real-world experiments.}

To test its robustness, we also evaluated the Sidekick protocol over a
real-world environment that resembled the scenario on the train
(\Cref{fig:sidekick:real-setup}). In this setup, a Lenovo ThinkPad laptop,
running Ubuntu 22.04.3 with a 4-Core Intel i7 CPU @ 2.60 GHz and 16 GB memory,
acted as a client to an AWS instance in the nearest geographical region. The
ThinkPad used as an access point (AP) a Lenovo Yoga laptop, running Ubuntu
20.04.6 with a 4-Core Intel i5 CPU @ 1.60 GHz and 4 GB memory, with a 2.4 GHz
Wi-Fi hotspot. The AP was connected to the Internet via a JEXtream cellular
modem with a 5G data plan. The AP ran Sidekick software.

We measured the link properties of each path segment to compare to
our emulation parameters. We measured delay and loss using 1000~\texttt{ping}s
over a 100 second period, and bandwidth using an \texttt{iperf3} test.
On the near segment between the ThinkPad client and the AP,
the min/avg/max/stdev RTT was 1.249/37.194/272.168/54.660 ms
at 49.8 Mbit/s bandwidth. We observed that loss increased
the further away the AP. In our experiments, the client was located roughly
200 feet away in a different room, with 3.6\% loss.
The far segment between the AP and the AWS server was
48.546/64.381/92.374/6.806 ms with 0.0\% loss at 30.9 Mbit/s.
In both environments, the cellular link was the bottleneck link in terms of
bandwidth, and the corresponding path segments in emulation had similar
minimum RTTs and average loss percentages.

\section{Real-world results}
\label{sec:sidekick:real-world}

\begin{figure}[t]
\centering
\begin{subfigure}{0.48\linewidth}
	\includegraphics[width=\linewidth]{sidekick/figures/fig8_real_world_webrtc.pdf}
	\caption{Low-latency media. CDF of per-packet de-jitter
	latencies over 10 one-minute trials per protocol.}
	\label{fig:sidekick:real-world:media}
\end{subfigure}
\begin{subfigure}{0.48\linewidth}
	\includegraphics[width=\linewidth]{sidekick/figures/fig8_real_world_retx.pdf}
	\caption{Path-aware congestion control.
	Median of 20 trials. Error bars are 1st and 3rd quartiles.}
	\label{fig:sidekick:real-world:pep-emulation}
\end{subfigure}
\caption{Real-world results. Experiments were run in a moderately well-attended
office environment over a Friday afternoon. Trials alternate between the
baseline and the Sidekick to account for variability in time of day.
}
\label{fig:sidekick:real-world}
\end{figure}


We discuss the results of our experiments replicating two of our scenarios in
the real world, using as context
these main differences between emulation and the real-world:

\begin{itemize}[noitemsep,topsep=0pt]
  \item The RTT is more variable as it depends on interactions in the
  wireless medium and the shared cellular path.
  \item Wireless loss can be more variable as nearby 2.4 GHz devices and
  physical barriers may interfere with the link. Wireless loss also tends
  to be more clustered in practice.
  \item The available bandwidth on the shared cellular path is more variable,
  and depends on the time of day.
\end{itemize}

\Cref{fig:sidekick:real-world} shows the results of running the low-latency
media and connection-splitting PEP emulation experiments in the real-world.
The baseline protocol with a Sidekick is able to
reduce the 99th percentile de-jitter latency of an audio stream
from 2.3~seconds to 204~ms---about a 91\% reduction---and
improve the goodput of a 50 MB HTTP/3 upload by about 50\%.
Although the improvements are more conservative compared to emulation in
\Cref{fig:sidekick:main-results:media} and
\Cref{fig:sidekick:main-results:pep-emulation}, each case still benefits the
base protocol under all circumstances, compared to end-to-end mechanisms alone.

Part of the difference can be attributed to the network setting. When there is
no loss on the near path segment, as can occasionally happen in a real Wi-Fi
link, we do not expect to see a difference with a Sidekick. When there is more
loss on the far path segment, which is variable and depends on the time, we
expect the benefit of the Sidekick to be less since this equally affects the
performance of the base protocol.

The other part of the difference could be made up by future work that better
adapts a Sidekick connection to real-world variability: The client could
improve path segment RTT estimation based on when the proxy receives packets,
and use this dynamic estimate in the calculation of $r$ used in $\beta$ and
$C$. The client could also use this estimate to dynamically adjust the quACK
interval. Finally, we could analyze theoretically how PACUBIC responds to
traffic patterns in the real world.

\section{Emulation results}
\label{sec:sidekick:emulation}

\begin{figure*}
\begin{subfigure}{0.34\textwidth}
\includegraphics[width=\linewidth]{sidekick/figures/fig4a_low_latency_media.pdf}
\caption{Scenario \#1:
 Reduced tail latency of de-jitter delay
with earlier retransmission in the low-latency media application. 5 minute trials.}
\label{fig:sidekick:main-results:media}
\end{subfigure}
\hfill
\begin{subfigure}{0.31\textwidth}
\includegraphics[width=0.97\linewidth]{sidekick/figures/fig4b_pep_emulation.pdf}
\caption{Scenario \#2: Improved goodput in the connection-splitting PEP emulation.
Error bars are the IQR of 20 trials.
}
\label{fig:sidekick:main-results:pep-emulation}
\end{subfigure}
\hfill
\begin{subfigure}{0.32\textwidth}
\includegraphics[width=0.99\linewidth]{sidekick/figures/fig4c_ack_reduction.pdf}
\caption{Scenario \#3:
High goodput independent of end-to-end ACK frequency in the ACK reduction scenario.
10 MB upload.}
\label{fig:sidekick:main-results:ack-reduction}
\end{subfigure}
\caption{
Comparing the end-to-end baseline protocol to the same protocol with a Sidekick
connection, using the success metrics for the three scenarios described in
\Cref{tab:sidekick:experimental-scenarios}.
}
% \dm{Maybe a notation like $x/4$ would be more suggestive than $4x$?}
\label{fig:sidekick:main-results}
\end{figure*}


We evaluated our implementation of the Sidekick protocol in a more controlled
emulation environment to answer the following questions:
\begin{enumerate}[noitemsep,topsep=0pt]
	\item Can Sidekicks improve the performance of secure transport protocols
	in a variety of scenarios while preserving the end-to-end behavior of the
	base protocols?
	\item Can a path-aware congestion control algorithm match the fairness of
	split TCP PEPs using CUBIC?
	\item How do the CPU overheads of encoding quACKs impact the maximum
	capacity of a proxy with a Sidekick?
	\item What link overheads does the power sum quACK add and how does it
	compare to the strawmen?
\end{enumerate}

\subsection{Performance of secure transport protocols with Sidekick}
\label{sec:sidekick:emulation:performance}

We first evaluate Sidekick's main performance goal: In each of the motivating
scenarios, we show that the Sidekick protocol can improve performance compared
to the base protocol alone, which would not be able to benefit from existing
PEPs. Each scenario has a different metric for success---tail latency,
throughput, or number of packets sent by the data receiver (corresponding to
energy usage or chance of Wi-Fi collisions)---demonstrating the versatility of
the Sidekick protocol.

\subsubsection{Low-latency media application.}

The Sidekick can reduce tail latencies in a low-latency media stream,
representing fewer drops and better quality of experience. The early
retransmissions induced by the Sidekick reduced the 99th percentile latency of
the de-jitter buffer delay from 48.6 ms to 2.2 ms---a 95\% reduction
(\Cref{fig:sidekick:main-results:media}). As long as the quACK interval is less
than the end-to-end RTT, the connection benefits from the Sidekick.

The Sidekick is beneficial in this scenario because it enables the client to
sooner detect and retransmit lost packets, and the server to sooner play
packets from its de-jitter buffer. The end-to-end mechanism takes one
additional received packet to notify of the loss and one end-to-end RTT to
retransmit and play the packet (20+52=72ms), resulting in three delayed
packets (the three ``steps" in \Cref{fig:sidekick:main-results:media}) in most
cases. The Sidekick takes up to two additional packets and one near path
segment RTT ($20+2=22$ms or $20\times2+2=42$ms), delaying either one or two
packets in comparison. Dropped ACKs and quACKs account for the $<2\%$ of
packets with even greater de-jitter latencies.

\subsubsection{Connection-splitting PEP emulation.}

The Sidekick improves upload speeds when there is a lossy, low-latency link
by using quACKs to inform the sender's congestion control.
In a scenario with $1\%$ random loss on the link between the proxy and the
data sender, the HTTP/3 (QUIC) client achieves $3.6\times$ the goodput for a
10 MB upload with a Sidekick compared to end-to-end QUIC
(\Cref{fig:sidekick:main-results:pep-emulation}).

When there is no random loss, the Sidekick does not impact the performance of
QUIC. There are no logical changes to the base protocol in this case because
all loss is on the bottleneck link on the far path segment, and the CPU
overheads of processing quACKs are negligible.

Knowing \emph{where} congestion occurs is an opportunity for creating smarter
congestion control. In PACUBIC, identifying where the loss occured let the data
sender reduce the congestion window proportionally to how many packets were
in-flight on each path segment. In \Cref{sec:sidekick:emulation:pacubic}, we
will show that our path-aware congestion control algorithm still matches the
fairness of connection-splitting TCP PEPs.

\subsubsection{ACK reduction scenario.}

Using quACKs in lieu of end-to-end ACKs allows the data receiver to
significantly reduce its ACK frequency while maintaining high goodput.
In our experiment, QUIC with a Sidekick sent $96\%$ fewer packets (mainly ACKs)
than end-to-end QUIC before the goodput dropped below 8.5 Mbit/s
(\Cref{fig:sidekick:main-results:ack-reduction}).
The quACK enables the data sender to promptly move the flow-control window
forward, as long as the last hop is reliable.

The goodput significantly degrades when reducing the end-to-end ACK frequency
without a Sidekick. When end-to-end QUIC reduces the ACK frequency to every
80 ms, the data receiver sends $247 / 138 = 1.8\times$ the packets at
$4.5 / 8.4 = 0.5\times$ the goodput, worse than QUIC with the Sidekick
in both dimensions (\Cref{fig:sidekick:main-results:ack-reduction}).
With a Sidekick, the data sender also does not need to change packet pacing to
avoid bursts in response to infrequent ACKs, which is why end-to-end QUIC
cannot send fewer than $\approx 240$ packets.

\subsubsection{Discussion: Configuring the Sidekick connection.}

\Cref{tab:sidekick:experimental-scenarios} shows the quACK interval and
threshold we elected for each scenario based on the considerations in
\Cref{sec:sidekick:design:messages}. In each experiment in
\Cref{fig:sidekick:main-results}, we also show how with less frequent quACKs
($2\times$ and $4\times$ the interval) and proportionally-adjusted thresholds,
the protocol performs worse, or more variably. Less frequent quACKs means the
client reacts later to feedback about the near path segment, and more often has
to rely on the end-to-end mechanism. The performance particularly degrades when
the quACK interval exceeds the end-to-end RTT. However, even in this case, the
base protocol with any Sidekick at all performs better than the base protocol
alone\@.

\subsection{Link overheads from sending quACKs}
\label{sec:sidekick:emulation:link-overheads}

\begin{figure}[h]
\begin{subfigure}{\columnwidth}
  % 5+
  %
  \setlength{\tabcolsep}{2pt}
  \footnotesize
  \centering
  \begin{tabular}{lccccccc}
    \toprule
    & \multicolumn{2}{c}{Data Sender$\rightarrow$} & \multicolumn{2}{c}{$\leftarrow$Proxy} & \multicolumn{2}{c}{$\leftarrow$Data Receiver} & \\
    & \bf Pkts & \bf Bytes & \bf Pkts & \bf Bytes & \bf Pkts & \bf Bytes & \bf Goodput \\
    \midrule
    QUIC E2E & $1.00\times$ & $1.00\times$ & $1.00\times$ & $1.00\times$ & $1.00\times$ & $1.00\times$ & $1.00\times$ \\
    Strawman 1a & $0.96\times$ & $1.01\times$ & \cellcolor{LighterRed}{$2.02\times$} & \cellcolor{LightestRed}{$1.56\times$} & $1.01\times$ & $1.03\times$ & \cellcolor{LighterGreen}{$3.33\times$} \\
    Strawman 1b & $0.94\times$ & $1.00\times$ & \cellcolor{LighterRed}{$2.00\times$} & \cellcolor{LightestRed}{$1.78\times$} & $1.00\times$ & $1.03\times$ & \cellcolor{LightGreen}{$3.53\times$} \\
    Strawman 1c & \cellcolor{LightestRed}{$1.83\times$} & $1.06\times$ & \cellcolor{LighterRed}{$2.01\times$} & \cellcolor{LightestRed}{$1.83\times$} & $1.00\times$ & $1.03\times$ & \cellcolor{LightGreen}{$3.46\times$} \\
    \bf \textcolor{black!50!blue}{Power Sum}   & \textcolor{black!50!blue}{\bf 0.94$\times$} & \textcolor{black!50!blue}{\bf 1.00$\times$} & \textcolor{black!50!blue}{\bf 1.03$\times$} & \textcolor{black!50!blue}{\bf 1.07$\times$} & \textcolor{black!50!blue}{\bf 1.00$\times$} & \textcolor{black!50!blue}{\bf 1.03$\times$} & \cellcolor{LightGreen}{\textcolor{black!50!blue}{\bf 3.55$\times$}} \\
    \bottomrule
  \end{tabular}
  % \includegraphics[width=\columnwidth]{figures/packet-overhead-retx.png}
  \caption{Scenario \#2: Connection-splitting PEP emulation.}
  \label{tab:packet-overhead:retx}
\end{subfigure}
\begin{subfigure}{\columnwidth}
  % \includegraphics[width=\columnwidth]{figures/packet-overhead-ackr.png}
  \setlength{\tabcolsep}{2pt}
  \footnotesize
  \centering
  \begin{tabular}{lccccccc}
    \toprule
    & \multicolumn{2}{c}{Data Sender$\rightarrow$} & \multicolumn{2}{c}{$\leftarrow$Proxy} & \multicolumn{2}{c}{$\leftarrow$Data Receiver} & \\
    & \bf Pkts & \bf Bytes & \bf Pkts & \bf Bytes & \bf Pkts & \bf Bytes & \bf Goodput \\
    \midrule
    QUIC E2E & $1.00\times$ & $1.00\times$ & $1.00\times$ & $1.00\times$ & $1.00\times$ & $1.00\times$ & $1.00\times$ \\
    Strawman 1a & $0.96\times$ & $1.00\times$ & \cellcolor{LightRed}{$9.94\times$} & \cellcolor{LighterRed}{$4.99\times$} & \cellcolor{LightGreen}{$0.04\times$} & \cellcolor{LightGreen}{$0.08\times$} & $1.02\times$ \\
    Strawman 1b & $0.96\times$ & $1.00\times$ & \cellcolor{LightRed}{$9.95\times$} & \cellcolor{LightRed}{$7.13\times$}      & \cellcolor{LightGreen}{$0.04\times$} & \cellcolor{LightGreen}{$0.08\times$} & $1.02\times$ \\
    Strawman 1c & \cellcolor{LightestRed}{$1.91\times$} & $1.05\times$ & \cellcolor{LightRed}{$9.73\times$} & \cellcolor{LightRed}{$7.41\times$}      & \cellcolor{LightGreen}{$0.04\times$} & \cellcolor{LightGreen}{$0.08\times$} & $0.97\times$ \\
    \bf \textcolor{black!50!blue}{Power Sum}    & \textcolor{black!50!blue}{\bf 0.96$\times$} & \textcolor{black!50!blue}{\bf 1.00$\times$} & \textcolor{black!50!blue}{\bf 1.09$\times$} & \cellcolor{LighterRed}{\textcolor{black!50!blue}{\bf 2.56$\times$}} & \cellcolor{LightGreen}{\textcolor{black!50!blue}{\bf 0.04$\times$}} & \cellcolor{LightGreen}{\textcolor{black!50!blue}{\bf 0.08$\times$}} & \textcolor{black!50!blue}{\bf 0.98$\times$} \\
    \bottomrule
  \end{tabular}
  \caption{Scenario \#3: ACK reduction.}
  \label{tab:packet-overhead:ackr}
\end{subfigure}
\caption{Link overheads for a 10 MB upload. The cells represent the multiplier
relative to the end-to-end QUIC baseline for each type of quACK\@.
Lower is better for number of packets and bytes sent on a link.
Higher goodput is better. Robin's power sum quACK achieves the success metric
for each scenario without incurring the link overheads of the strawmen.
We did not evaluate the contrived protocol in Scenario \#1.
}
\label{tab:packet-overhead}
\end{figure}


The other cost in terms of using Sidekick protocols is the additional data sent
by the proxy to the data sender. Too many additional bytes use up bandwidth,
and additional packets use up CPU\@. \Cref{tab:sidekick:packet-overheads} shows
the number of packets and bytes sent at each node comparing the strawmen and
power sum quACK to no Sidekick connection at all.

Using power sum quACKs increases the packets sent from the proxy to the data
sender
by 3-9\%. These packets either consist mostly
of end-to-end ACKs which are sent every packet in \texttt{quiche}, or end-to-end
ACKs that have been replaced by quACKs in the ACK reduction scenario.
We did not evaluate Scenario \#1 because it is based
on a contrived protocol that lacks many of these features, and the link
overheads would not really make sense.

This overhead is representative of the CPU overhead at the client, since
quACKs and ACKs take a similar number of cycles to process. In an experiment
with Scenario \#2 during a period of $\approx90$k incoming packets, ACKs took on
average 26065 cycles to process while the quACKs took 26369 cycles, 1\% more.
These cycles come from, i.e., the complex recovery and loss detection algorithms
implemented at the end host.

The strawmen have significantly higher link overheads compared to the power sum
quACK\@. The proxy sends up to 10$\times$ more packets using Strawman 1a, and
also slightly harms the goodput in the congestion control scenario.
The reduced goodput is due to the sender mis-identifying received packets as
dropped due to dropped quACKs.
The proxy achieves higher goodput with Strawman 1b but sends
more bytes. Strawman 1c increases the link overheads at both the proxy and the
data sender due to larger TCP headers and TCP ACKs.
We did not evaluate Strawman 2 due to its impractical decode time.

\subsection{CPU overheads of encoding at the proxy}
\label{sec:sidekick:emulation:cpu-overheads}

\begin{table}[h]
  \centering
  \small
  \begin{tabular}{lrrrr}
    \toprule
    & \multicolumn{2}{r}{\bf 25-Byte Payload} & \multicolumn{2}{r}{\bf 1468-Byte Payload}\\
    & \bf Cycles & \bf $\%$ & \bf Cycles & \bf $\%$ \\
    \midrule
    Sniff Packet & 22417 & 97.6 & 22408 & 97.5 \\
    Table Lookup &   247 &  1.1 &   251 &  1.1 \\
    Parse ID     &    23 &  0.1 &    22 &  0.1 \\
    Encode ID    &    74 &  0.3 &    69 &  0.3 \\
    Other        &   213 &  0.9 &   225 &  1.0 \\
    \midrule
    \emph{Total} & \emph{22974} & \emph{100.0} & \emph{22975} & \emph{100.0} \\
    \bottomrule
  \end{tabular}
  \caption{Breakdown of the CPU cycles spent processing each packet at the
  proxy. Most cycles are spent on general per-packet overheads as opposed to
  quACK-specific processing.
  }
  \label{tab:cpu-overhead}
\end{table}


The main bottleneck of Sidekick on a proxy is the CPU\@.
\Cref{tab:sidekick:cpu-overheads} shows a breakdown of the number of CPU cycles
in each step. The largest overhead was reading the packet contents from the
network interface ($97.5\%$ of the CPU cycles).

Encoding an identifier in a power sum quACK with $t=10$ used $74$ CPU
cycles ($0.9\%$). As a calculation of the theoretical maximum on a 2.30 GHz
% 2.30e9 / 74 = 31 million
CPU, the proxy would be able to process $31$ million packets/second on a single
core. The hash table lookup used $251$ cycles and parsing the pseudorandom
payload as an identifier used $22$ cycles.

In practice, we measured the maximum throughput of our Sidekick proxy to
be 464k packets/s with 25-byte payloads and 5.5 Gbit/s (458k packets/s) with
1468-byte packet payloads on a single core (assuming 1500-byte MTUs).
This experiment used multiple \texttt{iperf3} clients to simulate high
load until the proxy was unable to keep up with the load on a single core.
The packet payload size did not seem to affect results.

We find these achieved throughputs acceptable for edge routers such as Wi-Fi APs
and base stations. To deploy the Sidekick proxy on core routers, we would need
to reduce the overhead of reading packets from the NIC, such as by bypassing
the kernel/user-space protection boundary\footnote{ A kernel-bypass system like
Retina~\cite{wan2022retina} can achieve 25 Gbps on 2 cores while processing raw
packets with a 1000-cycle callback(Figure 5(a) in \cite{wan2022retina}). The
Sidekick equivalent would be a 500-cycle callback, and assuming all traffic has
requested Sidekick help. Throughput scales almost linearly with the number of
cores using symmetric RSS hashing. Thus we don't expect proxy overheads to be
an issue with modern 100 Gbps network speeds and an optimized implementation
even on commodity hardware. }~\cite{dpdk,mccanne1993bsd,wan2022retina} or using
native hardware~\cite{bosshart2014p4}. We could also scale on multiple cores
using symmetric RSS hashing~\cite{woo2012scalable}.

\subsection{TCP friendliness of path-aware CUBIC}
\label{sec:sidekick:emulation:pacubic}

\begin{figure}[t]
\centering
\includegraphics[width=\columnwidth]{sidekick/figures/fig5_baseline_bar_legend.pdf}
\begin{subfigure}{0.49\linewidth}
	\includegraphics[width=\linewidth]{sidekick/figures/fig5_baseline_loss0p.pdf}
	\caption{0\% loss.}
	\label{fig:sidekick:fairness-bar:loss0p}
\end{subfigure}
\begin{subfigure}{0.49\linewidth}
	\includegraphics[width=\linewidth]{sidekick/figures/fig5_baseline_loss1p.pdf}
	\caption{1\% loss.}
	\label{fig:sidekick:fairness-bar:loss1p}
\end{subfigure}
\caption{Median goodput for three upload data sizes with $0\%$ and $1\%$ loss on
Link 1. 20 trials. Error bars are 1st and 3rd quartiles.
With proxy assistance at $1\%$
loss, both QUIC and TCP match the performance of when there is no loss at all.
}
\label{fig:sidekick:fairness-bar}
\end{figure}

\begin{figure}[t]
\centering
\includegraphics[width=0.8\columnwidth]{sidekick/figures/fig6_legend.pdf}
\includegraphics[width=0.8\columnwidth]{sidekick/figures/fig6_loss_bw100_10M_delay_25ms_1ms.pdf}
\caption{Connection-splitting PEP emulation as a function of near-segment
	loss rate. In this emulation experiment, QUIC+Sidekick (running PACUBIC)
  performs similarly to TCP+PEP (each connection running CUBIC)
  and improves goodput compared with end-to-end protocols. The graph shows
  median goodput of a 10~MByte upload. QuACK interval is 30~ms, threshold
is 10. Error bars show IQR of 10 trials.
}
\label{fig:sidekick:fairness-line}
\end{figure}


It is easy to improve performance without regard to competing flows;
however, we demonstrate that PACUBIC can
match the fairness of split CUBIC in a TCP PEP connection\@.
We evaluate fairness using Scenario \#2 with varying amounts of loss on the
near path segment.

\subsubsection{QUIC vs.\ TCP\@.}
We first compare QUIC to TCP without either PEP\@.
As both connections use CUBIC, they exhibit similar
congestion control behavior and achieve nearly maximum throughput in the
emulated network with no random loss (\Cref{fig:sidekick:fairness-bar:loss0p}).
We attribute the differences to the slightly different retranmission and
loss recovery behaviors of QUIC and TCP\@. The PEPs do not affect the
performance.

With even a little loss on the near path segment, both QUIC and TCP dramatically
worsen, respectively achieving $28\%$ and $42\%$ of the goodput at $0\%$ loss,
for a 10 MB upload (\Cref{fig:sidekick:fairness-bar:loss1p}).
% 0.305 / 1.098 = 27.8%
% 0.467 / 1.121 = 41.7%
In both protocols, CUBIC treats every transmission error as a congestion event,
even though no amount of reducing the congestion window affects the error rate.
QUIC and TCP perform similarly to each other with proxy assistance and 1\%
loss on the near path segment.

\subsubsection{Sidekick vs.\ TCP PEP\@.}

\Cref{fig:sidekick:fairness-line} shows that QUIC with a Sidekick roughly
matches---as intended---the behavior of TCP with a PEP-assisted split
connection. At higher loss rates, the near path segment becomes the bottleneck
link even with earlier feedback about loss, causing the performance of TCP with
proxy assistance to drop. QUIC with a Sidekick follows a similar pattern
because of its path-aware congestion-control scheme
(\Cref{sec:sidekick:design:sender}). The results indicate that the Sidekick
protocol's gains do not come at the expense of congestion-control fairness
relative to the split TCP connection.

\section{Summary}
\label{sec:sidekick:summary}

