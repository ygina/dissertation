\section{Limitations}
\label{sec:sidekick:limitations}

The Sidekick approach, and our experiments, are subject to some limitations,
which we describe briefly here.

\subsubsection{Multipath scenarios.}
We have only considered Sidekick proxies along a single path, and not thought
extensively about how quACKs would interact with protocols such as
\mbox{TCPLS}~\cite{rochet2020tcpls} that use multiple paths or streams,
or even multipath QUIC~\cite{de2017multipath}.
To begin thinking about this question, we would have a more complex model of the
network: multiple PEPs along a single path, multiple paths each with varying
numbers of PEPs, and so on. The proxy can include additional information in the
Sidekick-reply packet to indicate which path the PEP assistance is on, and the
sender can infer from the RTT how far along a path each PEP is relative to
others. New Sidekick algorithms that come from this model could diagnose
troublesome paths, or better allocate network traffic in a multipath
connection. Existing algorithms could be applied to individual paths as if they
were single-path connections.

\subsubsection{Even more diverse network scenarios.}
The three scenarios we explored all consisted of a lossy Wi-Fi link and a
high-latency WAN link. Not all scenarios will be favorable to the
Sidekick protocol we designed.
If the ``lossy'' section of a network path were on the far path segment from the
sender, the sender would not have any more information about the problematic
link. To accomodate scenarios like this, Sidekick protocols will need
more features. For example, the \emph{proxy} would need some way to receive
quACKs from the data \emph{receiver}, as well as a mechanism to buffer and
retransmit packets~\cite{balakrishnan1995snoop,caini2006pepsal}.

There are likely other scenarios that could benefit from Sidekick protocols as
described, but we did not evaluate them. For example, if we replaced the lossy
Wi-Fi link with a modern wireless link that has a fluctuating physical
capacity~\cite{niu2015survey,burchardt2014vlc,koenig2013wireless}, the sender
may be able to more quickly adapt and make data available for transmission
whenever capacity intermittently becomes available.

\subsubsection{Practical deployment.}

The implementation of the Sidekick protocol exists as a research system that has
been evaluated in emulation and a limited set of real-world scenarios. Since
Sidekick protocols require the cooperation of middleboxes and client
applications, more work will be needed to standardize the discovery protocol
and wire format of Sidekick messages described in
\Cref{sec:sidekick:design:messages}, ideally with interest from the IETF. The
standards will need to establish several design choices such as how identifiers
are computed, how quACKs are transmitted, and the exact mechanisms for security
and backwards compatibility. We may also want to standardize sender behavior
for specific base protocols, though this could be opaque except to the sender.

The deployment of Sidekick protocols can be gradual and backwards-compatible
with parties that are either unaware of or do not want to participate in
Sidekick protocols.
To migrate existing client applications, one needs to modify the code to
discover a PEP and use information in a quACK to inform the base protocol.
To migrate middleboxes, they would need to be modified to listen for
Sidekick-request markers, then accumulate and send quACKs for participating
connections.

\subsubsection{Deeper analysis of path-aware congestion control.}

The correspondence between endpoint-driven PACUBIC and ``split CUBIC'' is good,
and both are better than end-to-end CUBIC in \Cref{fig:sidekick:fairness-line}),
but not exact. The appropriateness of the PACUBIC heuristic, and in general the
idea of path-aware congestion control, needs to be further explored. We discuss
this more in \Cref{sec:sidekick:appendix}.

\section{Summary}
\label{sec:sidekick:summary}

We presented \textit{Sidekick protocols}: an alternate approach to PEPs that
leaves the underlying protocol opaque and unmodified on the wire. We leveraged
a mathematical technique called a \textit{quACK} that enables proxies to refer
to packets of the underlying connection without the ability to observe
cleartext sequence numbers. We augmented a streaming protocol and a production
QUIC implementation (Cloudflare quiche) to make use of information arriving
from a proxy on a Sidekick connection, including a path-aware congestion control
mechanism called \textit{PACUBIC}. In emulation and a real-world evaluation,
the Sidekick protocol was able to improve the performance---tail latency,
throughput, or energy usage---of these end-to-end base protocols without
modifying the wire format or security properties.
