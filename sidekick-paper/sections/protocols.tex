\section{Motivating Scenarios}
\label{sec:motivation}

% We show that with just this information,

We focus on three scenarios where end hosts benefit from in-network assistance. In each one, a proxy server provides
feedback, called a quACK, to an end host: the data sender (\Cref{fig:sc-protocols}).
Recall that a quACK is a ``cumulative ACK + selective ACK'' over encrypted sequence numbers.
The data sender uses this feedback to influence its behavior on the base connection, without altering the wire format.

To be clear: the \sys protocol is not tied to a specific base protocol
nor to how the end hosts use the quACK information. The base protocol does not
need to be reliable, nor to have unique datagrams---we implemented and evaluated
the same \sys protocol and the same middlebox behavior across the different
scenarios in this paper.

\subsection{Low-Latency Media}
\label{sec:low-latency-media}

Consider a train passenger using on-board Wi-Fi to have a low-latency audio
conversation, using WebRTC/SRTP~\cite{rfc8834webrtc}, with a friend.
The end-to-end network path contains a low-latency, high-loss ``near'' path
segment (the Wi-Fi hop) followed by a high-latency, low-loss ``far'' path
segment (the cellular and wired path over the Internet).
The friend probably suffers from poor connection quality, experiencing drops in the
audio stream or high de-jitter buffer delays from waiting for retransmitted
packets to be played in order (\Cref{fig:media} in emulation, \Cref{fig:real-world:scenario1} in real world).

In the \sys approach, a \sys on the Wi-Fi access point sends quACKs to the audio
application on the user's laptop, assisting the base connection's data sender.
The sender uses quACKs
to retransmit packets sooner than they would have using negative acknowledgments (NACKs) from the receiver.
The end result is similar to the effect of prior PEPs, such as Snoop~\cite{balakrishnan1995snoop}
and Milliproxy~\cite{polese2017milliproxy}, that leverage TCP's cleartext sequence numbers
to trigger early retransmission on lossy wireless paths.

\subsection{Connection-Splitting PEP Emulation}

Consider the same train passenger as before but uploading a large file over the
Internet with a reliable transport protocol. If the protocol were
TCP, the train could deploy a split TCP PEP at the access point.
The split connection allows quick detection and retransmission of dropped
packets on the lossy Wi-Fi segment, while opening
up the congestion window on the high-latency cellular segment.

However, opaque transport protocols like QUIC can't benefit from (nor be harmed
by) connection-splitting PEPs. Without a PEP, QUIC relies on end-to-end
mechanisms over the entire path to detect losses, recover from them, and adjust
the congestion-control behavior. This leads to reduced upload speeds
(\Cref{fig:baseline-line} in emulation, \Cref{fig:real-world:scenario2} in real world).

With help from the same \sys PEP, the QUIC sender combines
information from quACKs and end-to-end ACKs to emulate the congestion-control
behavior of a split TCP connection (\Cref{sec:design:cubic}). The application
considers whether packets are lost on the near or far path segments, and
adjusts the congestion window accordingly while respecting the opacity of the
end-to-end base connection. The application also retransmits the packet as soon
as the loss has been detected.

The only guarantee the proxy makes to the sender via the quACK is that it
has received some packets.
% There is no contract between the sender and the proxy other than the
% proxy communicating that it has received some packets through a quACK.
To respect the end-to-end reliability contract with the receiver, the sender
does not delete packets that may need to be transmitted until it receives an
ACK, even if the packet has been quACKed.

\subsection{ACK Reduction}

Now consider a battery-powered device downloading a large file from the
Internet. To reduce how often the receiver's
radio needs to wake up, saving energy, the base connection can reduce the
frequency of end-to-end ACKs the device sends.
ACK reduction has also been shown to improve performance by reducing collisions
and contention over half-duplex links~\cite{custura2023reducing,li2020tack}.
The ACK frequency can be configured with a TCP kernel setting or proposed
QUIC extension~\cite{ietf-quic-ack-frequency-07}.

However, ACK reduction can also degrade throughput~\cite{custura2023reducing,custura2020impact}
(\Cref{fig:ack-reduction} in emulation).
The sender receives more delayed feedback about loss, and has to carefully
pace packets to avoid bursts in the large delay between ACKs.
One proposal has the PEP acknowledge packets on behalf of the
receiver~\cite{kliazovich2012arqproxy}, leveraging cleartext TCP sequence
numbers, but it does not apply to opaque transport protocols.

In this case, a \sys at the Wi-Fi access point (or a cellular base station)
quACKs to the sender on behalf of the receiver.
The receiver still occasionally wakes up its radio to send ACKs,
but the sender uses the more frequent quACKs to advance its flow-control and
congestion-control windows.

The sender respects the end-to-end reliability contract by only deleting
packets in response to ACKs, but disregards the receiver's flow control
by using quACKs to advance the flow-control window.
If the sender only used ACKs to advance the window,
it would waste time waiting between ACKs to send packets with too small a
window, and need to pace sent packets on receiving a large ACK with too large a
window.
