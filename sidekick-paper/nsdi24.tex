%%%%%%%%%%%%%%%%%%%%%%%%%%%%%%%%%%%%%%%%%%%%%%%%%%%%%%%%%%%%%%%%%%%%%%%%%%%%%%%%
% Template for USENIX papers.
%
% History:
%
% - TEMPLATE for Usenix papers, specifically to meet requirements of
%   USENIX '05. originally a template for producing IEEE-format
%   articles using LaTeX. written by Matthew Ward, CS Department,
%   Worcester Polytechnic Institute. adapted by David Beazley for his
%   excellent SWIG paper in Proceedings, Tcl 96. turned into a
%   smartass generic template by De Clarke, with thanks to both the
%   above pioneers. Use at your own risk. Complaints to /dev/null.
%   Make it two column with no page numbering, default is 10 point.
%
% - Munged by Fred Douglis <douglis@research.att.com> 10/97 to
%   separate the .sty file from the LaTeX source template, so that
%   people can more easily include the .sty file into an existing
%   document. Also changed to more closely follow the style guidelines
%   as represented by the Word sample file.
%
% - Note that since 2010, USENIX does not require endnotes. If you
%   want foot of page notes, don't include the endnotes package in the
%   usepackage command, below.
% - This version uses the latex2e styles, not the very ancient 2.09
%   stuff.
%
% - Updated July 2018: Text block size changed from 6.5" to 7"
%
% - Updated Dec 2018 for ATC'19:
%
%   * Revised text to pass HotCRP's auto-formatting check, with
%     hotcrp.settings.submission_form.body_font_size=10pt, and
%     hotcrp.settings.submission_form.line_height=12pt
%
%   * Switched from \endnote-s to \footnote-s to match Usenix's policy.
%
%   * \section* => \begin{abstract} ... \end{abstract}
%
%   * Make template self-contained in terms of bibtex entires, to allow
%     this file to be compiled. (And changing refs style to 'plain'.)
%
%   * Make template self-contained in terms of figures, to
%     allow this file to be compiled.
%
%   * Added packages for hyperref, embedding fonts, and improving
%     appearance.
%
%   * Removed outdated text.
%
%%%%%%%%%%%%%%%%%%%%%%%%%%%%%%%%%%%%%%%%%%%%%%%%%%%%%%%%%%%%%%%%%%%%%%%%%%%%%%%%

\synctex=1
\documentclass[letterpaper,twocolumn,10pt,table]{article}
\usepackage{usenix-2020-09}

% to be able to draw some self-contained figs
\usepackage{tikz}
\usepackage{amsmath}

% custom imports
\usepackage[english]{babel}
\usepackage{blindtext}
\usepackage{hyperref}
\usepackage{enumitem}
\usepackage{cleveref}
\usepackage{booktabs}
\usepackage{listings}
\usepackage{booktabs}
\usepackage{subcaption}
\usepackage{float}
\usepackage{xcolor}
\usepackage[compact]{titlesec}
% \usepackage[tight]{subfigure}

%-------------------------------------------------------------------------------
\begin{document}
%-------------------------------------------------------------------------------

\newcommand{\sys}{sidekick\xspace}
\newcommand{\Sys}{Sidekick\xspace}
\newcommand{\michael}[1]{\textcolor{red}{{\sf (MW: #1)}}}
\newcommand{\keith}[1]{\textcolor{red}{{\sf (KW: #1)}}}
\newcommand{\masot}[1]{\textcolor{red}{{\sf (MS: #1)}}}
\newcommand{\gina}[1]{\textcolor{red}{\sf {(GY: #1)}}}
\newcommand{\dm}[1]{\textcolor{red}{\sf (DM: #1)}}
% \newcommand{\michael}[1]{\textcolor{red}{}}
% \newcommand{\masot}[1]{\textcolor{red}{}}
% \newcommand{\gina}[1]{\textcolor{purple}{}}
% \newcommand{\todo}[1]{\textcolor{red}{}}

%don't want date printed
\date{}

% make title bold and 14 pt font (Latex default is non-bold, 16 pt)
\title{\Sys: In-Network Assistance for Secure End-to-End Transport Protocols}

%for single author (just remove % characters)
\author{
{\rm Gina Yuan}\\
Stanford University
\and
{\rm Matthew Sotoudeh}\\
Stanford University
\and
{\rm David K. Zhang}\\
Stanford University
\and
{\rm Michael Welzl}\\
University of Oslo
\and
{\rm David Mazières}\\
Stanford University
\and
{\rm Keith Winstein}\\
Stanford University
} % end author

\hyphenation{clear-text}

\maketitle

\pagestyle{empty}

%-------------------------------------------------------------------------------
%-------------------------------------------------------------------------------
\begin{abstract}
%-------------------------------------------------------------------------------

This paper describes and evaluates new techniques for
network-originated retransmissions for end-to-end transport
connections, yielding performance benefits for encrypted transport
protocols in lossy settings.  We use set-reconciliation techniques
based on the Rateless IBLT to let the receiver efficiently acknowledge
encrypted packets to a network function, without modifying the sender
or the underlying wire format. With these tools, endpoints can receive
network-originated retransmissions appropriate for them, without an
additional layer of encapsulation, tunnel, or link-layer
retransmissions.

\end{abstract}

%-------------------------------------------------------------------------------

\section{Introduction}
\label{sec:intro}

In the Internet's canonical model, transport is end-to-end and
implemented only in hosts. Traditionally, routers and other network
components forwarded IP datagrams without regard to their payloads or flow membership~\cite{saltzer1984endtoend,
  clark1988darpa}; only hosts thought about connections, reliable
delivery, or flow-by-flow congestion control.% \dm{Connections and reliable
%  delivery, yes, but for congestion, what about stuff like RED and
%  ECN?}

In practice, however, the best behavior for a transport protocol
depends on the particulars of the network path. An appropriate retransmission
or congestion-control scheme for a heavily-multiplexed wired
network wouldn't be ideal for paths that include a high-delay
satellite link,
% unreliable Wi-Fi,
Wi-Fi with bulk ACKs and frequent reordering,
or a
cellular WWAN~\cite{kuhn2021quic-over-sat,goyal2017abc}.
%\gina{More citations?}
% Moreover, end-to-end
% retransmissions can be wasteful when a long network path includes a
% single hop with nontrivial noncongestive loss (\Cref{fig:mininet}).

By the 1990s, many networks had broken from the canonical model by
deploying in-network TCP accelerators, also known as
``performance-enhancing proxies'' (PEPs)~\cite{rfc3135}.
TCP PEPs can split
an end-to-end connection into multiple concatenated connections~\cite{kapoor2005achieving,caini2006pepsal,davern2011httpep,farkas2012splittcp,hayes2019mmwave},
buffer and retransmit packets over a lossy link~\cite{balakrishnan1995snoop,polese2017milliproxy},
virtualize congestion control~\cite{cronkite2016vcc,he2016acdc,mihaly2012mobilePEP}, resegment the byte
stream, and enable forward error correction, explicit congestion notification,
or other segment-specific enhancements.
Because TCP isn't encrypted or
authenticated, PEPs can achieve this transparently, without the knowledge or
cooperation of end hosts. Roughly 20--40\% of Internet paths cross at least one TCP PEP~\cite{imc2011handley, edeline2019bottomup}.

While many flows benefit from PEPs, their use carries a
cost: protocol ossification~\cite{papastergiou2017deossifying, edeline2019bottomup}. When a middlebox
inserts itself in a connection and enforces its preconceptions about
the transport protocol, it can thwart the
protocol's evolution, dropping traffic that uses an upgraded version or new options. TCP PEPs have hindered or complicated the
deployment of many TCP improvements, such as ECN++, tcpcrypt, TCP extended
options, and multipath TCP~\cite{mandalari2018ecnplusplus,imc2011handley,raiciu2012multipathtcp}.
%
% \dm{Are there more examples besides multipath? (maybe resegmenting,
%   ECN, TCP-AO, larger sequence numbers for high
%   bandwidth-delay-product networks, ACK suppression to save power,
%   forward error correction???, multicast???)}
% \michael{For sure we can find a reference for ECN. Multicast seems like a rathole. The other things... are they needed? Perhaps what we have is good enough?}

In response to this ossification, and to an increased emphasis on
privacy and security, post-TCP transport protocols have been designed to be
impervious to meddling middleboxes, by encrypting and
authenticating the transport header. We call these newer transport
protocols ``opaque.'' The most prevalent
is QUIC~\cite{rfc9000}, found in billions of
installed Web browsers and millions of servers~\cite{zirngibl2021quicdeployment};
other opaque transport protocols are used in WebRTC/SRTP~\cite{rfc8834webrtc},
Zoom~\cite{zoom}, BitTorrent~\cite{bittorrent}, and Mosh/SSP~\cite{winstein2012mosh}.

This opacity means that middleboxes can't interpose themselves on a
connection or understand the sequence
numbers of packets in transit.  This prevents PEPs
from providing assistance, reducing---in some situations---the
performance of opaque transport
protocols~\cite{border2020quicsat-presentation,kuhn2021quic-over-sat,martin2022bbr-quic-sat,border2022evaluating,kosek2022quicpep}.
It's possible to co-design protocols and PEPs to
preserve security and privacy while permitting assistance
from credentialed middleboxes~\cite{ford2008logjam,sherry2015blindbox,
  dogar2012tapa,iyengar2009flow}, but challenging to do so without tightly
coupling these components, risking ossification and fragility.

In this paper, we propose a method for in-network assistance of opaque
transport protocols that tries to resolve this tension. Our approach leaves
the transport protocol unchanged on the wire: a secure end-to-end connection between hosts, opaque to middleboxes and free to
evolve. No PEPs are credentialed to decrypt the transport protocol's
headers.

Instead, we propose a second protocol to be spoken on an adjacent
connection between an end host and a PEP.
We call this the \emph{\bf \sys protocol}, and its contents
are \emph{about} the packets of the underlying, or ``base,'' connection.
\Sys PEPs assist end hosts by reporting what they've observed about the packets of
the opaque base connection, without coupling their assistance to the
details of the base protocol. End hosts use this information
to influence decisions about how and when to send or resend packets on
the base connection, approximating some of the performance benefits of traditional PEPs.
A similar functional separation was first proposed by \cite{yuan2022sidecar},
but this paper presents the first concrete realization of the idea and its
nuanced interactions with real transport protocols.

One key technical challenge with this approach is how the \sys can
efficiently refer to ranges of packets in an opaque base connection. These
packets appear random to the middlebox, and referring to a range of, e.g., 100 opaque
packets in the presence of loss and reordering is not as simple as saying
``up to 100'' when there are cleartext sequence numbers.
In \Cref{sec:quack}, we present and
evaluate a mathematical tool called a \emph{\bf quACK} that concisely
represents a selective acknowledgment of opaque, randomly identified
packets. The quACK is based on the insight that we can model the
problem as a system of power sum polynomial equations if there is a
practical bound on the maximum number of ``holes'' among the
packets being ACKed. We created an optimized implementation~\cite{quack-github},
building on related theoretical
work~\cite{eppstein2011straggler,minsky2003set,karpovsky2003data}.

A second challenge is how the end host should use information from a
\sys connection to obtain a performance benefit for its base connection.
Since the performance benefit comes from changing behavior at the end host
rather than the middlebox, transport protocols need to incorporate this
information into their existing algorithms for, e.g., loss detection and
retransmission, which have gotten increasingly complex over time.
% Ideally, the benefit would match what
% a non-opaque transport protocol would receive from a traditional PEP.
To explore this, we designed a \sys protocol we call Robin, and implemented
it in three scenarios:
\begin{itemize}[noitemsep,topsep=2pt]
% \begin{itemize}[itemsep=0pt]
\item A low-latency audio stream over an Internet path that includes a Wi-Fi path segment
  (low latency with loss), followed by a WAN path segment (higher latency
  with low loss). Can the \sys PEP reduce the de-jitter buffer delay
  by triggering earlier retransmissions on loss?

\item An upload over the same path. Can an opaque transport protocol like QUIC,
  aided by a \sys PEP at the point between these two path segments, match
  the throughput of TCP over a connection-splitting PEP?

\item A battery-powered receiver, downloading data from the Internet over Wi-Fi.
  If the Wi-Fi access point sends \sys quACKs on behalf of the receiver,
  can it reduce the number of times the receiver's radio needs to wake up
  to send an end-to-end ACK?
\end{itemize}

\smallskip

A third technical challenge is how knowledge about \emph{where}
loss occurs along a path should influence a congestion-control scheme.
The challenge in any such scheme is how to maximize the congestion window
while sharing the network fairly with competing flows.
We present a path-aware modification to the CUBIC congestion-control
algorithm~\cite{ha2008cubic}, which we call \mbox{\textbf{PACUBIC}},
that approximates the congestion-control behavior of a PEP-assisted split TCP
CUBIC connection while making its decisions entirely on the host.
% that updates the congestion window proportionally to the number of
% bytes-in-flight \emph{on the path segment that experienced} the congestion
% event.

\paragraph{Summary of results.}

Concretely realized, the quACK expresses the equivalent of TCP's
cumulative + selective ACK over opaque (randomly identified) packets
in 48 bytes, tolerating up to 10 missing packets before the last
``selective ACK.'' On a recent x86-64 CPU, it takes 33 ns/packet for a \sys PEP
to encode a quACK, and 3~$\mu$s for an end host to decode it. These
overheads compared
well with several alternatives
(\Cref{sec:quack:microbenchmarks}).

We implemented Robin in a low-latency media client
based on the WebRTC standard, and an HTTP/3 client using the Cloudflare
implementation of QUIC~\cite{quiche} and the \texttt{libcurl}~\cite{libcurl}
implementation of HTTP/3. We evaluated the three scenarios in
real-world and emulation experiments.
In real-world experiments using an unmodified local Wi-Fi network to access our
nearest AWS datacenter, the \sys was able to trigger early retransmissions
to fill in gaps in the audio of a latency-sensitive audio stream, reducing the
receiver's de-jitter delay from 2.3~seconds to 204~ms---about a 91\% reduction
(\Cref{fig:real-world}). The \sys was also able to improve the speed of an
HTTP/3 (QUIC) upload by about 50\%.

In emulation experiments of the ``battery-powered receiver'' scenario,
the \sys PEP was able to reduce the need for the receiver to send ACKs
by sending proxy acknowledgments on its behalf---ACKs the sender used
to advance its flow-control and congestion-control windows. The
receiver only needed to wake up its radio to send occasional
end-to-end ACKs, which the sender used to discard data from its
buffer (\Cref{fig:ack-reduction}).

Also in an emulation experiment, we confirmed that PACUBIC's
performance approximates a split CUBIC connection (two TCP CUBIC
connections separated by a PEP), responding to loss events on the
different path segments similarly to how the individual CUBIC flows would
(\Cref{fig:loss-vs-tput}). The results indicate that the \sys protocol's gains
do not come at the
expense of congestion-control fairness relative to a split CUBIC connection.

\smallskip

The rest of this paper describes the \sys's motivating scenarios (\Cref{sec:motivation}), explores the quACK's design and implementation (\Cref{sec:quack}), discusses
the concrete \sys protocol we built around quACKs (\Cref{sec:design}) and
its implementation in two base protocols (\Cref{sec:implementation}), and then evaluates the protocol
in real-world and emulation experiments (\Cref{sec:evaluation}).

%In opaque transport protocols today, the entirety of the connection
%is governed by ACKs from the other host. Hosts typically respond to
%end-to-end ACKs in the following ways:

%\begin{enumerate}[noitemsep]
%  \item Update the congestion window.
%  \item Retransmit lost packets.
%  \item Move the sending window.
%  \item Forget packets in the retransmission buffer.
%\end{enumerate}

\begin{figure}[t]
	\centering
	% \includegraphics[width=\linewidth]{figures/sc-legend.pdf}\\
	\includegraphics[width=\linewidth]{figures/sc_protocol.pdf}%
\caption{The proxy generates quACKs, in-network acknowledgments, based on
the opaque packets it observes in the base protocol. It quACKs to an end
host, the data sender, which sends or resends packets on the base protocol as a result.
Although we only show one side of the connection, the \sys could assist
either end host of a bidirectional flow.
\vspace{-0.4cm}
}
\label{fig:sc-protocols}
\end{figure}


%What could an end host learn from acknowledgments that originate in
%the \emph{middle} of the connection?
%The end host learns which packets have been lost and received, just like from
%end-to-end ACKs, but it also learns \emph{where} on the path these events have
%occurred.
% We call this acknowledgement from the proxy a \emph{quACK}.
%Unlike a traditional PEP, which reads and interprets the packets
%of an end-to-end connection, sometimes interposing on the connection itself,
%a quACK-sending \sys enables the \emph{end host} to make performance-enhancing
%decisions based on the particulars of the network path.
% make path-aware decisions, \emph{without} violating the end-to-end principle.

%Knowing where on the path packets are lost and received is a powerful tool:
%The end host can
%update the congestion window with knowledge of where the loss occurred,
%retransmit lost packets sooner than detected by end-to-end ACKs,
%and move the sending window in lieu of energy-draining end-to-end ACKs.
%(Forgetting packets in the retransmission buffer is best left to end-to-end
%ACKs given the consequences of being unable to retransmit a lost packet.)
%We motivate the \sys protocol with three scenarios that highlight each of these
%use cases for in-network acknowledgments (\Cref{sec:motivation}).
%\gina{Describe motivating scenarios in more detail in the intro? See: early
%  retransmission over partly-lossy paths, proxy acknowledgments to save energy,
%  and a congestion-control technique to emulate a ``split'' connection by
%  considering where loss occurred along a path.}

%A key technical challenge that has to be solved to enable the \sys protocol
%is the following: \textbf{what is in a quACK and how do we decode it?}
%That is, if sequence numbers are opaque to a PEP, then how can a
%\sys protocol usefully refer to the packets of the underlying transport protocol
%in a way the host can understand?
% More specifically: \textbf{how can a PEP efficiently express a
%   ``cumulative ACK + selective ACK'' over encrypted sequence numbers?}

%We present an efficient construction of the quACK based on the insight that we
%can model the problem as a system of
%power sum polynomial equations when we have a bound on the maximum number of
%missing elements (\Cref{sec:quack}). The power sum quACK enables
%the \sys to transfer the \emph{least} amount of data for the endpoint to
%\emph{efficiently} decode \emph{exactly} which packets have been received.
%We microbenchmark our highly-optimized implementation of the mathematical
%tool to demonstrate its practicality, building on related
%theoretical work of the algebraic technique~\cite{eppstein2011straggler,minsky2003set,karpovsky2003data}.

% One meaningful contribution of our QUIC integration is a modification to the
% CUBIC congestion control algorithm that uses knowledge of \emph{where} loss
% occurs.
%Another interesting technical challenge is how to use knowledge of \emph{where}
%loss occurs to improve congestion control while remaining fair.
%We present a \emph{path-aware} modification
%to the CUBIC congestion control algorithm~\cite{ha2008cubic}
%called \textbf{PACUBIC},
%updating the congestion window proportionally to the
%number of bytes in-flight on the path segment of the congestion event.
%While the congestion behavior of PACUBIC is able to emulate that
%of connection-splitting TCP PEPs, its decisions are made at the end host.

%In the rest of this paper, we describe Robin, our design and implementations
%of the \sys protocol (\Cref{sec:design,sec:implementation}). We implement a
%library for the quACK mathematical tool and a
%\sys binary that sniffs packets and sends quACKs.
%We integrate Robin with two base protocol clients: a simple media client for
%low-latency streaming and an HTTP/3 client
%based on \texttt{libcurl}~\cite{libcurl}
%and a production implementation of QUIC from Cloudflare called \texttt{quiche}~\cite{quiche}.

% We believe that many scenarios would benefit from a \sys protocol,
% allowing the next generation of secure
% transport protocols (including QUIC) to match the performance of
% PEP-accelerated TCP, without compromising their security and privacy
% or promoting ossification. In this paper, we describe several of these
% scenarios and quantitatively evaluate one of them.

% \subsection*{Summary of results}

%We compare these base protocols both with Robin and with end-to-end
%mechanisms alone, and evaluate their performances in our motivating scenarios
%(\Cref{sec:evaluation}).
%In emulation, Robin is able to increase goodput 3.6$\times$, reduce tail latency
%22$\times$, and send 25$\times$ fewer packets at the data receiver in each
%scenario, respectively.
%We show that PACUBIC matches
%the fairness of connection-splitting TCP PEPs, and the \sys
%can process up to 183k packets/s on a single core while sending less than 3\%
%additional packets on the link to the data sender.
%We demonstrate that Robin is robust in a real-world environment with a
%lossy Wi-Fi link and a high-latency cellular path segment.
%Finally, we conclude (\Cref{sec:conclusion}).
% We believe this work does not raise substantial ethical issues.

\input{sections/protocols.tex}
%-------------------------------------------------------------------------------
\section{Scalability of the Packrat Proxy}
\label{sec:eack}
%-------------------------------------------------------------------------------
An on-path proxy handles many tens to hundreds of thousands of concurrent connections at once.
Any per-packet overhead incurred by the \Sys protocol at the proxy must be
extremely small. The \Sys proxy encodes every packet in the base connection and, unlike
the Sidekick proxy, also decodes every acknowledgment
($91\!\times$ more expensive than encoding in \cite{yuan2024sidekick}).
We want to explore whether a different type of acknowledgment of encrypted packets
can be more suitable in the setting of in-network
retransmissions for its computational efficiency.

The problem of providing a concise acknowledgment of encrypted packet identifiers
boils down to a more general problem called set reconciliation. In this problem,
two parties, each with a set, learn the items missing in the other
set~\cite{minsky2003set,eppstein2011straggler}.
There are two well-known solutions with different tradeoffs. The quACK
applies the space-optimal deterministic solution based on
power sum polynomials, which is easy to reason about for correctness~\cite{yuan2024sidekick}.
Other work applies the probabilistic Bloom filter solution
for its computational efficiency~\cite{yang2024practical,summermatter2021byzantine},
which inspires our construction of the eACK.


% In particular, we want to improve the computational efficiency of decoding
% (a $91\!\times$ more expensive operation than encoding in the power sum quACK),
% since it now incurs a per-packet overhead at the proxy compared to the Sidekick protocol,

% \thea{Emphasize here the key diff from basic connection splitters -- it's the encoding and decoding}

% \thea{For decoding, need to motivate why we know that decoding is expensive specifically.}

% \thea{I find this paragraph confusing in both the description of the problem
% and what prior work it's referring to.}
% Prior work on the \textit{(sub)set reconciliation problem}, where two parties
% each with a set learn the $m$ items missing from the subset, provides useful
% context on how to optimize these operations~\cite{eppstein2011straggler}.
% There are two main classes of solutions in prior work.
%  \thea{need to intro the power sum quack and how it was applied by your NSDI paper -
%  otherwise the transition to IBLT doesn't make sense.}
% The power sum quACK applies the
% space-optimal, deterministic solution and encodes the set as the first $m$
% power sum symmetric polynomials. Can we also apply the other solution and encode
% the set in a probabilistic Bloom Filter?
% \thea{I think this last sentence is abrupt}

% \thea{Again, I think there needs to be more motivation behind exploring IBLT quacks
% and some reference to (your) prior work that applied the power sum quack}

Set reconciliation with a Bloom filter has mainly been applied to large distributed systems~\cite
{yang2024practical,summermatter2021byzantine}. However, in the \Sys setting
where items are network packets, the encoding must must fit in a typical packet
MTU of 1500 bytes, and decoding must complete in millisecond RTT timescales or
less. We discuss the nuances of using the IBLT at a small granularity, and
also explore the tradeoffs of the eACK compared to the quACK.

Another way to reduce the impact of decoding at the proxy is simply to send smaller
and fewer eACKs. When the eACK sender is at the proxy, such as in Sidekick,
it doesn't have much flexibility in changing how often it eACKs given
the base protocol is completely opaque. But when the eACK sender is co-located
with an application endpoint, the endpoint can leverage its knowledge about the
base acknowledgment scheme to dynamically adjust the eACK.
We describe two optimizations: \textit{rateless eACKs} and \textit{selective eACKing}.

% \thea{This last part isn't really about decoding. Maybe make section about load on proxy,
% then go decoding (motivate why the focus on decoding) --> reducing load. can name that
% connection management is out of scope for this section because it's the same as
% connection splitters and is evaluated later.}

\begin{lstfloat}[t]
\begin{lstlisting}[language=Rust]
trait eACK {
    fn count() -> usize;
    fn last_identifier() -> u32;
    fn encode(item: u32);
    fn remove(item: u32);
    fn sub(rhs: eACK) -> eACK;
    fn decode() -> &[u32];
}
\end{lstlisting}
\captionof{lstlisting}{Pseudocode interface for the implementation of the eACK
 in the \texttt{eack} library. Each item is a 4-byte random identifier
 referring to a network packet.}
\label{lst:quack-interface}
\end{lstfloat}


\subsection{IBLT Acknowledgment}
\label{sec:eack:iblt}

In this section, we apply
the Rateless Invertible Bloom Lookup Table~\cite{yang2024practical}
to describe a new probabilistic construction of an acknowledgment
for encrypted packets. We describe its data structure and implementation
and formalize an interface for these acknowledgments in \Cref{lst:quack-interface}.

\paragraph{Rateless IBLT crash course.} The IBLT is like the final boss version
 of the Bloom filter~\cite{goodrich2011invertible}. The data structure consists of $t$ cells. Each
 cell is represented by an XOR and a count of items in the cell. A
 deterministic hash function maps each item to a small number of randomly
 distributed cells. The Rateless IBLT presents a novel pseudorandom mapping
 algorithm that maps an item to $O(\log(t))$ cells, with greater density in
 cells with smaller indexes~\cite{yang2024practical}. We leverage this
 efficient mapping and its rateless properties in the eACK.

The IBLT allows the insertion and deletion of items. To find the set difference
between the items in two IBLTs, we first ``subtract'' the corresponding XORs
and counts. Then we decode the items in the difference IBLT by deleting items
from cells with a count of 1 or -1 until no such cells remain. Decoding is
successful if all counts and XORs are 0 at the end, and can fail
probabilistically due to hash collisions.

% \paragraph{Data structure.} The fields in the IBLT eACK are similar: a 4-byte
% count of received elements, a $4$-byte identifier of the last packet
% received, and then $t$ coded symbols. However, each coded symbol instead of
% being a power sum, is an XOR of packet identifiers mapped to that cell as
% well as an $1$-byte count.

% The number of coded symbols maintained at the endpoint and proxy are
% pre-determined. In the case of the IBLT eACK, we generally need some
% constant factor more symbols than the threshold number of missing packets to
% decode those packets. Otherwise, the fields of the data structure are pretty
% similar, with different implementations for the function interface.

\paragraph{Data structure.}


The eACK implements
the interface defined in \Cref{lst:quack-interface} and makes direct use of the Rateless IBLT.
The \texttt{encode()} and \texttt{remove()} functions implement IBLT insertion
and deletion, while \texttt{sub()} and \texttt{decode()} implement
the decoding process of the difference IBLT.
Each item in the encoding is a 4-byte identifier referring to a packet. The
identifier is specified by a fixed offset into the randomly encrypted packet
payload. Each \texttt{Symbol} (\Cref{fig:payloads:client}) corresponds to a
cell, containing an XOR of 4-byte identifiers and a count.

The main advantage of the eACK is the computational complexity of its
operations (\Cref{tab:quack-complexity}). Unlike the quACK,
which encodes each item in all $t$ symbols, the eACK only uses
$O(\log(t))$ symbols, making both encoding and decoding more efficient.

\begin{table}[t]
    \centering
    \begin{tabular}{r l l}
        \toprule
        \bf & Power Sum~\cite{yuan2024sidekick} & IBLT \\
        \midrule
        Encode & $O(t)$ & $O(\log(t))$ \\
        Decode & $O(Nt)$ & $O(m\log(t))$ \\
        $t=$Size & $O(m)$ & $O(m)$ \\
        \bottomrule
    \end{tabular}
    \caption{The computational complexity of the encoding and decoding
    operations of each type of eACK, and the number of symbols. The IBLT
    is theoretically better or the same in all regards, but it
    incurs constant overheads in the size and in hashing.
    $N$ is the number of packets at the proxy, $m < N$ is the number of
     missing packets, and $t$ is the number of symbols.}
    \label{tab:quack-complexity}
\end{table}

\paragraph{Serialization in a packet MTU.} The main disadvantage of the IBLT in
 the \Sys setting is its size in practice. The IBLT typically requires a constant factor
 more symbols than the number of items in the set to decode the set without a
 collision. Compared to power sums, each symbol also requires a count. While
 these constant factors are inconsequential in larger distributed systems,
 which reconcile larger sets with encodings much larger than an MTU, they
 matter here.

We carefully selected the bit widths for each field. Note that each eACK needs
to fit in a single UDP datagram, which limits us to $\approx\!1400$ bytes
excluding headers. If each XOR is $4$ bytes, then there can be at most $350$
packets in the set difference. Note that $8$-byte identifiers are unnecessary
because collisions are unlikely in these small set difference sizes. The count
in each symbol needs to be as large as the set difference size. It can also be
an unsigned integer because we do subset (and not set) reconciliation, and we
can subtract with overflow. An 8-bit integer goes up to $255$, which is
sufficiently close. Thus the largest eACK contains $255$ symbols or $4 +
4 + 255 \cdot 5 = 1283$ bytes in the \texttt{eACK} payload.

% If computation matters more than link overheads and occasional failure, can
% there be a version of a quACK that uses the IBLT? The rateless feature does
% not really apply here because since round-trips are costly for the short
% feedback loop of the sidekick protocol.

\subsection{IBLT vs. Power Sum Microbenchmarks}
\label{sec:eack:microbenchmarks}

Both the IBLT and power sums are derived from set reconciliation
techniques, so when should we prefer one to the other?
We adapt the power sum quACK from \cite{yuan2024sidekick} for the interface
in \Cref{lst:quack-interface}
so that they can be used interchangeably in our microbenchmarks and compare
it to our implementation of the IBLT eACK.
In this section, we compare the overheads in practice.
We run the microbenchmarks on a single core of an AWS m4.xlarge instance.

\paragraph{Encoding.}

% The computational complexity of encoding is more efficient in the IBLT than
% power sums, updating only $O(\log(t))$ symbols instead of all $t$ (\Cref
% {tab:quack-complexity}).
Encoding in the IBLT is faster in practice when there are at least
$\approx\!30$ symbols (\Cref{fig:quack:encode}).
Each update in the IBLT uses an expensive square root instruction in the
pseudorandom mapping algorithm, adding constant factor overheads that impact
smaller numbers of symbols.

\paragraph{Decoding.}

% Decoding is also more efficient in computational complexity in the IBLT (\Cref
% {tab:quack-complexity}).
Decoding in the IBLT is much faster in practice for any number of symbols
(\Cref{fig:quack:decode}). Note that the power sum quACK actually uses a decoding
method that is linear in \textit{all} packets received, not just the \textit
{missing} packets, due to the complexity of symmetric polynomial
factorization.

\begin{figure}[t]
    \centering
    \begin{subfigure}[b]{0.49\linewidth}
        \centering
        \includegraphics[width=\linewidth]{figures/quack_encode.pdf}
        \caption{Encode time.}
        \label{fig:quack:encode}
    \end{subfigure}
    \begin{subfigure}[b]{0.49\linewidth}
        \centering
        \includegraphics[width=\linewidth]{figures/quack_decode.pdf}
        \caption{Decode time.}
        \label{fig:quack:decode}
    \end{subfigure}
    \caption{IBLT vs. power sum eACK microbenchmarks. The number of trials is
     such that the cumulative time is at least $100$ ms. With rateless eACKs,
     the client can encode much larger numbers of symbols $t$ while the proxy
     decodes fewer symbols in the common case.
     % The logscale axes emphasize the overheads at smaller numbers.
     }
    \label{fig:quack}
\end{figure}
\begin{figure}[t]
    \centering
    \includegraphics[width=0.9\linewidth]{packrat-paper/figures/quack_multiplier.pdf}
    \caption{The CDF of the minimum number of symbols $t$ needed to successfully
    decode an quACK for various numbers of missing packets $m$, 100000 trials.}
    \label{fig:iblt-quack}
\end{figure}

\paragraph{Non-determinism.} It is unknown
 how many symbols are required in the IBLT to decode the same number of
 $m$ missing packets using power sums in the previous benchmarks. There is a constant
 factor overhead ($1.35\times$ on average~\cite{yang2024practical}).
 \Cref{fig:iblt-quack} shows the CDF of the minimum number of symbols to decode
 various $m$ as this constant multiplier increases. $m=1$ trivially
 requires one symbol, while the multiplier decreases for higher $m$ to
 achieve the same success rates.

The eACK sender does not know how many symbols it needs to encode to later
decode a certain $m$. Using power sums, this is exactly $m$ symbols. In
the eACK, $4 \cdot m$ symbols have at least a $\!98.8\%$ success rate for
all $m$ evaluated in \Cref{fig:iblt-quack}. When $m$ is large, the eACK utilizes more
of the link, but maintaining more symbols at the proxy and client means
the \Sys has a greater worst-case tolerance for errors with the same encoding
overheads. The \Sys proxy can reset the connection if it cannot decode an eACK.

\paragraph{Summary.}

To make the IBLT data structure suitable for small, network packets, we
carefully consider constant factor overheads in the
number of symbols and the computational efficiency. We find the eACK
is most likely to be useful in settings with high, bursty loss, where
encoding and decoding large numbers of symbols is more efficient than the
quACK. However, the quACK is still more efficient when
loss is small or infrequent.
% An IBLT quACK with $100$ symbols can encode packets at $131$ $\mu$s/packet. In
% comparison, the power sum quACK can only encode with $42$ symbols in the same amount
% of time, having a lower worst-case tolerance for errors. Decoding $42$ missing
% packets takes $103$ $\mu$s in the IBLT quACK, $21\%$ faster than the power sum
% quACK.

\subsection{eACKing with Client Hints}
\label{sec:eack:hints}

We describe two optimizations at the client for dynamically adjusting the eACK
based on knowledge of the base connection to send smaller and fewer eACKs
overall.
We evaluate the impact of these optimizations in \Cref{sec:evaluation:link-overheads}.

\paragraph{Rateless eACKs.} The number of symbols in the eACK is configured
 for the worst case, but it is wasteful to consistently send hundreds of bytes
 over the wire for each eACK. Unlike set reconciliation in distributed
 systems, eACKs are transmitted at millisecond RTT timescales.

In the blockchain setting, the two parties negotiate to determine the number of
missing items, and to send additional symbols if the current number is not
enough to probabilistically decode. This minimizes the number of symbols sent
over the wire, but the in-network retransmission setting does not have the time
to negotiate over multiple RTTs. How else can we apply the rateless property to
eACKs?

In fact, the client can estimate the number of retransmissions it expects from
the proxy and send a smaller eACK, while locally encoding enough symbols for
the worst case. For example, the client can count the number of gaps in an ACK,
or the packets for which it would send a NACK. Both the IBLT and power sums
have the property that a strict prefix of symbols is sufficient to decode a
smaller error.

% We implement this in \texttt{serialize\_rateless()} (\Cref
% {lst:quack-interface}), and find that in practice this optimization
% significantly reduces link overheads while allowing the \Sys protocol to
% dynamically adjust to changing loss conditions (\Cref{sec:evaluation}).
% The client can also select the number of symbols with less precision without
% concern for the potential link overheads.

\paragraph{Selective eACKing.} If the client does not expect it needs a
 retransmission, there is not much point in sending an eACK. In NACK schemes,
 where the client detects loss and asks for a retransmission, it can choose to
 selectively eACK only when it would otherwise send a NACK. Note that the client can omit
 regular eACKs without the cache exploding in size because the proxy
 optimistically evicts packets, and these are likely to be received or the
 client would have NACKed.\\

\section{\Sys Protocol}
\label{sec:design}

This section describes Robin, our design for a \sys protocol built
around quACKs. This includes the setup and configuration of a Robin
\sys connection, how a sender detects loss from a quACK, and a path-aware
modification to CUBIC called PACUBIC, for congestion-controlled base protocols.

\subsection{PEP Discovery Mechanism}

\Sys connections can be configured explicitly or implicitly.  In systems that
explicitly configure proxies, such as Apple's iCloud Private Relay~\cite{icloud-private-relay}
based on MASQUE~\cite{kosek2021masque,kramer2021masquepep}, proxies can simply negotiate
sending quACKs during session establishment.  In most other settings,
such as 4G/5G cellular networks, PEPs have traditionally been deployed
as transparent proxies, silently interposing on end-to-end
connections.  Senders therefore need a way to detect transparent \sys
proxies and inform them of where to send quACKs.  Because of network
address translation, all communication to the proxy must be initiated
by the sender or use the same IP addresses and port numbers of the
base connection.

Our current design has senders signal quACK support by sending a
distinguished packet containing a 128-byte \emph{\sys-request} marker.  Such
inline signaling could confuse receivers, but {\sys}s target
protocols such as QUIC that discard cryptographically unauthenticated
data anyway.  It would be cleaner to signal support through
out-of-band UDP options~\cite{ietf-tsvwg-udp-options-28}, which we hope to do
once they are standardized.

The proxy replies to a \sys-request packet by sending a special packet
from the receiver's IP address and port number back to the sender.
This packet contains a \emph{\sys-reply} marker, an opaque session ID, and an
IP address and port number for communicating with the proxy.  Upon
receiving the \sys-reply packet, the sender begins communicating
directly with the proxy from a different UDP port.  It initially sends
back the session ID and configuration parameters to start receiving
quACKs.

\paragraph{Security.}
A malicious third-party could execute a reflection amplification attack that
generates a large amount of traffic while hiding its source. This is
possible because the sender requests quACKs to a different port and (for some
carrier-grade NATs) IP address from the underlying session. To mitigate this,
each quACK contains a quota, initially 1, of remaining quACKs the proxy will
send as well as an updated session ID\@.
The quota and session ID ensure only the sender can increase the quota or
otherwise reconfigure the session.

An adversarial PEP could send misleading information to the sender. Note that
only on-path PEPs can send credible information, since they refer to unique
packet identifiers.
To mitigate this, the sender can consider PEP feedback along with
end-to-end metrics to determine whether to keep using the PEP. The sender can
always opt out of the PEP, and the PEP cannot actively manipulate traffic any
more than outside a \sys setting.

\subsection{Configuration Messages}
\label{sec:design:configuration}

The data sender can send various other messages to the proxy
to configure the connection or reset bad state.

\paragraph{Protocol parameters.}
The sender configures (i) the quACK interval of the PEP and (ii) the threshold
number of missing packets $t$, or otherwise selects \sys-specific settings
such as how an identifier is computed.

% Senders select $b$ based on their collision tolerance and target
% quACK size. In most cases, we recommend $b=4$ bytes.

The quACK interval is expressed in terms of time or number of packets,
 e.g., every $N$ milliseconds or every $N$ packets, as in a TCP delayed ACK.
The sender determines the desired interval based on its estimated
RTT of the base connection and its application objectives, e.g.,
more frequently for latency-sensitive applications or lower-RTT paths.
%

The threshold represents the bound on the number of missing packets
between quACKs, in practice the number of ``holes'' among the packets that are
selectively ACKed. The threshold depends on the quACK interval, and
should be set based on how precise loss detection needs to be and
other qualities of the link.
For example, the threshold is larger to detect congestive loss in the queue of a
bottleneck link, or smaller to still detect transmission error on a lossy link.

% When the \sys protocol is initiated, both
% the sender and receiver initialize $t$ power sums to $0$.
% To bound the size of the quACK while preserving a unique solution, all power sum
% arithmetic is performed modulo
% % some prime number---in particular,
% the largest
% prime that can be expressed in $b$ bits.

% When the receiver is ready to send a quACK, it sends the $b\cdot t$ bits
% corresponding to its $t$ power sums, and the count, to the sender.
% % To decode the
% % quACK, the sender computes the polynomial $p(x)$ where the roots are exactly the
% % missing packet identifiers, and solves for the polynomial. As an optimization,
% % when the set of candidate roots $S$ is small, the sender evaluates $p(x)$ for
% % every packet $x \in S$ and records the zeros as missing packets.
% The sender subtracts the received count from its own count to determine the
% number of missing packets $m$.
% Note that the number of bits used to represent the count only needs to be big
% enough to represent this difference, and the count itself can wraparound.
% If the difference also wraps around, then the polynomial equations either
% cannot be solved or the solutions do not correspond to packets in $S$.


\paragraph{Resets.}
Robin allows the sender to tell the PEP to reinitialize the quACK.
This is helpful if the quACK becomes
invalid, e.g., if $m$ exceeds the threshold $t$. It is
always safe to reset the quACK, or even to ignore the \sys entirely and
fall back to the base protocol's end-to-end mechanisms.
% \dm{Do we want the PEP to be able to reset the sender?  Might be
%   useful to send one last special quACK when evicting something from
%   the cache.}

\subsection{Sender Behavior}

In this section, we discuss two particular sender-side behaviors that are enabled by
the \sys protocol and which are helpful across several scenarios: detecting packet loss
from a decoded quACK and congestion control.

\subsubsection{Detecting Loss}
\label{sec:design:detecting-loss}

The sender knows definitively which packets have been received by the proxy from
a decoded quACK. Next, it must determine from the remaining packets which ones
have been dropped and which are still in-flight, including if there has been a
reordering of packets. In-flight packets are later
classified as received or dropped based on future quACKs.

When there is no reordering, the packets that are dropped are just the ``holes''
among the packets that are selectively ACKed by the quACK. In particular, these
are the holes when considering sent packets in the order they were sent up to
the last element received, which represents the last selective ACK.
To identify these dropped packets, the sender encodes $t$ cumulative power sums
of its sent packets up to the last element received.
The difference between these power sums and the power
sums in the quACK represents the dropped packets. The sender ``removes'' the
identifiers of dropped packets from its cumulative power sums, ensuring that
the only packets that contribute to the threshold limit are those that
went missing since decoding the last quACK.
%  \michael{can we, instead of ``accumulates'', say: ``locally calculates'' ? This would seem clearer. Or are you trying to
% say that this is a continuously updated number?  If so, perhaps: ``locally calculates a cumulatively updated power sum quACK ...''?}

To account for reordering in loss detection, Robin implements an algorithm
similar to the 3-duplicate ACK rule in TCP~\cite{rfc5681tcp,rfc2001tcp}.
In TCP, if three or more duplicate ACKs are received in a row, it is a strong
indication that a segment has been lost. Robin considers a packet lost only if
three or more packets sent after the missing packet have been received.
Other mechanisms could involve timeouts for individual packets similar to the
RACK-TLP loss detection algorithm for TCP~\cite{rfc8985}.

\subsubsection{Path-Aware CUBIC Congestion Control}
\label{sec:design:cubic}

Congestion-controlled base protocols must have a congestion response to lost
packets that they retransmit due to quACKs, similar to if the loss were
discovered by the end-to-end ACK.
This ensures friendliness with end-to-end congestion control algorithms that do
consider the loss, such as CUBIC~\cite{ha2008cubic} in the presence of a
connection-splitting TCP PEP.
Here, we propose PACUBIC, an algorithm that emulates this ``split CUBIC''
behavior. PACUBIC uses knowledge of where loss occurs to improve connection
throughput compared to end-to-end CUBIC, while remaining fair to competing flows.

Recall that CUBIC~\cite{ha2008cubic} reduces its congestion window by a
multiplicative decrease factor,
$\beta = \beta^* = 0.7$, when observing loss (a congestion event), and otherwise increases
its window based on a real-time dependent cubic function with scaling factor
$C=C^*=0.4$:
\[
cwnd = C(T-K)^3 + w_{max} \text{ where } K = \sqrt[3]{\frac{w_{max}(1-\beta)}{C}}.
\]

\noindent Here, $cwnd$ is the current congestion window,
$w_{max}$ is the window size just before the last reduction,
and $T$ is the time elapsed since the last window reduction.

While a split CUBIC connection has \emph{two} congestion windows,
end-to-end PACUBIC only has \emph{one} window representing the in-flight bytes
of the end-to-end connection.
Conceptually, we want an algorithm that enables PACUBIC's single
congestion window to match the sum of the split connection's two congestion
windows.

PACUBIC effectively makes it so that we reduce and grow $cwnd$
proportionally to the number of in-flight bytes on the path segment
of where the last congestion event occurred.
Let $r$ be the estimated ratio of the RTT of the near path segment
(between the data sender and the proxy) to the RTT of the entire connection
(between end hosts).
We use $r$ as a proxy for the ratio of the number of in-flight bytes.
If the last congestion event came from a quACK, we use the same real-time
dependent cubic function but with the following
constants\footnote{See \Cref{sec:appendix:pacubic} for more intuition behind $\beta'$ and $C'$.}
\[
\beta = 1 - r(1-\beta^*)\text{ and }C = \frac{C^*}{r^3}.
\]
\noindent If the last congestion event came from an end-to-end ACK, then we use
the original $\beta$ and $C$ as above.

While this algorithm resembles the congestion behavior of split CUBIC, it is
simply an approximation. PACUBIC does not know the exact number of bytes
in-flight on each path segment, and the sum of the two congestion windows is simply a
heuristic for an inherently different split connection. The main takeaway is
that knowing where loss occurs can inform congestion control. We generally
hope that quACKs can lead to the development of smarter, path-aware algorithms.

%-------------------------------------------------------------------------------
\section{Implementation}
\label{sec:implementation}
%-------------------------------------------------------------------------------

We now describe our implementation of the \Sys protocol, which includes
the \texttt{eack} library (\Cref{sec:eack:microbenchmarks}),
a \texttt{packrat} library used for three client integrations,
and a \Sys proxy. We also describe the implementations of the
applications and baselines we use to evaluate \Sys.

\subsection{Applications}

We evaluate the \Sys protocol in three applications with different performance
metrics to explore the versatility of in-network retransmissions:
a high-throughput HTTP/3 file download, a low-latency media stream with forward
error correction, and a reliable multicast stream whose server has limited
capacity to handle end-to-end unicast retransmissions.

\paragraph{HTTP/3 file download.}

The HTTP/3 benchmark uses the default client and server in the Picoquic QUIC
implementation~\cite{picoquic}. Picoquic is an implementation of QUIC in C
based on the IETF standard used for experimentation on extensions to QUIC in
the IETF, with simplicity and correctness as its goals. Both endpoints use
CUBIC for congestion control.

% We modified the server to pre-cache data in memory and for the client to be able
% to query variable data sizes via an HTTP/GET request.

The goodput is measured by dividing the connection time by the requested data
size, excluding headers.
The connection time is measured as when the client sends the first
byte of its request to when it receives all requested bytes.
We use 25 MB which is large enough such that the goodput is stable.

\paragraph{Low-latency media with FEC.}

We implement an endpoint for streaming low-latency audio data from the server
to the client, using a simple repetition code for forward error correction.
The server (a) sends a packet every 20 ms
(b) containing audio from the last 40 ms (so there's an overlap in each
packet). Each audio packet contains 480 bytes of data, representing an audio
stream at 96 kbit/s.

The client begins playback with 40 ms in its buffer, stalling if frames arrive
too late and fast-forwarding if behind the target buffer size.
If there is a gap in the
playback buffer, the client sends a NACK with the sequence number of the
missing frame. The client retransmits NACKs, up to one per RTT, until it has
received the missing frame. On receiving a NACK, the server immediately
retransmits.

The one-way latency is measured from the time the data is produced in the real
world to when it is encoded, transmitted, and available in the client's
buffer. For example, a frame containing 40 ms of data sent over a network path
with 150 ms delay has a minimum one-way latency of 190 ms.

\paragraph{Reliable IP multicast stream.}

The IP multicast application is similar to the low-latency media application
except the server streams data to a multicast IP address. The server does not
use forward error correction, and each packet contains 240 bytes of data. Multiple clients
subscribe to the multicast IP address, and if a client is missing data,
it sends an end-to-end NACK to the
multicast server to receive a unicast retransmission.
The primary metric is the number of end-to-end retransmissions requested by
the clients. That is, we measure the number of packets that the server
must (unicast) retransmit, which captures both server and network load.

\subsection{Baselines}

The primary baseline is just the end-to-end protocol, which is the default for
encrypted transport protocols that cannot receive transport-layer assistance
from proxies. 

We additionally aim to compare \Sys to link-layer loss recovery approaches,
as discussed in \Cref{sec:background}.
To mimic these approaches,
we implement a reliable tunnel that
encapsulates and retransmits packets
similar to Wi-Fi. The tunnel sender encapsulates IP datagrams in a header with
a plaintext sequence number, and the tunnel receiver replies with a 64-bit
Block ACK~\cite{ieee80211e}. The sender retransmits gaps in the acknowledgment up
to $7$ times, the default limit in Wi-Fi, and takes care to avoid duplicates. We
also implement an option for the receiver to order packets before releasing
them to the application, behavior that we believe is unspecified in the Wi-Fi
standard but can dramatically impact application performance.
We refer to these variants as ordered and unordered tunnels.

Finally, we benchmark our HTTP/3 application against a ``true'' split
connection. We implement a \texttt{picoquic} connection splitter that decrypts
and re-encrypts packets in two separate QUIC connections from one endpoint to
another. We emphasize that these connection-splitters do not actually exist. To
be deployed, the proxy would need to be credentialed with access to underlying
sequence numbers, the antithesis of the transport purists who encrypted these
headers in the first place~\cite{duke2023rfc}. Since TCP splitters \textit
{are} commonly deployed~\cite{honda2011still,rfc3135}, our goal is to
understand how much QUIC is ``losing out'' by not splitting its connection, and
how closely \Sys can help QUIC achieve the same performance benefits without
ossification.

\subsection{Client Integrations}

\begin{lstfloat}[t]
\begin{lstlisting}[language=Rust]
struct Packrat {
    /// Serialize a Packrat message to send
    fn send_init(params: ...) -> SendBuf;
    fn send_eack(); -> SendBuf;
    fn send_eack_rateless(m: usize); -> SendBuf;
    /// Handle the incoming message and return the
    /// payload if it is a `Retransmit` message.
    fn recv_payload(buf: RecvBuf) -> &[u8]
    /// Manage eACK state
    fn is_ready() -> bool;
    fn insert(id: u32) -> bool;
    fn reset();
}
\end{lstlisting}
\vspace{-0.3cm}
\captionof{lstlisting}{Pseudocode interface for the \texttt{packrat} library.
 The library consolidates shared functionality at the client for sending
 eACKs. The client is responsible for reading and writing to the actual Packrat
 socket.}
\label{lst:quacker-interface}
\end{lstfloat}

Clients participate in the \Sys protocol by knowingly opting in to proxy
assistance and sending eACKs. Clients share much of the same functionality,
such as initializing the \Sys connection, maintaining a cumulative eACK of
received packets, and determining when to eACK based on a set frequency. We
implement a library for this shared functionality (\Cref{lst:quacker-interface})
and integrate it into our three clients.
The library uses $\approx\!700$ lines of Rust and includes C bindings.
% quacker$ cloc .

The library can serialize and deserialize messages in the \Sys connection, but
the client is responsible for reading and writing to the actual socket. Each
application uses a packet loop to interact with a socket for the base
connection. We incorporate the \Sys socket into the same loop, which allows us to
consider hints from the application for rateless and selective eACKing
(\Cref{sec:eack:hints}). We also incorporate a reorder delay in the base
connection (\Cref{sec:packet-protocol:problem}).
% Each integration uses a few hundred lines of code.


% Separately, we also implemented a ``sniffing quacker'' that sniffs the interface
% on the client associated with this base connection as opposed to being on-path.
% We find that this mostly works although we can't apply the optimizations of
% selective eACKing. Also, although we did not observe significant issues, there
% aren't guarantees about the synchronization of sniffing and base packets which
% makes correctness more difficult to reason about. However, it is a quick way to
% evaluate the potential benefits of a eACKer for a generic protocol. We do not
% evaluate the sniffing eACKer in this section.

\subsection{Proxy Implementation}
\label{sec:implementation:proxy}

% \thea{If time - I know this is just implementation, but wondering if there is
% anything we could/should do to make this section more interesting?
% Reference any design challenges? Remind people of cool design things from S3-4?}

We implement the \Sys proxy as a network bridge that uses raw sockets to read
and write packets between two interfaces. The proxy needs to
be on-path and intercept the packets, as opposed to sniffing them, so
its \texttt{InitACK} and \texttt{Reset} messages can be ordered along with the
data packets.

The proxy inspects each packet for magic discovery packets, those that match the
4-tuples of the base connections it is helping, and also eACKs.
It handles each packet as described in \Cref
{sec:packrat-protocol:proxy-behavior}. We use our \texttt{eack} library to
encode and decode eACKs. The proxy use $\approx\!1000$ lines of Rust.

% bin/sidekick.rs
% cache/base.rs
% cache/mod.rs
% sidekick/base.rs
% sidekick/mod.rs
% stream.rs

\paragraph{Multicast proxy.}

We also implement an extension to the \Sys proxy for IP multicast, or more
generally, multiple clients that share a base data stream. The proxy maintains
a fixed-size cache for the multicast 4-tuple with optimistic eviction only. For
each client, the proxy maintains a eACK and a \textit{virtual cache}. The
virtual cache contains (1) a global index in the base cache of the client's
first unacknowledged packet and (2) pairs of global indexes that indicate which
packet to insert and where to insert it because it was retransmitted to the
client. This allows the proxy to maintain state proportional only to the number
of outstanding retransmissions per client.

% \paragraph{Packet ordering.}

% The proxy needs to be on-path and order all incoming and outgoing packets in
% something like a packet loop, at least for a single connection. This is unlike
% when the sidekick connection is near the data \textit{sender}, where the proxy
% can simply sniff packets off the path.

% In our implementation, we configure the network interfaces to not forward
% packets, and then we write a program that intercepts packets from raw sockets
% for two interfaces, respectively, and forwards incoming packets from one
% interface to the other. This involves a data copy which inherently adds
% overhead, but the overhead is not fundamental if we were to use DPDK or some
% other kernel pointers.

% Serialization is important, for example, because say we process a eACK out of
% sync of the data path, we may process a eACK from the future if the proxy
% hasn't encoded packets it has already sent because it hasn't had a chance to
% sniff them. The DiscoverAck and Reset packets are both serialization points in
% a connection. This allows sidekick connections to be established or reset at
% any point in the base connection, and it is important that there is not too
% much reordering.

% \paragraph{Sidekick functionality.}

% The sidekick functionality in the proxy leverages the eACK library from \cite
% {yuan2024sidekick}, along with an extension for the IBLT eACK. It encodes each
% packet in the base connection, and then decodes eACKs, retransmitting any
% packets determined to be missing. If it hits a memory limit when encoding
% packets and adding them to the cache, it will reset the connection and
% communicate this limit to the client to configure its eACK frequency. If it
% hits a limit due to the congestion window, it will drop the packet from the
% data sender and not forward it to the data receiver.

% In addition, for shared streams such as in the IP multicast application, the
% proxy maintains a fixed-size cache for the base connection with optimistic
% eviction. It also maintains a virtual buffer of packets for each client as
% opposed to a single buffer that duplicates all the packet data. The virtual
% buffer consists of a 32-bit index for the next index in the base buffer to
% receive, as well as a ring buffer of base index and virtual index pairs at
% which to insert retransmitted packets.

% \paragraph{Cache drop policy.}

% The proxy has similar memory constraints to a connection-splitting TCP PEP per
% base connection, and we won't try to make it better than that. Since our caches
% are re-implemented, we have a similar policy that newly added packets exceeding
% the capacity are simply not added until the packets already in the cache are
% evicted.

% Similar to the TCP PEP, in addition to OS memory limits we also need some sense
% of a congestion control window. This allows us to achieve the same fairness as
% a split connection of the same congestion control algorithm. However, since the
% behavior differs based on CCA and even implementation of the CCA, we simply
% select one and call it a day.

\section{Implementation}
\label{sec:sidekick:implementation}

\begin{table}[ht]
  \centering
  \begin{tabular}{l r r}
    \hline
    \textbf{Module} & \textbf{Language} & \textbf{LOC} \\
    \hline
    Media server/client + integration & Rust & 478 \\
    \texttt{quiche} client integration & Rust & 1821 \\
    \texttt{libcurl} client integration & C & 1459 \\
    Sidekick proxy binary & Rust & 833 \\
    \hline
  \end{tabular}
  \caption{Lines of code in the Sidekick protocol implementation.
  }
  \label{tab:sidekick:lines-of-code}
\end{table}


We now describe our implementation of the Sidekick protocol for several
applications. We integrated Sidekick functionality with a simple media client
for low-latency streaming and an HTTP/3 (QUIC) client. We used the power sum
construction of the quACK from our quACK library in \Cref
{sec:quack:implementation}. The total implementation of the proxy and client
integrations used 4591 LOC (\Cref{tab:sidekick:lines-of-code}).

\subsection{End-to-end applications}
\label{sec:sidekick:implementation:applications}

The baselines we evaluated against were the performance of two secure transport
protocols without proxy assistance, and the fairness of a split CUBIC
connection.

\subsubsection{Low-latency media application.}
We implemented a simple server and client in Rust for streaming low-latency
media. The client sends a numbered packet containing 240 bytes of data every
20 milliseconds, representing an audio stream at 96 kbit/s.
The sequence number is encrypted on the wire.

The server receives packets. If it receives a nonconsecutive sequence number,
it sends a NACK back to the client that contains the sequence number of each
missing packet. The client's behavior on NACK is to retransmit the packet. The
server retransmits NACKs, up to one per RTT, until it has received the packet.

The server's application behavior is to store incoming packets in a buffer
and play them as soon as the next packet in the sequence is available. The
de-jitter buffer delay is the length of time between when the packet is stored
to when it can be played in-order. Some packets can be played immediately.

\subsubsection{HTTP/3 file upload application.}

We used the popular \texttt{libcurl}~\cite{libcurl} file transfer library as the
basis for our HTTP client, and an \texttt{nginx} webserver. The client makes an
HTTP POST request to the server. Both are patched with \texttt{quiche}~\cite
{quiche}, a production implementation of the QUIC protocol from Cloudflare, to
provide support for HTTP/3.

For our TCP baselines, we used the same file upload application with the
default HTTP/1.1 server and client. We used a split-connection
TCP PEP~\cite{caini2006pepsal} that intercepts the TCP
SYN packet in the three-way handshake, pretends to be the other side of that
connection, and initiates a new connection to the real endpoint.
Both clients use CUBIC congestion control.

\subsection{Client integrations with Sidekick}
\label{sec:sidekick:implementation:client-integrations}

In each application, we modified only the \emph{client} to speak the Sidekick
protocol and respond to in-network feedback. The server remained unchanged.
The modifications were in two parts: following the discovery mechanism to
establish bi-directional communication with the proxy, and using the information
in the quACK to modify transport layer behavior.

\begin{table*}[h]
  \centering
  % \renewcommand{\arraystretch}{0.000023}
  \small
  \begin{tabular}{lllll}
    \toprule
    & \bf Data Sender  & \bf Proxy $\leftrightarrow$ Data & \bf QuACK     & \\
    & \bf (Client) $\leftrightarrow$ Proxy & \bf Receiver (Server) & \bf Interval & \bf Threshold \\
    \midrule
    \#1 Low-latency media & $1$ms, $100$ Mbit/s, & $25$ms, $10$ Mbit/s, & $2$ pkts & $8$ \\
                          & $3.6\%$ loss         & $0\%$ loss           &          &     \\

    \#2 Connection-splitting & $1$ms, $100$ Mbit/s, & $25$ms, $10$ Mbit/s, & $30$ ms & $10$ \\
    PEP emulation            & $1.0\%$ loss         & $0\%$ loss           &         & \\

    \#3 ACK reduction & $25$ms, $10$ Mbit/s, & $1$ms, $100$ Mbit/s & $15$ ms & $50$ \\
                      & $0\%$ loss,          & $0\%$ loss          &         & \\
    \bottomrule
  \end{tabular}
  \caption{Experimental scenarios. Link 1 connects the data sender (client) to
  the proxy, while Link 2 connects the proxy to the data receiver (server).
  The quACK interval and threshold represent our Sidekick configuration.
  }
  \label{tab:experimental-scenarios}
\end{table*}


\subsubsection{Low-latency media client.}

The media client has two open UDP sockets: one for the base connection and one
for the Sidekick connection. When it receives a quACK, it detects lost packets
without reordering and immediately retransmits them. The protocol does not have
a congestion window nor a flow-control window. The client also sends reset and
configuration messages over the Sidekick connection.

\subsubsection{HTTP/3 file upload client.}

The HTTP/3 client similarly has an adjacent UDP socket for the Sidekick
connection on which it receives quACKs and sends reset and configuration
messages. The client passes the quACK to our modified \texttt{quiche} library,
which interprets the quACK and makes transport layer decisions. From the
client's perspective, \texttt{quiche} tells \texttt{libcurl} exactly what bytes
to send over the wire.

Our modified \texttt{quiche} library uses the quACK to inform the
retransmission behavior, congestion window, and flow-control window. The library
immediately retransmits lost \emph{frames} in a newly-numbered
packet, as opposed to the lost \emph{packet}, similar to QUIC's original
retransmission mechanism. We implement PACUBIC,
described in \Cref{sec:sidekick:design:sender}.
We also move the flow-control window (without forgetting packets in the
retransmission buffer), but only in the ACK reduction scenario, when the
congestion window is nearly representative of that of the Sidekick connection's
path segment.

\subsection{Sidekick proxy}
\label{sec:sidekick:implementation:proxy}

Our proxy sniffs incoming packets of a network interface using the
\texttt{recvfrom} system call on a raw socket.
It stores a hash table using Rust's standard library \texttt{HashMap} that maps
socket pairs to their respective quACKs, and incrementally updates the quACKs
for flows that have requested Sidekick assistance. It also sends quACKs at
their configured frequencies and listens for configuration messages.

\section{Evaluation methodology}
\label{sec:sidekick:methodology}

\begin{figure}[t]
\centering
\includegraphics[width=0.7\linewidth]{sidekick/figures/setup_real.pdf}
\caption{Real-world experimental setup.
}
\label{fig:sidekick:real-setup}
\end{figure}


We modeled the scenarios from \Cref{sec:sidekick:motivating}. We use the same
m4.xlarge AWS instance as before for the emulated experiments.

\subsubsection{Emulation experiments.}

We emulated a two-hop network topology (\Cref{fig:sidekick:overview}) in
mininet, configuring the link properties using \texttt{tc}.
In emulation, we represented
each link by a constant delay (with variability induced by the queue), a random
loss percentage, and a maximum bandwidth.
\Cref{tab:sidekick:experimental-scenarios} describes the parameters
for each link to model---e.g., lossy Wi-Fi or a high-latency cellular
path---as well as the metrics for success in that scenario.
Link 1 connects the data sender (client) to the proxy,
while Link 2 connects the proxy to the data receiver (server).
On the proxy, we either run a Sidekick,
a connection-splitting TCP PEP~\cite{caini2006pepsal}, or nothing at all.

\subsubsection{Real-world experiments.}

To test its robustness, we also evaluated the Sidekick protocol over a
real-world environment that resembled the scenario on the train
(\Cref{fig:sidekick:real-setup}). In this setup, a Lenovo ThinkPad laptop,
running Ubuntu 22.04.3 with a 4-Core Intel i7 CPU @ 2.60 GHz and 16 GB memory,
acted as a client to an AWS instance in the nearest geographical region. The
ThinkPad used as an access point (AP) a Lenovo Yoga laptop, running Ubuntu
20.04.6 with a 4-Core Intel i5 CPU @ 1.60 GHz and 4 GB memory, with a 2.4 GHz
Wi-Fi hotspot. The AP was connected to the Internet via a JEXtream cellular
modem with a 5G data plan. The AP ran Sidekick software.

We measured the link properties of each path segment to compare to
our emulation parameters. We measured delay and loss using 1000~\texttt{ping}s
over a 100 second period, and bandwidth using an \texttt{iperf3} test.
On the near segment between the ThinkPad client and the AP,
the min/avg/max/stdev RTT was 1.249/37.194/272.168/54.660 ms
at 49.8 Mbit/s bandwidth. We observed that loss increased
the further away the AP. In our experiments, the client was located roughly
200 feet away in a different room, with 3.6\% loss.
The far segment between the AP and the AWS server was
48.546/64.381/92.374/6.806 ms with 0.0\% loss at 30.9 Mbit/s.
In both environments, the cellular link was the bottleneck link in terms of
bandwidth, and the corresponding path segments in emulation had similar
minimum RTTs and average loss percentages.

\section{Real-world results}
\label{sec:sidekick:real-world}

\begin{figure}[t]
\centering
\begin{subfigure}{0.48\linewidth}
	\includegraphics[width=\linewidth]{sidekick/figures/fig8_real_world_webrtc.pdf}
	\caption{Low-latency media. CDF of per-packet de-jitter
	latencies over 10 one-minute trials per protocol.}
	\label{fig:sidekick:real-world:media}
\end{subfigure}
\begin{subfigure}{0.48\linewidth}
	\includegraphics[width=\linewidth]{sidekick/figures/fig8_real_world_retx.pdf}
	\caption{Path-aware congestion control.
	Median of 20 trials. Error bars are 1st and 3rd quartiles.}
	\label{fig:sidekick:real-world:pep-emulation}
\end{subfigure}
\caption{Real-world results. Experiments were run in a moderately well-attended
office environment over a Friday afternoon. Trials alternate between the
baseline and the Sidekick to account for variability in time of day.
}
\label{fig:sidekick:real-world}
\end{figure}


We discuss the results of our experiments replicating two of our scenarios in
the real world, using as context
these main differences between emulation and the real-world:

\begin{itemize}[noitemsep,topsep=0pt]
  \item The RTT is more variable as it depends on interactions in the
  wireless medium and the shared cellular path.
  \item Wireless loss can be more variable as nearby 2.4 GHz devices and
  physical barriers may interfere with the link. Wireless loss also tends
  to be more clustered in practice.
  \item The available bandwidth on the shared cellular path is more variable,
  and depends on the time of day.
\end{itemize}

\Cref{fig:sidekick:real-world} shows the results of running the low-latency
media and connection-splitting PEP emulation experiments in the real-world.
The baseline protocol with a Sidekick is able to
reduce the 99th percentile de-jitter latency of an audio stream
from 2.3~seconds to 204~ms---about a 91\% reduction---and
improve the goodput of a 50 MB HTTP/3 upload by about 50\%.
Although the improvements are more conservative compared to emulation in
\Cref{fig:sidekick:main-results:media} and
\Cref{fig:sidekick:main-results:pep-emulation}, each case still benefits the
base protocol under all circumstances, compared to end-to-end mechanisms alone.

Part of the difference can be attributed to the network setting. When there is
no loss on the near path segment, as can occasionally happen in a real Wi-Fi
link, we do not expect to see a difference with a Sidekick. When there is more
loss on the far path segment, which is variable and depends on the time, we
expect the benefit of the Sidekick to be less since this equally affects the
performance of the base protocol.

The other part of the difference could be made up by future work that better
adapts a Sidekick connection to real-world variability: The client could
improve path segment RTT estimation based on when the proxy receives packets,
and use this dynamic estimate in the calculation of $r$ used in $\beta$ and
$C$. The client could also use this estimate to dynamically adjust the quACK
interval. Finally, we could analyze theoretically how PACUBIC responds to
traffic patterns in the real world.

\section{Emulation results}
\label{sec:sidekick:emulation}

\begin{figure*}
\begin{subfigure}{0.34\textwidth}
\includegraphics[width=\linewidth]{sidekick/figures/fig4a_low_latency_media.pdf}
\caption{Scenario \#1:
 Reduced tail latency of de-jitter delay
with earlier retransmission in the low-latency media application. 5 minute trials.}
\label{fig:sidekick:main-results:media}
\end{subfigure}
\hfill
\begin{subfigure}{0.31\textwidth}
\includegraphics[width=0.97\linewidth]{sidekick/figures/fig4b_pep_emulation.pdf}
\caption{Scenario \#2: Improved goodput in the connection-splitting PEP emulation.
Error bars are the IQR of 20 trials.
}
\label{fig:sidekick:main-results:pep-emulation}
\end{subfigure}
\hfill
\begin{subfigure}{0.32\textwidth}
\includegraphics[width=0.99\linewidth]{sidekick/figures/fig4c_ack_reduction.pdf}
\caption{Scenario \#3:
High goodput independent of end-to-end ACK frequency in the ACK reduction scenario.
10 MB upload.}
\label{fig:sidekick:main-results:ack-reduction}
\end{subfigure}
\caption{
Comparing the end-to-end baseline protocol to the same protocol with a Sidekick
connection, using the success metrics for the three scenarios described in
\Cref{tab:sidekick:experimental-scenarios}.
}
% \dm{Maybe a notation like $x/4$ would be more suggestive than $4x$?}
\label{fig:sidekick:main-results}
\end{figure*}


We evaluated our implementation of the Sidekick protocol in a more controlled
emulation environment to answer the following questions:
\begin{enumerate}[noitemsep,topsep=0pt]
	\item Can Sidekicks improve the performance of secure transport protocols
	in a variety of scenarios while preserving the end-to-end behavior of the
	base protocols?
	\item Can a path-aware congestion control algorithm match the fairness of
	split TCP PEPs using CUBIC?
	\item How do the CPU overheads of encoding quACKs impact the maximum
	capacity of a proxy with a Sidekick?
	\item What link overheads does the power sum quACK add and how does it
	compare to the strawmen?
\end{enumerate}

\subsection{Performance of secure transport protocols with Sidekick}
\label{sec:sidekick:emulation:performance}

We first evaluate Sidekick's main performance goal: In each of the motivating
scenarios, we show that the Sidekick protocol can improve performance compared
to the base protocol alone, which would not be able to benefit from existing
PEPs. Each scenario has a different metric for success---tail latency,
throughput, or number of packets sent by the data receiver (corresponding to
energy usage or chance of Wi-Fi collisions)---demonstrating the versatility of
the Sidekick protocol.

\subsubsection{Low-latency media application.}

The Sidekick can reduce tail latencies in a low-latency media stream,
representing fewer drops and better quality of experience. The early
retransmissions induced by the Sidekick reduced the 99th percentile latency of
the de-jitter buffer delay from 48.6 ms to 2.2 ms---a 95\% reduction
(\Cref{fig:sidekick:main-results:media}). As long as the quACK interval is less
than the end-to-end RTT, the connection benefits from the Sidekick.

The Sidekick is beneficial in this scenario because it enables the client to
sooner detect and retransmit lost packets, and the server to sooner play
packets from its de-jitter buffer. The end-to-end mechanism takes one
additional received packet to notify of the loss and one end-to-end RTT to
retransmit and play the packet (20+52=72ms), resulting in three delayed
packets (the three ``steps" in \Cref{fig:sidekick:main-results:media}) in most
cases. The Sidekick takes up to two additional packets and one near path
segment RTT ($20+2=22$ms or $20\times2+2=42$ms), delaying either one or two
packets in comparison. Dropped ACKs and quACKs account for the $<2\%$ of
packets with even greater de-jitter latencies.

\subsubsection{Connection-splitting PEP emulation.}

The Sidekick improves upload speeds when there is a lossy, low-latency link
by using quACKs to inform the sender's congestion control.
In a scenario with $1\%$ random loss on the link between the proxy and the
data sender, the HTTP/3 (QUIC) client achieves $3.6\times$ the goodput for a
10 MB upload with a Sidekick compared to end-to-end QUIC
(\Cref{fig:sidekick:main-results:pep-emulation}).

When there is no random loss, the Sidekick does not impact the performance of
QUIC. There are no logical changes to the base protocol in this case because
all loss is on the bottleneck link on the far path segment, and the CPU
overheads of processing quACKs are negligible.

Knowing \emph{where} congestion occurs is an opportunity for creating smarter
congestion control. In PACUBIC, identifying where the loss occured let the data
sender reduce the congestion window proportionally to how many packets were
in-flight on each path segment. In \Cref{sec:sidekick:emulation:pacubic}, we
will show that our path-aware congestion control algorithm still matches the
fairness of connection-splitting TCP PEPs.

\subsubsection{ACK reduction scenario.}

Using quACKs in lieu of end-to-end ACKs allows the data receiver to
significantly reduce its ACK frequency while maintaining high goodput.
In our experiment, QUIC with a Sidekick sent $96\%$ fewer packets (mainly ACKs)
than end-to-end QUIC before the goodput dropped below 8.5 Mbit/s
(\Cref{fig:sidekick:main-results:ack-reduction}).
The quACK enables the data sender to promptly move the flow-control window
forward, as long as the last hop is reliable.

The goodput significantly degrades when reducing the end-to-end ACK frequency
without a Sidekick. When end-to-end QUIC reduces the ACK frequency to every
80 ms, the data receiver sends $247 / 138 = 1.8\times$ the packets at
$4.5 / 8.4 = 0.5\times$ the goodput, worse than QUIC with the Sidekick
in both dimensions (\Cref{fig:sidekick:main-results:ack-reduction}).
With a Sidekick, the data sender also does not need to change packet pacing to
avoid bursts in response to infrequent ACKs, which is why end-to-end QUIC
cannot send fewer than $\approx 240$ packets.

\subsubsection{Discussion: Configuring the Sidekick connection.}

\Cref{tab:sidekick:experimental-scenarios} shows the quACK interval and
threshold we elected for each scenario based on the considerations in
\Cref{sec:sidekick:design:messages}. In each experiment in
\Cref{fig:sidekick:main-results}, we also show how with less frequent quACKs
($2\times$ and $4\times$ the interval) and proportionally-adjusted thresholds,
the protocol performs worse, or more variably. Less frequent quACKs means the
client reacts later to feedback about the near path segment, and more often has
to rely on the end-to-end mechanism. The performance particularly degrades when
the quACK interval exceeds the end-to-end RTT. However, even in this case, the
base protocol with any Sidekick at all performs better than the base protocol
alone\@.

\subsection{Link overheads from sending quACKs}
\label{sec:sidekick:emulation:link-overheads}

\begin{figure}[h]
\begin{subfigure}{\columnwidth}
  % 5+
  %
  \setlength{\tabcolsep}{2pt}
  \footnotesize
  \centering
  \begin{tabular}{lccccccc}
    \toprule
    & \multicolumn{2}{c}{Data Sender$\rightarrow$} & \multicolumn{2}{c}{$\leftarrow$Proxy} & \multicolumn{2}{c}{$\leftarrow$Data Receiver} & \\
    & \bf Pkts & \bf Bytes & \bf Pkts & \bf Bytes & \bf Pkts & \bf Bytes & \bf Goodput \\
    \midrule
    QUIC E2E & $1.00\times$ & $1.00\times$ & $1.00\times$ & $1.00\times$ & $1.00\times$ & $1.00\times$ & $1.00\times$ \\
    Strawman 1a & $0.96\times$ & $1.01\times$ & \cellcolor{LighterRed}{$2.02\times$} & \cellcolor{LightestRed}{$1.56\times$} & $1.01\times$ & $1.03\times$ & \cellcolor{LighterGreen}{$3.33\times$} \\
    Strawman 1b & $0.94\times$ & $1.00\times$ & \cellcolor{LighterRed}{$2.00\times$} & \cellcolor{LightestRed}{$1.78\times$} & $1.00\times$ & $1.03\times$ & \cellcolor{LightGreen}{$3.53\times$} \\
    Strawman 1c & \cellcolor{LightestRed}{$1.83\times$} & $1.06\times$ & \cellcolor{LighterRed}{$2.01\times$} & \cellcolor{LightestRed}{$1.83\times$} & $1.00\times$ & $1.03\times$ & \cellcolor{LightGreen}{$3.46\times$} \\
    \bf \textcolor{black!50!blue}{Power Sum}   & \textcolor{black!50!blue}{\bf 0.94$\times$} & \textcolor{black!50!blue}{\bf 1.00$\times$} & \textcolor{black!50!blue}{\bf 1.03$\times$} & \textcolor{black!50!blue}{\bf 1.07$\times$} & \textcolor{black!50!blue}{\bf 1.00$\times$} & \textcolor{black!50!blue}{\bf 1.03$\times$} & \cellcolor{LightGreen}{\textcolor{black!50!blue}{\bf 3.55$\times$}} \\
    \bottomrule
  \end{tabular}
  % \includegraphics[width=\columnwidth]{figures/packet-overhead-retx.png}
  \caption{Scenario \#2: Connection-splitting PEP emulation.}
  \label{tab:packet-overhead:retx}
\end{subfigure}
\begin{subfigure}{\columnwidth}
  % \includegraphics[width=\columnwidth]{figures/packet-overhead-ackr.png}
  \setlength{\tabcolsep}{2pt}
  \footnotesize
  \centering
  \begin{tabular}{lccccccc}
    \toprule
    & \multicolumn{2}{c}{Data Sender$\rightarrow$} & \multicolumn{2}{c}{$\leftarrow$Proxy} & \multicolumn{2}{c}{$\leftarrow$Data Receiver} & \\
    & \bf Pkts & \bf Bytes & \bf Pkts & \bf Bytes & \bf Pkts & \bf Bytes & \bf Goodput \\
    \midrule
    QUIC E2E & $1.00\times$ & $1.00\times$ & $1.00\times$ & $1.00\times$ & $1.00\times$ & $1.00\times$ & $1.00\times$ \\
    Strawman 1a & $0.96\times$ & $1.00\times$ & \cellcolor{LightRed}{$9.94\times$} & \cellcolor{LighterRed}{$4.99\times$} & \cellcolor{LightGreen}{$0.04\times$} & \cellcolor{LightGreen}{$0.08\times$} & $1.02\times$ \\
    Strawman 1b & $0.96\times$ & $1.00\times$ & \cellcolor{LightRed}{$9.95\times$} & \cellcolor{LightRed}{$7.13\times$}      & \cellcolor{LightGreen}{$0.04\times$} & \cellcolor{LightGreen}{$0.08\times$} & $1.02\times$ \\
    Strawman 1c & \cellcolor{LightestRed}{$1.91\times$} & $1.05\times$ & \cellcolor{LightRed}{$9.73\times$} & \cellcolor{LightRed}{$7.41\times$}      & \cellcolor{LightGreen}{$0.04\times$} & \cellcolor{LightGreen}{$0.08\times$} & $0.97\times$ \\
    \bf \textcolor{black!50!blue}{Power Sum}    & \textcolor{black!50!blue}{\bf 0.96$\times$} & \textcolor{black!50!blue}{\bf 1.00$\times$} & \textcolor{black!50!blue}{\bf 1.09$\times$} & \cellcolor{LighterRed}{\textcolor{black!50!blue}{\bf 2.56$\times$}} & \cellcolor{LightGreen}{\textcolor{black!50!blue}{\bf 0.04$\times$}} & \cellcolor{LightGreen}{\textcolor{black!50!blue}{\bf 0.08$\times$}} & \textcolor{black!50!blue}{\bf 0.98$\times$} \\
    \bottomrule
  \end{tabular}
  \caption{Scenario \#3: ACK reduction.}
  \label{tab:packet-overhead:ackr}
\end{subfigure}
\caption{Link overheads for a 10 MB upload. The cells represent the multiplier
relative to the end-to-end QUIC baseline for each type of quACK\@.
Lower is better for number of packets and bytes sent on a link.
Higher goodput is better. Robin's power sum quACK achieves the success metric
for each scenario without incurring the link overheads of the strawmen.
We did not evaluate the contrived protocol in Scenario \#1.
}
\label{tab:packet-overhead}
\end{figure}


The other cost in terms of using Sidekick protocols is the additional data sent
by the proxy to the data sender. Too many additional bytes use up bandwidth,
and additional packets use up CPU\@. \Cref{tab:sidekick:packet-overheads} shows
the number of packets and bytes sent at each node comparing the strawmen and
power sum quACK to no Sidekick connection at all.

Using power sum quACKs increases the packets sent from the proxy to the data
sender
by 3-9\%. These packets either consist mostly
of end-to-end ACKs which are sent every packet in \texttt{quiche}, or end-to-end
ACKs that have been replaced by quACKs in the ACK reduction scenario.
We did not evaluate Scenario \#1 because it is based
on a contrived protocol that lacks many of these features, and the link
overheads would not really make sense.

This overhead is representative of the CPU overhead at the client, since
quACKs and ACKs take a similar number of cycles to process. In an experiment
with Scenario \#2 during a period of $\approx90$k incoming packets, ACKs took on
average 26065 cycles to process while the quACKs took 26369 cycles, 1\% more.
These cycles come from, i.e., the complex recovery and loss detection algorithms
implemented at the end host.

The strawmen have significantly higher link overheads compared to the power sum
quACK\@. The proxy sends up to 10$\times$ more packets using Strawman 1a, and
also slightly harms the goodput in the congestion control scenario.
The reduced goodput is due to the sender mis-identifying received packets as
dropped due to dropped quACKs.
The proxy achieves higher goodput with Strawman 1b but sends
more bytes. Strawman 1c increases the link overheads at both the proxy and the
data sender due to larger TCP headers and TCP ACKs.
We did not evaluate Strawman 2 due to its impractical decode time.

\subsection{CPU overheads of encoding at the proxy}
\label{sec:sidekick:emulation:cpu-overheads}

\begin{table}[h]
  \centering
  \small
  \begin{tabular}{lrrrr}
    \toprule
    & \multicolumn{2}{r}{\bf 25-Byte Payload} & \multicolumn{2}{r}{\bf 1468-Byte Payload}\\
    & \bf Cycles & \bf $\%$ & \bf Cycles & \bf $\%$ \\
    \midrule
    Sniff Packet & 22417 & 97.6 & 22408 & 97.5 \\
    Table Lookup &   247 &  1.1 &   251 &  1.1 \\
    Parse ID     &    23 &  0.1 &    22 &  0.1 \\
    Encode ID    &    74 &  0.3 &    69 &  0.3 \\
    Other        &   213 &  0.9 &   225 &  1.0 \\
    \midrule
    \emph{Total} & \emph{22974} & \emph{100.0} & \emph{22975} & \emph{100.0} \\
    \bottomrule
  \end{tabular}
  \caption{Breakdown of the CPU cycles spent processing each packet at the
  proxy. Most cycles are spent on general per-packet overheads as opposed to
  quACK-specific processing.
  }
  \label{tab:cpu-overhead}
\end{table}


The main bottleneck of Sidekick on a proxy is the CPU\@.
\Cref{tab:sidekick:cpu-overheads} shows a breakdown of the number of CPU cycles
in each step. The largest overhead was reading the packet contents from the
network interface ($97.5\%$ of the CPU cycles).

Encoding an identifier in a power sum quACK with $t=10$ used $74$ CPU
cycles ($0.9\%$). As a calculation of the theoretical maximum on a 2.30 GHz
% 2.30e9 / 74 = 31 million
CPU, the proxy would be able to process $31$ million packets/second on a single
core. The hash table lookup used $251$ cycles and parsing the pseudorandom
payload as an identifier used $22$ cycles.

In practice, we measured the maximum throughput of our Sidekick proxy to
be 464k packets/s with 25-byte payloads and 5.5 Gbit/s (458k packets/s) with
1468-byte packet payloads on a single core (assuming 1500-byte MTUs).
This experiment used multiple \texttt{iperf3} clients to simulate high
load until the proxy was unable to keep up with the load on a single core.
The packet payload size did not seem to affect results.

We find these achieved throughputs acceptable for edge routers such as Wi-Fi APs
and base stations. To deploy the Sidekick proxy on core routers, we would need
to reduce the overhead of reading packets from the NIC, such as by bypassing
the kernel/user-space protection boundary\footnote{ A kernel-bypass system like
Retina~\cite{wan2022retina} can achieve 25 Gbps on 2 cores while processing raw
packets with a 1000-cycle callback(Figure 5(a) in \cite{wan2022retina}). The
Sidekick equivalent would be a 500-cycle callback, and assuming all traffic has
requested Sidekick help. Throughput scales almost linearly with the number of
cores using symmetric RSS hashing. Thus we don't expect proxy overheads to be
an issue with modern 100 Gbps network speeds and an optimized implementation
even on commodity hardware. }~\cite{dpdk,mccanne1993bsd,wan2022retina} or using
native hardware~\cite{bosshart2014p4}. We could also scale on multiple cores
using symmetric RSS hashing~\cite{woo2012scalable}.

\subsection{TCP friendliness of path-aware CUBIC}
\label{sec:sidekick:emulation:pacubic}

\begin{figure}[t]
\centering
\includegraphics[width=\columnwidth]{sidekick/figures/fig5_baseline_bar_legend.pdf}
\begin{subfigure}{0.49\linewidth}
	\includegraphics[width=\linewidth]{sidekick/figures/fig5_baseline_loss0p.pdf}
	\caption{0\% loss.}
	\label{fig:sidekick:fairness-bar:loss0p}
\end{subfigure}
\begin{subfigure}{0.49\linewidth}
	\includegraphics[width=\linewidth]{sidekick/figures/fig5_baseline_loss1p.pdf}
	\caption{1\% loss.}
	\label{fig:sidekick:fairness-bar:loss1p}
\end{subfigure}
\caption{Median goodput for three upload data sizes with $0\%$ and $1\%$ loss on
Link 1. 20 trials. Error bars are 1st and 3rd quartiles.
With proxy assistance at $1\%$
loss, both QUIC and TCP match the performance of when there is no loss at all.
}
\label{fig:sidekick:fairness-bar}
\end{figure}

\begin{figure}[t]
\centering
\includegraphics[width=0.8\columnwidth]{sidekick/figures/fig6_legend.pdf}
\includegraphics[width=0.8\columnwidth]{sidekick/figures/fig6_loss_bw100_10M_delay_25ms_1ms.pdf}
\caption{Connection-splitting PEP emulation as a function of near-segment
	loss rate. In this emulation experiment, QUIC+Sidekick (running PACUBIC)
  performs similarly to TCP+PEP (each connection running CUBIC)
  and improves goodput compared with end-to-end protocols. The graph shows
  median goodput of a 10~MByte upload. QuACK interval is 30~ms, threshold
is 10. Error bars show IQR of 10 trials.
}
\label{fig:sidekick:fairness-line}
\end{figure}


It is easy to improve performance without regard to competing flows;
however, we demonstrate that PACUBIC can
match the fairness of split CUBIC in a TCP PEP connection\@.
We evaluate fairness using Scenario \#2 with varying amounts of loss on the
near path segment.

\subsubsection{QUIC vs.\ TCP\@.}
We first compare QUIC to TCP without either PEP\@.
As both connections use CUBIC, they exhibit similar
congestion control behavior and achieve nearly maximum throughput in the
emulated network with no random loss (\Cref{fig:sidekick:fairness-bar:loss0p}).
We attribute the differences to the slightly different retranmission and
loss recovery behaviors of QUIC and TCP\@. The PEPs do not affect the
performance.

With even a little loss on the near path segment, both QUIC and TCP dramatically
worsen, respectively achieving $28\%$ and $42\%$ of the goodput at $0\%$ loss,
for a 10 MB upload (\Cref{fig:sidekick:fairness-bar:loss1p}).
% 0.305 / 1.098 = 27.8%
% 0.467 / 1.121 = 41.7%
In both protocols, CUBIC treats every transmission error as a congestion event,
even though no amount of reducing the congestion window affects the error rate.
QUIC and TCP perform similarly to each other with proxy assistance and 1\%
loss on the near path segment.

\subsubsection{Sidekick vs.\ TCP PEP\@.}

\Cref{fig:sidekick:fairness-line} shows that QUIC with a Sidekick roughly
matches---as intended---the behavior of TCP with a PEP-assisted split
connection. At higher loss rates, the near path segment becomes the bottleneck
link even with earlier feedback about loss, causing the performance of TCP with
proxy assistance to drop. QUIC with a Sidekick follows a similar pattern
because of its path-aware congestion-control scheme
(\Cref{sec:sidekick:design:sender}). The results indicate that the Sidekick
protocol's gains do not come at the expense of congestion-control fairness
relative to the split TCP connection.

\section{Limitations}

The \sys approach, and our experiments, are subject to some
limitations, which we describe briefly here.

\paragraph{Multipath scenarios.}
We have only considered \sys proxies along a single path, and not thought
extensively about how quACKs would interact with protocols such as
\mbox{TCPLS}~\cite{rochet2020tcpls} that use multiple paths or streams,
or even multipath QUIC~\cite{de2017multipath}.
To begin thinking about this question, we would have a more complex model of
the network: multiple PEPs along a single path, multiple paths each with varying
numbers of PEPs, and so on. The proxy can include
additional information in the \sys-reply packet to indicate which path the
PEP assistance is on, and the sender can infer from the RTT how far along a path each PEP
is relative to others. New \sys algorithms that come from this model could
diagnose troublesome paths, or better allocate network traffic in a multipath
connection. Existing algorithms could be applied to individual paths as if they
were single-path connections.

\paragraph{Even more diverse network scenarios.}
The three scenarios we explored all consisted of a lossy Wi-Fi link and a
high-latency WAN link. Not all scenarios will be favorable to the
\sys protocol we designed.
If the ``lossy'' section of a network path were on the far path segment from the
sender, the sender would not have any more information about the problematic
link. To accomodate scenarios like this, \sys protocols will need
more features. For example, the \emph{proxy} would need some way to receive
quACKs from the data \emph{receiver}, as well as a mechanism to buffer and
retransmit packets~\cite{balakrishnan1995snoop,caini2006pepsal}.

There are likely other scenarios that could benefit from \sys protocols as
described, but we did not evaluate them. For example, if we replaced the lossy Wi-Fi
link with a modern wireless link that has a fluctuating physical
capacity~\cite{niu2015survey,burchardt2014vlc,koenig2013wireless},
the sender may be able to more quickly adapt and make
data available for transmission whenever capacity intermittently becomes available.
% More dynamic real-world evaluations, such as actually riding a train, will be
% needed to characterize the practical improvement that {\sys}s can yield in
% true real-world situations.

% \paragraph{Overheads at \sys proxies aren't trivial.}

% Computing quACKs is inexpensive and plausible for a Wi-Fi access point---less
% than 200~ns/pkt on a commodity CPU, or $\approx$2.5~Gbit/s per core in software.
% But \Cref{tab:cpu-overhead} shows that, at
% least in our current software implementation, vastly greater overhead
% comes just from \emph{maintaining per-flow state}
% accessed on a per-packet basis. To make \sys
% proxies feasible on affordable devices, this will need to be accelerated.


\paragraph{Practical deployment.}

The implementation of Robin exists as a research system that has been evaluated
in emulation and a limited set of real-world scenarios. Since \sys protocols
require the cooperation of middleboxes and client applications, more work will
be needed to standardize the discovery protocol and wire format of \sys messages
described in \Cref{sec:design}, ideally with interest from the IETF.
The standards will need to establish several design choices such as how
identifiers are computed, how quACKs are transmitted, and the exact mechanisms
for security and backwards compatibility.
We may also want to standardize sender behavior for specific base protocols,
though this could be opaque except to the sender.

The deployment of \sys protocols can be gradual and backwards-compatible
with parties that are either unaware of or do not want to participate in \sys
protocols.
To migrate existing client applications, one needs to modify the code to
discover a PEP and use information in a quACK to inform the base protocol.
To migrate middleboxes, they would need to be modified to listen for
\sys-request markers, then accumulate and send quACKs for participating
connections.


\paragraph{Deeper analysis of path-aware congestion control.}

The correspondence between endpoint-driven PACUBIC and ``split CUBIC''
is good, and both are better than end-to-end CUBIC in
\Cref{fig:loss-vs-tput}), but not exact. The appropriateness of the
PACUBIC heuristic, and in general the idea of path-aware congestion
control, needs to be further explored. We discuss this more in
\Cref{sec:appendix:pacubic}.


\chapter{Conclusion}
\section{Summary}
\section{Future Work}

\subsection{Real-World Studies}
\subsection{Deployment and Standardization}
\subsection{Set Reconciliation in Packet-Scale Settings}

\section{Concluding Remarks}
...
\prefacesection{Quacknowledgments}
I would like to thank...


%-------------------------------------------------------------------------------
\bibliographystyle{abbrv}
\bibliography{reference}

\newpage
\appendix
\section{Appendix: Intuitive analysis of path-aware CUBIC}
\label{sec:sidekick:appendix}

\begin{figure*}[ht]
\centering
\includegraphics[width=0.8\linewidth]{sidekick/figures/cwnd_legend.pdf}\\
\begin{subfigure}{0.32\linewidth}
  \includegraphics[width=\linewidth]{sidekick/figures/cwnd_split_loss0p.pdf}
  \caption{Split CUBIC, 0\% loss.}
  \label{fig:time-cwnd:split-loss0p}
\end{subfigure}
\begin{subfigure}{0.32\linewidth}
  \includegraphics[width=\linewidth]{sidekick/figures/cwnd_pacubic_loss0p.pdf}
  \caption{PACUBIC, 0\% loss.}
  \label{fig:time-cwnd:pacubic-loss0p}
\end{subfigure}
\begin{subfigure}{0.32\linewidth}
  \includegraphics[width=\linewidth]{sidekick/figures/cwnd_cubic_loss0p.pdf}
  \caption{CUBIC, 0\% loss.}
  \label{fig:time-cwnd:cubic-loss0p}
\end{subfigure}
\begin{subfigure}{0.32\linewidth}
  \includegraphics[width=\linewidth]{sidekick/figures/cwnd_split_loss1p.pdf}
  \caption{Split CUBIC, 1\% loss.}
  \label{fig:time-cwnd:split-loss1p}
\end{subfigure}
\begin{subfigure}{0.32\linewidth}
  \includegraphics[width=\linewidth]{sidekick/figures/cwnd_pacubic_loss1p.pdf}
  \caption{PACUBIC, 1\% loss.}
  \label{fig:time-cwnd:pacubic-loss1p}
\end{subfigure}
\begin{subfigure}{0.32\linewidth}
  \includegraphics[width=\linewidth]{sidekick/figures/cwnd_cubic_loss1p.pdf}
  \caption{CUBIC, 1\% loss.}
  \label{fig:time-cwnd:cubic-loss1p}
\end{subfigure}
\caption{Congestion window of a long-running upload in Scenario \#2
(\Cref{tab:sidekick:experimental-scenarios}) with $0\%$ and $1\%$ loss on the
near path segment. The cwnd is measured at the data sender,
except for split CUBIC whose split connection also has a cwnd at the proxy.
PACUBIC reacts to every congestion event while keeping the cwnd high.
CUBIC performs poorly when there is loss on the near path segment.
CUBIC and PACUBIC are implemented in QUIC, while split CUBIC is implemented
in TCP using a PEP.
}
\label{fig:time-cwnd}
\end{figure*}

Here, we dive deeper into the intuition behind the PACUBIC constants
(\Cref{sec:sidekick:design:sender}), including how they were derived and why
the PACUBIC algorithm achieves similar congestion behavior to the CUBIC
algorithm in a split connection---we call this behavior ``split CUBIC''.

Consider the same network topology as \Cref{fig:sidekick:overview} in which a
data sender uploads a large file to a data receiver, with help from a Sidekick
proxy in the middle of the connection. The near path segment connects the sender
to the proxy, and the far path segment connects the proxy to the receiver.
The near segment is low-delay with varying random loss, and the far segment is
high-delay with no random loss. The far segment is the bottleneck link in terms
of bandwidth.
The actual link parameters are the same as in Scenario \#2 of
\Cref{tab:sidekick:experimental-scenarios}.

We first discuss how split CUBIC would behave in this setting to conceptually
motivate PACUBIC. Consider the congestion windows of each half of the split
connection, one taken at the data sender and one at the proxy
(\Cref{fig:time-cwnd:split-loss0p,fig:time-cwnd:split-loss1p}). The far path
segment experiences only congestive loss, leading the window at the proxy to
fluctuate around the segment's BDP regardless of the loss on the near path
segment. The window at the data sender independently determines whether the
packets that reach the proxy will be able to fully utilize the window set at
the far path segment. The data sender is able to achieve this at low random
loss rates, but becomes the bottleneck as loss rates increase
(\Cref{fig:sidekick:fairness-line}).

While split CUBIC has two windows, PACUBIC only has one
window representing the in-flight bytes of the end-to-end connection.
PACUBIC considers loss detected from both quACKs and end-to-end ACKs.
Conceptually, we want an algorithm that would enable PACUBIC's single
congestion window to match the sum of CUBIC's two congestion windows, or
the total number of in-flight bytes.
% That is the motivation behind adjusting the window proportionally to the
% number of in-flight bytes on each path segment, depending on where the loss
% occurred.

With no random loss on the near path segment, PACUBIC
(\Cref{fig:time-cwnd:pacubic-loss0p}) behaves the same as normal CUBIC
(\Cref{fig:time-cwnd:cubic-loss0p}). The congestion window is entirely governed
by end-to-end ACKs since the far path segment is the bottleneck link. Note that
while the sender may be able to deduce that a loss occurred on the far path
segment by combining info from the quACK with the end-to-end ACK, PACUBIC
conservatively treats the loss as occurring anywhere on the path.

With some random loss on the near path segment, PACUBIC grows and reduces cwnd
based on where the last congestion event occurs
(\Cref{fig:time-cwnd:pacubic-loss1p}). Note that if the congestion window $cwnd$
represents the bytes in-flight in the end-to-end connection, then $r \cdot cwnd$
represents the proportion of bytes in-flight on the near path segment. At a
high level, if the data sender discovers loss on the near path segment via the
quACK, it holds the $(1-r)\cdot cwnd$ portion of the ``far window'' constant
while applying the CUBIC algorithm to the remaining $r \cdot w_{max}$ of the
``near window,'' representing the bottleneck link.

Mathematically, instead of reducing $w_{max}$, the window size just before the
last reduction, by $(1-\beta^*) \cdot w_{max}$, PACUBIC reduces it by only
$[1 - (1-r(1-\beta^*))] \cdot w_{max} = r(1-\beta^*) \cdot w_{max}$.
That is $r$ times the original reduction, a \emph{smaller} amount.
We use the RTT ratio $r$ (near path segment to end-to-end)
% indicating the RTT ratio of near path segment vs.~far path segment)
as a proxy for the ratio of the number of in-flight bytes.

Similarly, instead of using a cubic growth function with scaling factor $C^*$
and inflection point $K = K^* = \sqrt[3]{w_{max}(1-\beta^*)/C^*}$,
we use a larger scaling factor $C = C^*/r^3$
and thus a shorter inflection point
\[
K = \sqrt[3]{\frac{w_{max}(1-\beta)}{C}}
= \sqrt[3]{\frac{r\cdot w_{max}(1-\beta^*)}{C^* / r^3}}
= r^{4/3} \cdot K^*.
\]
The shorter inflection point leads the congestion window to \emph{grow more
quickly} since the sender also reacts to feedback about loss more quickly over
the low-delay link.
% Since similarly proportional to $w_{max}$, so with $C'$, we both reduce the
% inflection point $K$ and the growth rate $C$ of the function proportionally to
% the number of bytes in-flight on the near path segment.

At times, there can be loss detected both in quACKs and in end-to-end ACKs.
The end-to-end ACKs have a greater effect since they reduce the congestion
window by a larger proportion, until the remaining path segment with loss is the
bottleneck link. In this scenario with loss, the bottleneck link at equilibrium
is the near path segment.
At this point, the quACK primarily determines the congestion window updates. If
the far path segment were to become the bottleneck again, the data sender would
detect a congestion event via the end-to-end ACK.

PACUBIC has several limitations. Although it beats end-to-end CUBIC, it still
performs worse than split CUBIC, especially at high loss rates
(\Cref{fig:sidekick:fairness-line}). Also, it doesn't consider loss on the far
path segment any differently than original CUBIC, unlike split CUBIC which
treats the two split connections independently. PACUBIC emulates the congestion
control behavior and fairness of split CUBIC fairly well as a heuristic, but
would benefit from an analysis in a wider variety of network scenarios. It
would also benefit from a side-by-side fairness comparison against other
congestion control algorithms that perform well in the same scenarios. We'd
like the primary takeaway of PACUBIC to be that knowing where loss occurs can
cleverly inform congestion control.


%%%%%%%%%%%%%%%%%%%%%%%%%%%%%%%%%%%%%%%%%%%%%%%%%%%%%%%%%%%%%%%%%%%%%%%%%%%%%%%%
\end{document}
%%%%%%%%%%%%%%%%%%%%%%%%%%%%%%%%%%%%%%%%%%%%%%%%%%%%%%%%%%%%%%%%%%%%%%%%%%%%%%%%

%%  LocalWords:  endnotes includegraphics fread ptr nobj noindent
%%  LocalWords:  pdflatex acks
