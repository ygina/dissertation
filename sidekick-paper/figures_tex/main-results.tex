\begin{figure*}
% python latencies.py --percentile 99 --box-and-whiskers --http base quack_2p_8 -t 20
% python latencies.py --percentile 99 --box-and-whiskers -t 20
\begin{subfigure}{0.34\textwidth}
\includegraphics[width=\linewidth]{figures/fig4a_low_latency_media.pdf}
\caption{Scenario \#1: Low-latency media.
 Reduced tail latency of de-jitter delay
with earlier retransmission. 5 minute trials.}
\label{fig:media}
\end{subfigure}
\hfill
% python data_size_vs_tput.py --mean --median -t 10 --http quic quack_30ms_10 quack_60ms_20 quack_120ms_40
% python data_size_vs_tput.py --mean --median -t 10 --http quack_30ms_10 quic
\begin{subfigure}{0.31\textwidth}
\includegraphics[width=0.97\linewidth]{figures/fig4b_pep_emulation.pdf}
\caption{Scenario \#2: Connection-splitting PEP emulation. Improved goodput.
20 trials median. Error bars are 1st and 3rd quartiles.
}
\label{fig:baseline-line}
\end{subfigure}
\hfill
\begin{subfigure}{0.32\textwidth}
\includegraphics[width=0.99\linewidth]{figures/fig4c_ack_reduction.pdf}
\caption{Scenario \#3: ACK reduction.
High goodput independent of end-to-end ACK frequency.
10 MB upload.}
\label{fig:ack-reduction}
\end{subfigure}
\caption{
Comparing the end-to-end baseline protocol to the same protocol with a \sys
connection, using the success metrics for the three scenarios described in
\Cref{tab:experimental-scenarios}. The \textsf{\Sys($N$x)} data points show
the performance at $N$x the quACK interval (sent less frequently) and
threshold of the default configurations specified in
\Cref{tab:experimental-scenarios}.
\vspace{-0.3cm}
}
% \dm{Maybe a notation like $x/4$ would be more suggestive than $4x$?}
\label{fig:main-results}
\end{figure*}
