\begin{table*}[h]
  \centering
  % \renewcommand{\arraystretch}{0.000023}
  \small
  \begin{tabular}{llllllll}
    \toprule
                 &                 &            & \bf QuACK     & \bf Thre-     &                                  & \bf Emu-   & \bf Real- \\
    \bf Scenario & \bf Link 1      & \bf Link 2 & \bf Interval & \bf shold     & \bf Success Metric               & \bf lated? & \bf World? \\
    \midrule
    \#1 Low-  & $1$ ms delay, $3.6\%$  & $25$ ms delay, $0\%$ & $2$ pkts & $8$ & Reduce tail latency of how long    & Yes & Yes \\
    latency           & loss, $100$ Mbit/s   & loss, $10$ Mbit/s    &         &      & packets are queued in the data     & & \\
    media &                      &                      &         &      & receiver's de-jitter buffer.   & & \\

    \#2 Connec-   & $1$ ms delay, $1.0\%$  & $25$ ms delay, $0\%$ & $30$ ms & $10$ & Achieve high throughput; match   & Yes & Yes \\
    tion- split-   & loss, $100$ Mbit/s   & loss, $10$ Mbit/s    &         &      & the performance, congestion con- & & \\
    ting PEP &                      &                      &         &      & trol behavior, and fairness of   & & \\
    emulation   &                      &                      &         &      & connection-splitting TCP PEPs.   & & \\

    \#3 ACK          & $25$ ms delay, $0\%$ & $1$ ms delay, $0\%$  & $15$ ms & $50$ & Reduce ACK frequency of data re- & Yes & No \\
    reduction    & loss, $10$ Mbit/s    & loss, $100$ Mbit/s   &         &      & ceiver; achieve high throughput. & & \\
    \bottomrule
  \end{tabular}
  \caption{Experimental scenarios. Link 1 connects the data sender (client) to
  the proxy, while Link 2 connects the proxy to the data receiver (server).
  The quACK interval and threshold represent our \sys configuration.
  \vspace{-0.3cm}
  }
  \label{tab:experimental-scenarios}
\end{table*}
