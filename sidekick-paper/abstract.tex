\begin{abstract}
In response to concerns about protocol ossification and privacy,
post-TCP transport protocols such as QUIC and Web\-RTC include
end-to-end encryption and authentication at the transport layer. This
makes their packets opaque to middleboxes, freeing the
transport protocol to evolve but preventing some in-network
innovations and performance improvements. This paper describes
\emph{\sys protocols}: an approach to in-network assistance
for opaque transport protocols
where in-network intermediaries help endpoints %that use opaque transport protocols
by sending information adjacent to the underlying connection, which
remains opaque and unmodified on the wire.

A key technical challenge is how the \sys connection can efficiently
refer to ranges of
packets of the underlying connection without the ability to observe
cleartext sequence numbers. We present a mathematical tool
called a \emph{quACK} that concisely represents a selective acknowledgment
of opaque packets, without access to cleartext sequence numbers.

In real-world and emulation-based evaluations, the \sys improved
performance in several scenarios: early retransmission
over lossy Wi-Fi paths, proxy acknowledgments to save energy, and a
path-aware congestion-control mechanism we call \emph{PACUBIC} that emulates a
``split'' connection.

\end{abstract}
