\chapter{QuACKs}
\label{sec:quack}

In a Sidekick connection, a proxy needs to be able to refer to and efficiently
acknowledge a set of packets even when they are randomly encrypted. Since
packets in a TCP connection are labeled with cleartext sequence numbers, it is
straightforward to refer to a set of TCP packets using cumulative and selective
ACKs. But this problem is technically challenging for packets of secure
transport protocols without access to cleartext sequence numbers or the
ossification of other fields.

We mathematically define the quACK problem in the context of network packets
sent over a lossy path segment, and discuss how to select \emph{identifiers} to
refer to encrypted packets. We analyze strawman solutions to the quACK problem
that use either too much space or computation. Then, we present two efficient
constructions of the quACK inspired by related theoretical work in set
reconciliation~\cite{minsky2003set,eppstein2011straggler}, as well as coding
theory and graph theory~\cite{karpovsky2003data}. Both constructions require a
bound on the maximum number of missing elements, a threshold $t$, and have
different tradeoffs. The \textit{power sum} construction is space-optimal and
deterministic, but requires an exact bound. The \textit{invertible Bloom lookup
table (IBLT)} construction is more computationally scalable with an approximate
bound, but also probabilistic with constant factor overheads.

Concretely realized, the power sum quACK uses 48~bytes to express the equivalent
of TCP's cumulative + selective ACK over the 32-bit identifiers of randomly
encrypted packets, tolerating up to 10 missing packets before the last
``selective ACK.'' On a recent x86-64 CPU, it takes 26~ns to encode a single
packet in the power sum quACK, and 15.5~$\mu$s to decode which received packets
the quACK represents. The IBLT quACK takes 42~ns to encode a single packet and
1.3~$\mu$s to decode a quACK, using at least 58~bytes. These overheads compared
well with several alternatives
(\Cref{sec:quack:psum-microbenchmarks,sec:quack:iblt-microbenchmarks}).

\section{The QuACK Problem}
\label{sec:quack:problem}

In the quACK problem, a sender (such as an endpoint) transmits a
multiset\footnote{A ``multiset'' means the same element can be transmitted more
than once.} of elements $S$. At any given time, a receiver (such as a proxy)
has received a subset $R \subseteq S$ of the sent elements. We would like the
receiver to communicate a small amount of information to the sender, who then
efficiently decodes the missing elements---the set difference $S \setminus
R$---knowing $S$. This small amount of information is called the ``quACK''. The
problem is: \textbf{what is in a quACK and how do we decode it?}

\subsection{Identifiers}
\label{sec:quack:problem:identifiers}

How exactly do we refer to the elements in the quACK problem that have been sent
or received? In a networking context, these elements are packets. Traditional
TCP proxies have been able to interpose their own concise, cumulative
acknowledgments using cleartext sequence numbers, but this is not possible with
modern, secure transport protocols. Even if a connection did expose an
unencrypted numerical field, we would not want to refer to that field at risk
of ossifying that protocol.

Instead, we need a function that deterministically maps
a packet to a random $b$-byte \emph{identifier}. The most trivial solution
that applies to all base protocols is
to hash the entire payload. Another option if the payload is already
pseudorandom (e.g., QUIC) is to take the first $b$ bytes from a fixed
offset of that payload. Although the latter option would rely on those bytes
to remain pseudorandom, it is computationally more efficient because it
does not require reading the entire payload.

\subsubsection{Collisions.}
The main considerations when selecting the number of bytes, $b$, in an
identifier is the tolerance for collisions compared with the extra data
needed to refer to these packets on the link. The larger $b$ is, the lower
the collision probability but the greater the link overhead.

Define the collision probability to be the probability that a randomly-chosen
$b$-byte identifier in a list of $n$ packets maps to more than one packet in
that list.
If we assume that identifiers are uniformly distributed,
this probability is equal to $1-(1 - 1/256^{b})^{n-1}$.
When $n=25$, \gina{use $n=500$?} using $4$ bytes results in an almost negligible chance of
collision while using $2$ bytes results in a 0.04\% chance
(\Cref{tab:collision-prob}).

\begin{table}[ht]
  \centering
  \begin{tabular}{|l|l|l|l|l|}
    \hline
    \textbf{Identifier Bytes} & 1 & 2 & 4 & 8 \\ % & 64 \\
    \hline
    \textbf{Collision Prob.}
    % & 0.98 & 0.015 & $2.3\text{e-07}$ & ${\approx}0$ \\ %$5.4\text{e-17}$ \\
    & 0.090 & 0.0004 & $5.6\text{e-09}$ & ${\approx}0$ \\ %$5.4\text{e-17}$ \\
    \hline
  \end{tabular}
  \caption{Collision probabilities for $n=25$.
  \vspace{-0.2cm}
  }
  \label{tab:collision-prob}
\end{table}


When handling collisions, a sender who is decoding a quACK has a list of $n$
packets it is trying to classify as received or missing
(\Cref{sec:quack:constructions}). Note that collisions are also known to the
sender beforehand. If there is a collision between a packet that is received
and a packet that is missing, the fate of that identifier is considered
indeterminate. Either the protocol can still function with approximate
statistics (e.g., congestion control) or it can fall back to an end-to-end
mechanism (e.g., retransmission).

\subsection{Strawman solutions}
\label{sec:quack:problem:strawmen}

\begin{table*}[ht]
  \centering
  % \renewcommand{\arraystretch}{0.000023}
  \small
  \begin{tabular}{llllll}
    \toprule
    & \bf & \bf & \bf Num Additional & \bf Packet Payload & \bf Cumu-\\
    & \bf Per-Packet Encode Time & \bf Decode Time & \bf Proxy Packets & \bf Size (bytes) & \bf lative?\\
    \midrule
    Strawman 1a & Parse identifier & N/A & $n$ & $b$ & No \\
    Strawman 1b & Parse identifier, move sliding & N/A & $n$ & $b \cdot window$ & No \\
    & window  & & & & \\
    Strawman 1c & Parse identifier & N/A & $n$ (TCP headers) & $b$ & No \\
    Strawman 2 & Parse identifier, & Concatenate and hash & $1$ & $32+4$ & Yes \\
    & concatenate and hash & $\binom{n}{m}$ subsets & & (hash and count) & \\
    Power Sums & Parse identifier, $t$ modular & Plug $n$ candidate roots into a & $1$ & $4+b+b\cdot t$ & Yes \\
    & multiplications and additions & degree-$m$ polynomial OR solve & & (count, last value, & \\
    & & system of $m$ polynomial equations & & $t$ power sums) & \\
    \bottomrule
  \end{tabular}
  \caption{Strawmen compared to the power sum quACK representing $n$ packets
  sent by the data sender, $m$ missing packets, and $b$-byte identifiers. The
  power sum quACK uses the threshold $t$. The total data overhead of each
  quACK must consider the packet payload size along with transport headers.
  We evaluate the overheads in practice in \Cref{sec:sidekick:evaluation:cpu-overheads}.
  }
  \label{tab:strawmen-theoretical}
\end{table*}

\begin{table}[t]
    \centering
    \begin{tabular}{r l l}
        \toprule
        \bf & Power Sum~\cite{yuan2024sidekick} & IBLT \\
        \midrule
        Encode & $O(t)$ & $O(\log(t))$ \\
        Decode & $O(Nt)$ & $O(m\log(t))$ \\
        $t=$Size & $O(m)$ & $O(m)$ \\
        \bottomrule
    \end{tabular}
    \caption{The computational complexity of the encoding and decoding
    operations of each type of eACK, and the number of symbols. The IBLT
    is theoretically better or the same in all regards, but it
    incurs constant overheads in the size and in hashing.
    $N$ is the number of packets at the proxy, $m < N$ is the number of
     missing packets, and $t$ is the number of symbols.}
    \label{tab:quack-complexity}
\end{table}

A problem that is simple with cumulative and selective acknowledgments of
plaintext sequence numbers is deceivingly challenging for pseudorandom
packet identifiers. Consider the following strawman solutions to the quACK
problem:

\subsubsection{Strawman 1: Echo every identifier.}
Strawman 1a, similar to \cite{li-tsvwg-loops-problem-opportunities-06,kramer2020lwpep},
echoes the identifier of every received packet in a new UDP packet to the data
sender.  Decoding is trivial given the identifiers are unmodified.
This strawman adds significant link overhead in terms of additional packets.
Additionally, since the strawman is not cumulative, losing a quACK means the
end host could falsely consider a packet to be lost, creating a congestion
event or spurious retransmission.

Strawman 1b echoes a sliding window of identifiers over UDP such that there is overlap
in the identifiers referred to by consecutive quACKs.
This solution is slightly more resilient to loss, but uses more bytes and is
still not guaranteed to be reliable.
Another variant batches identifiers to reduce the number of packets, but this
solution is even less resilient to loss.

We also consider a Strawman 1c that echoes every identifier over TCP with
\texttt{TCP\_NODELAY} to send every identifier in its own packet.
This ensures there are no false positives when detecting lost packets,
but adds even more link overhead in terms of TCP headers and additional ACKs
from the data sender (every other packet by default in the Linux kernel).

\subsubsection{Strawman 2: Cumulative hash of all identifiers.}
Strawman 2 sends a SHA-256 hash of a sorted concatenation of all the
received packets in a UDP packet, and the sender hashes every subset of the same size of
sent packets until it finds the subset with the same hash (assuming collision resistance).
The strawman includes
a count of the packets received to determine the size of the subset to hash.
As the number of missing packets exceeds even a moderate amount, the number
of subsets to calculate a hash for explodes, making the strawman impractical to decode.\\

\noindent
One might also suggest the receiver send negative acknowledgments of the packets
it has not received. However, unlike sequence numbers where one can
determine a gap in received packets, there is no way to tell with random
identifiers what packet is missing or should be expected next.

\section{Efficient constructions of the quACK in packet-scale settings}
\label{sec:quack:constructions}

The problem of providing a concise acknowledgment of encrypted packet
identifiers closely resembles a more general problem called set reconciliation.
In this problem, two parties, each with a set, communicate a small amount of
information to learn the set difference---the elements missing in the other
set~\cite{minsky2003set,eppstein2011straggler}. There are two well-known classes
of solutions to this set reconciliation problem with different tradeoffs
in communication and computation cost.
We provide background in the literature on the set reconciliation problem,
then describe the efficient power sum and invertible Bloom lookup table (IBLT)
constructions of the quACK based on these two classes of solutions.

\subsection{Background: Set reconciliation}
\label{sec:quack:constructions:background}

The first class of solutions models the problem as a system of symmetric power
sum polynomial equations. This construction is deterministic as the set
difference is just the solutions to this system of equations. It is also
space-optimal; \cite{minsky2003set} introduced symmetric polynomials as a
solution to the set reconciliation problem with nearly optimal communication
and tractable computation cost. \cite{dodis2004fuzzy} improved the computation
cost in decoding, but found the problem to be intractable for larger parameters.
The power sum solution has mainly been restricted to more theoretical
literature.

The second class of solutions encodes the elements in an Invertible Bloom Lookup
Table (IBLT), and is more computationally scalable than the first class. The
IBLT is a data structure based on the Bloom filter~\cite{goodrich2011invertible},
and probabilistically decodes the set difference with a multiplicative factor
on the comunication cost. This multiplicative factor has been shown to be small
on average: $1.23$\footnote{The $1.23\times$ multiplicative factor on the
communication cost is derived from $1/\alpha$, where $\alpha \approx 0.81$ in
\cite{baek2023simple}.}--$1.35\times$~\cite{yang2024practical,baek2023simple}.
In practice, set reconciliation with a Bloom-filter-like data structure has
mainly been applied to larger distributed systems such as blockchain and social
media synchronization~\cite{yang2024practical,summermatter2021byzantine}.

The quACK setting differs from previous applications of set reconciliation in
several ways. In this setting, the elements in the problem are network packets,
or more specifically packet \textit{identifiers}. Only the receiver
communicates information, and the sender performs subset and not set
reconciliation on the missing packets. Unlike the blockchain setting, the two
parties synchronize on millisecond-RTT timescales compared to seconds. The
communication must be encoded in a single network packet, not a stream of
packets. In addition, at these smaller timescales, fewer elements
(typically tens) are reconciled at once, and optimizations for constant factor
overheads matter more.

\subsection{Power sums: A space-optimal and deterministic solution}
\label{sec:quack:constructions:power-sum}

We describe a construction of the quACK that models the problem as a system of
power sum polynomial equations when we have an exact bound on the maximum
number of missing elements, a threshold $t$. Unlike the previous strawmen, this
construction is efficient to decode, and its size is proportional only to $t$.

Consider the simplest case, when the receiver is only missing a single element.
The receiver maps packet identifiers to a finite field,
i.e. modulo the largest prime that fits in $b$ bytes,
 and communicates the sum $\sum_{x \in R} x$ of the received
elements to
the sender. The sender computes the sum $\sum_{x \in S} x$ of the sent elements
and subtracts the sum from the receiver, calculating:
\[
    \sum_{x \in S} x - \sum_{x \in R} x = \sum_{x \in S\setminus R} x,
\]
which is the sum of elements in the set difference. In the case of a single
missing element, the sum is exactly the value of the missing element.

In fact, we can generalize this scheme to any number of missing elements $m$.
Instead of transmitting only a single sum, the receiver communicates
the first $m$ \emph{power sums} to the sender, where the $i$-th power sum of a
multiset $R$ is defined as $\sum_{x \in R} x^i$.
The sender then computes the first $m$ power sums of $S$ and calculates the
respective differences $d_i$ for $i \in [1,m]$, producing the following
system of $m$ equations:
\[
    \left\{\, \sum_{x \in S\setminus R} x^i = d_i \mid i \in [1,m] \right\}.
\]

Instead of transmitting an unbounded number of power sums, the receiver only
maintains and sends the first $t$ power sums. Efficiently solving these $t$
power sum polynomial equations in $t$ variables in a finite field is a
well-understood algebra problem~\cite{eppstein2011straggler}. The solutions are
exactly $x \in S \setminus R$.

\subsection{Invertible Bloom lookup table: A computationally scalable approach}
\label{sec:quack:constructions:iblt}

Now we apply the IBLT to describe a probabilistic construction of the quACK in
a Bloom-filter-like data structure. The IBLT construction has better computation
complexity than the power sum construction, but incurs higher constant factor
overheads and is probabilistic.
Recall that the IBLT data structure consists of some number of cells, where each
cell contains an XOR and a count of elements in the cell. In the IBLT, $t$ is
only an approximate bound on the number of missing elements, so we set the
number of cells to some multiplicative factor more than $t$.

When the sender or receiver adds an element to the IBLT, we leverage the
pseudorandom mapping algorithm of the Rateless IBLT~\cite
{yang2024practical}. This algorithm maps each element to $O(\log(t))$ cells,
with a greater density in the cells with smaller indexes. This resembles
rateless error-correcting codes with an infinite stream of coded symbols in
that a prefix of an IBLT with an infinite number of cells can be used to
deterministically decode the elements in the IBLT.

The IBLT also allows the removal of elements and subtraction of two IBLTs with
the same number of cells. To find the set difference between the elements in
two IBLTs, we first ``subtract'' the corresponding XORs and counts. To decode
the elements encoded in the difference IBLT, we remove elements from cells with
a count of 1 until no such cells remain. (Note that the count cannot be
negative with subset reconciliation.) Decoding is successful if all counts
and XORs are 0 at the end, and can fail probabilistically due to collisions in
the mapping algorithm.

% the Rateless Invertible Bloom Lookup Table~\cite{yang2024practical}
% to describe a new probabilistic construction of an acknowledgment
% for encrypted packets. We describe its data structure and implementation
% and formalize an interface for these acknowledgments in \Cref{lst:quack-interface}.

% \subsubsection{Rateless IBLT crash course.} The IBLT is like the final boss version
%  of the Bloom filter~\cite{goodrich2011invertible}. The data structure consists of $t$ cells. Each
%  cell is represented by an XOR and a count of items in the cell. A
%  deterministic hash function maps each item to a small number of randomly
%  distributed cells. The Rateless IBLT presents a novel pseudorandom mapping
%  algorithm that maps an item to $O(\log(t))$ cells, with greater density in
%  cells with smaller indexes~\cite{yang2024practical}. We leverage this
%  efficient mapping and its rateless properties in the quACK.

% The IBLT allows the insertion and deletion of items. To find the set difference
% between the items in two IBLTs, we first ``subtract'' the corresponding XORs
% and counts. Then we decode the items in the difference IBLT by deleting items
% from cells with a count of 1 or -1 until no such cells remain. Decoding is
% successful if all counts and XORs are 0 at the end, and can fail
% probabilistically due to hash collisions.

\section{Implementation}
\label{sec:quack:implementation}

The implementation of the quACK must be practical for packet-scale settings. The
communication must fit in a single network packet, ideally even less to be
considered a control packet for congestion control purposes. For example, a TCP
ACK is only 20 bytes on the wire, excluding Ethernet and IP headers. The
computation cost must also be practical for decoding set differences at
millisecond-RTT timescales and encoding packets when they arrive every few
nanoseconds. The implementation should also provide a convenient interface to
interchangeably use different constructions of the quACK.

\subsection{Wire format}

The actual format of the quACK includes three fields: (i) a 4-byte count of
received elements, (ii) the identifier of the last element received, and
(iii) $t$ symbols. The power sum quACK symbol is a power sum, and for the IBLT
quACK it is a count and an XOR of identifiers. We assume quACKs to be sent over
UDP, though the actual mechanism is not tied to the design. Since the decoder
does not know the number of missing packets $m$ ahead of time, the decoder
takes the difference between the number of packets it has sent and the count in
the quACK to calculate $m$. Sending the last element received is an
optimization that allows $m$ to represent just the ``holes'' among the packets
being selectively ACKed, excluding the possibly many consecutive elements that
are in-flight (\Cref{sec:sidekick:sender}).

\subsubsection{Serializing the IBLT quACK in a packet MTU.}

The main disadvantage of the IBLT in
the Packrat setting is its size in practice. The IBLT typically requires a constant factor
more symbols than the number of items in the set to decode the set without a
collision. Compared to power sums, each symbol also requires a count. While
these constant factors are inconsequential in larger distributed systems,
which reconcile larger sets with encodings much larger than an MTU, they
matter here.

We carefully selected the bit widths for each field. Note that each quACK needs
to fit in a single UDP datagram, which limits us to $\approx\!1400$ bytes
excluding headers. If each XOR is $4$ bytes, then there can be at most $350$
packets in the set difference. Note that $8$-byte identifiers are unnecessary
because collisions are unlikely in these small set difference sizes. The count
in each symbol needs to be as large as the set difference size. It can also be
an unsigned integer because we do subset (and not set) reconciliation, and we
can subtract with overflow. An 8-bit integer goes up to $255$, which is
sufficiently close. Thus the largest quACK contains $255$ symbols or $4 +
4 + 255 \cdot 5 = 1283$ bytes in the \texttt{quACK} payload.

\subsection{QuACK API}

\begin{lstfloat}[t]
\begin{lstlisting}[language=Rust]
trait eACK {
    fn count() -> usize;
    fn last_identifier() -> u32;
    fn encode(item: u32);
    fn remove(item: u32);
    fn sub(rhs: eACK) -> eACK;
    fn decode() -> &[u32];
}
\end{lstlisting}
\captionof{lstlisting}{Pseudocode interface for the implementation of the eACK
 in the \texttt{eack} library. Each item is a 4-byte random identifier
 referring to a network packet.}
\label{lst:quack-interface}
\end{lstfloat}


The API for the quACK consists of \texttt{encode()} and \texttt{remove()}
functions to add and remove individual packet identifiers from the quACK,
respectively. When the sender receives a quACK to decode, it \texttt{sub()}s
it from the quACK that has encoded its sent packets, and then \texttt{decode()}s
the subsequent difference quACK to get the elements in the difference.

The power sum quACK implements the \texttt{encode()}, \texttt{remove()},
\texttt{sub()}, and \texttt{decode()} functions of the programming interface...

The IBLT quACK implements
the interface defined in \Cref{lst:quack-interface} and makes direct use of the Rateless IBLT.
The \texttt{encode()} and \texttt{remove()} functions implement IBLT insertion
and deletion, while \texttt{sub()} and \texttt{decode()} implement
the decoding process of the difference IBLT.
Each item in the encoding is a 4-byte identifier referring to a packet.
Each \texttt{Symbol} (\Cref{fig:payloads:client}) corresponds to a
cell, containing an XOR of 4-byte identifiers and a count.

The main advantage of the IBLT quACK is the computational complexity of its
operations (\Cref{tab:quack-complexity}). Unlike the power sum quACK,
which encodes each item in all $t$ symbols, the IBLT quACK only uses
$O(\log(t))$ symbols, making both encoding and decoding more efficient.

\section{Power sum microbenchmarks}
\label{sec:quack:psum-microbenchmarks}

%-------------------------------------------------------------------------------
\section{Scalability of the Packrat Proxy}
\label{sec:eack}
%-------------------------------------------------------------------------------
An on-path proxy handles many tens to hundreds of thousands of concurrent connections at once.
Any per-packet overhead incurred by the \Sys protocol at the proxy must be
extremely small. The \Sys proxy encodes every packet in the base connection and, unlike
the Sidekick proxy, also decodes every acknowledgment
($91\!\times$ more expensive than encoding in \cite{yuan2024sidekick}).
We want to explore whether a different type of acknowledgment of encrypted packets
can be more suitable in the setting of in-network
retransmissions for its computational efficiency.

The problem of providing a concise acknowledgment of encrypted packet identifiers
boils down to a more general problem called set reconciliation. In this problem,
two parties, each with a set, learn the items missing in the other
set~\cite{minsky2003set,eppstein2011straggler}.
There are two well-known solutions with different tradeoffs. The quACK
applies the space-optimal deterministic solution based on
power sum polynomials, which is easy to reason about for correctness~\cite{yuan2024sidekick}.
Other work applies the probabilistic Bloom filter solution
for its computational efficiency~\cite{yang2024practical,summermatter2021byzantine},
which inspires our construction of the eACK.


% In particular, we want to improve the computational efficiency of decoding
% (a $91\!\times$ more expensive operation than encoding in the power sum quACK),
% since it now incurs a per-packet overhead at the proxy compared to the Sidekick protocol,

% \thea{Emphasize here the key diff from basic connection splitters -- it's the encoding and decoding}

% \thea{For decoding, need to motivate why we know that decoding is expensive specifically.}

% \thea{I find this paragraph confusing in both the description of the problem
% and what prior work it's referring to.}
% Prior work on the \textit{(sub)set reconciliation problem}, where two parties
% each with a set learn the $m$ items missing from the subset, provides useful
% context on how to optimize these operations~\cite{eppstein2011straggler}.
% There are two main classes of solutions in prior work.
%  \thea{need to intro the power sum quack and how it was applied by your NSDI paper -
%  otherwise the transition to IBLT doesn't make sense.}
% The power sum quACK applies the
% space-optimal, deterministic solution and encodes the set as the first $m$
% power sum symmetric polynomials. Can we also apply the other solution and encode
% the set in a probabilistic Bloom Filter?
% \thea{I think this last sentence is abrupt}

% \thea{Again, I think there needs to be more motivation behind exploring IBLT quacks
% and some reference to (your) prior work that applied the power sum quack}

Set reconciliation with a Bloom filter has mainly been applied to large distributed systems~\cite
{yang2024practical,summermatter2021byzantine}. However, in the \Sys setting
where items are network packets, the encoding must must fit in a typical packet
MTU of 1500 bytes, and decoding must complete in millisecond RTT timescales or
less. We discuss the nuances of using the IBLT at a small granularity, and
also explore the tradeoffs of the eACK compared to the quACK.

Another way to reduce the impact of decoding at the proxy is simply to send smaller
and fewer eACKs. When the eACK sender is at the proxy, such as in Sidekick,
it doesn't have much flexibility in changing how often it eACKs given
the base protocol is completely opaque. But when the eACK sender is co-located
with an application endpoint, the endpoint can leverage its knowledge about the
base acknowledgment scheme to dynamically adjust the eACK.
We describe two optimizations: \textit{rateless eACKs} and \textit{selective eACKing}.

% \thea{This last part isn't really about decoding. Maybe make section about load on proxy,
% then go decoding (motivate why the focus on decoding) --> reducing load. can name that
% connection management is out of scope for this section because it's the same as
% connection splitters and is evaluated later.}

\begin{lstfloat}[t]
\begin{lstlisting}[language=Rust]
trait eACK {
    fn count() -> usize;
    fn last_identifier() -> u32;
    fn encode(item: u32);
    fn remove(item: u32);
    fn sub(rhs: eACK) -> eACK;
    fn decode() -> &[u32];
}
\end{lstlisting}
\captionof{lstlisting}{Pseudocode interface for the implementation of the eACK
 in the \texttt{eack} library. Each item is a 4-byte random identifier
 referring to a network packet.}
\label{lst:quack-interface}
\end{lstfloat}


\subsection{IBLT Acknowledgment}
\label{sec:eack:iblt}

In this section, we apply
the Rateless Invertible Bloom Lookup Table~\cite{yang2024practical}
to describe a new probabilistic construction of an acknowledgment
for encrypted packets. We describe its data structure and implementation
and formalize an interface for these acknowledgments in \Cref{lst:quack-interface}.

\paragraph{Rateless IBLT crash course.} The IBLT is like the final boss version
 of the Bloom filter~\cite{goodrich2011invertible}. The data structure consists of $t$ cells. Each
 cell is represented by an XOR and a count of items in the cell. A
 deterministic hash function maps each item to a small number of randomly
 distributed cells. The Rateless IBLT presents a novel pseudorandom mapping
 algorithm that maps an item to $O(\log(t))$ cells, with greater density in
 cells with smaller indexes~\cite{yang2024practical}. We leverage this
 efficient mapping and its rateless properties in the eACK.

The IBLT allows the insertion and deletion of items. To find the set difference
between the items in two IBLTs, we first ``subtract'' the corresponding XORs
and counts. Then we decode the items in the difference IBLT by deleting items
from cells with a count of 1 or -1 until no such cells remain. Decoding is
successful if all counts and XORs are 0 at the end, and can fail
probabilistically due to hash collisions.

% \paragraph{Data structure.} The fields in the IBLT eACK are similar: a 4-byte
% count of received elements, a $4$-byte identifier of the last packet
% received, and then $t$ coded symbols. However, each coded symbol instead of
% being a power sum, is an XOR of packet identifiers mapped to that cell as
% well as an $1$-byte count.

% The number of coded symbols maintained at the endpoint and proxy are
% pre-determined. In the case of the IBLT eACK, we generally need some
% constant factor more symbols than the threshold number of missing packets to
% decode those packets. Otherwise, the fields of the data structure are pretty
% similar, with different implementations for the function interface.

\paragraph{Data structure.}


The eACK implements
the interface defined in \Cref{lst:quack-interface} and makes direct use of the Rateless IBLT.
The \texttt{encode()} and \texttt{remove()} functions implement IBLT insertion
and deletion, while \texttt{sub()} and \texttt{decode()} implement
the decoding process of the difference IBLT.
Each item in the encoding is a 4-byte identifier referring to a packet. The
identifier is specified by a fixed offset into the randomly encrypted packet
payload. Each \texttt{Symbol} (\Cref{fig:payloads:client}) corresponds to a
cell, containing an XOR of 4-byte identifiers and a count.

The main advantage of the eACK is the computational complexity of its
operations (\Cref{tab:quack-complexity}). Unlike the quACK,
which encodes each item in all $t$ symbols, the eACK only uses
$O(\log(t))$ symbols, making both encoding and decoding more efficient.

\begin{table}[t]
    \centering
    \begin{tabular}{r l l}
        \toprule
        \bf & Power Sum~\cite{yuan2024sidekick} & IBLT \\
        \midrule
        Encode & $O(t)$ & $O(\log(t))$ \\
        Decode & $O(Nt)$ & $O(m\log(t))$ \\
        $t=$Size & $O(m)$ & $O(m)$ \\
        \bottomrule
    \end{tabular}
    \caption{The computational complexity of the encoding and decoding
    operations of each type of eACK, and the number of symbols. The IBLT
    is theoretically better or the same in all regards, but it
    incurs constant overheads in the size and in hashing.
    $N$ is the number of packets at the proxy, $m < N$ is the number of
     missing packets, and $t$ is the number of symbols.}
    \label{tab:quack-complexity}
\end{table}

\paragraph{Serialization in a packet MTU.} The main disadvantage of the IBLT in
 the \Sys setting is its size in practice. The IBLT typically requires a constant factor
 more symbols than the number of items in the set to decode the set without a
 collision. Compared to power sums, each symbol also requires a count. While
 these constant factors are inconsequential in larger distributed systems,
 which reconcile larger sets with encodings much larger than an MTU, they
 matter here.

We carefully selected the bit widths for each field. Note that each eACK needs
to fit in a single UDP datagram, which limits us to $\approx\!1400$ bytes
excluding headers. If each XOR is $4$ bytes, then there can be at most $350$
packets in the set difference. Note that $8$-byte identifiers are unnecessary
because collisions are unlikely in these small set difference sizes. The count
in each symbol needs to be as large as the set difference size. It can also be
an unsigned integer because we do subset (and not set) reconciliation, and we
can subtract with overflow. An 8-bit integer goes up to $255$, which is
sufficiently close. Thus the largest eACK contains $255$ symbols or $4 +
4 + 255 \cdot 5 = 1283$ bytes in the \texttt{eACK} payload.

% If computation matters more than link overheads and occasional failure, can
% there be a version of a quACK that uses the IBLT? The rateless feature does
% not really apply here because since round-trips are costly for the short
% feedback loop of the sidekick protocol.

\subsection{IBLT vs. Power Sum Microbenchmarks}
\label{sec:eack:microbenchmarks}

Both the IBLT and power sums are derived from set reconciliation
techniques, so when should we prefer one to the other?
We adapt the power sum quACK from \cite{yuan2024sidekick} for the interface
in \Cref{lst:quack-interface}
so that they can be used interchangeably in our microbenchmarks and compare
it to our implementation of the IBLT eACK.
In this section, we compare the overheads in practice.
We run the microbenchmarks on a single core of an AWS m4.xlarge instance.

\paragraph{Encoding.}

% The computational complexity of encoding is more efficient in the IBLT than
% power sums, updating only $O(\log(t))$ symbols instead of all $t$ (\Cref
% {tab:quack-complexity}).
Encoding in the IBLT is faster in practice when there are at least
$\approx\!30$ symbols (\Cref{fig:quack:encode}).
Each update in the IBLT uses an expensive square root instruction in the
pseudorandom mapping algorithm, adding constant factor overheads that impact
smaller numbers of symbols.

\paragraph{Decoding.}

% Decoding is also more efficient in computational complexity in the IBLT (\Cref
% {tab:quack-complexity}).
Decoding in the IBLT is much faster in practice for any number of symbols
(\Cref{fig:quack:decode}). Note that the power sum quACK actually uses a decoding
method that is linear in \textit{all} packets received, not just the \textit
{missing} packets, due to the complexity of symmetric polynomial
factorization.

\begin{figure}[t]
    \centering
    \begin{subfigure}[b]{0.49\linewidth}
        \centering
        \includegraphics[width=\linewidth]{figures/quack_encode.pdf}
        \caption{Encode time.}
        \label{fig:quack:encode}
    \end{subfigure}
    \begin{subfigure}[b]{0.49\linewidth}
        \centering
        \includegraphics[width=\linewidth]{figures/quack_decode.pdf}
        \caption{Decode time.}
        \label{fig:quack:decode}
    \end{subfigure}
    \caption{IBLT vs. power sum eACK microbenchmarks. The number of trials is
     such that the cumulative time is at least $100$ ms. With rateless eACKs,
     the client can encode much larger numbers of symbols $t$ while the proxy
     decodes fewer symbols in the common case.
     % The logscale axes emphasize the overheads at smaller numbers.
     }
    \label{fig:quack}
\end{figure}
\begin{figure}[t]
    \centering
    \includegraphics[width=0.9\linewidth]{packrat-paper/figures/quack_multiplier.pdf}
    \caption{The CDF of the minimum number of symbols $t$ needed to successfully
    decode an quACK for various numbers of missing packets $m$, 100000 trials.}
    \label{fig:iblt-quack}
\end{figure}

\paragraph{Non-determinism.} It is unknown
 how many symbols are required in the IBLT to decode the same number of
 $m$ missing packets using power sums in the previous benchmarks. There is a constant
 factor overhead ($1.35\times$ on average~\cite{yang2024practical}).
 \Cref{fig:iblt-quack} shows the CDF of the minimum number of symbols to decode
 various $m$ as this constant multiplier increases. $m=1$ trivially
 requires one symbol, while the multiplier decreases for higher $m$ to
 achieve the same success rates.

The eACK sender does not know how many symbols it needs to encode to later
decode a certain $m$. Using power sums, this is exactly $m$ symbols. In
the eACK, $4 \cdot m$ symbols have at least a $\!98.8\%$ success rate for
all $m$ evaluated in \Cref{fig:iblt-quack}. When $m$ is large, the eACK utilizes more
of the link, but maintaining more symbols at the proxy and client means
the \Sys has a greater worst-case tolerance for errors with the same encoding
overheads. The \Sys proxy can reset the connection if it cannot decode an eACK.

\paragraph{Summary.}

To make the IBLT data structure suitable for small, network packets, we
carefully consider constant factor overheads in the
number of symbols and the computational efficiency. We find the eACK
is most likely to be useful in settings with high, bursty loss, where
encoding and decoding large numbers of symbols is more efficient than the
quACK. However, the quACK is still more efficient when
loss is small or infrequent.
% An IBLT quACK with $100$ symbols can encode packets at $131$ $\mu$s/packet. In
% comparison, the power sum quACK can only encode with $42$ symbols in the same amount
% of time, having a lower worst-case tolerance for errors. Decoding $42$ missing
% packets takes $103$ $\mu$s in the IBLT quACK, $21\%$ faster than the power sum
% quACK.

\subsection{eACKing with Client Hints}
\label{sec:eack:hints}

We describe two optimizations at the client for dynamically adjusting the eACK
based on knowledge of the base connection to send smaller and fewer eACKs
overall.
We evaluate the impact of these optimizations in \Cref{sec:evaluation:link-overheads}.

\paragraph{Rateless eACKs.} The number of symbols in the eACK is configured
 for the worst case, but it is wasteful to consistently send hundreds of bytes
 over the wire for each eACK. Unlike set reconciliation in distributed
 systems, eACKs are transmitted at millisecond RTT timescales.

In the blockchain setting, the two parties negotiate to determine the number of
missing items, and to send additional symbols if the current number is not
enough to probabilistically decode. This minimizes the number of symbols sent
over the wire, but the in-network retransmission setting does not have the time
to negotiate over multiple RTTs. How else can we apply the rateless property to
eACKs?

In fact, the client can estimate the number of retransmissions it expects from
the proxy and send a smaller eACK, while locally encoding enough symbols for
the worst case. For example, the client can count the number of gaps in an ACK,
or the packets for which it would send a NACK. Both the IBLT and power sums
have the property that a strict prefix of symbols is sufficient to decode a
smaller error.

% We implement this in \texttt{serialize\_rateless()} (\Cref
% {lst:quack-interface}), and find that in practice this optimization
% significantly reduces link overheads while allowing the \Sys protocol to
% dynamically adjust to changing loss conditions (\Cref{sec:evaluation}).
% The client can also select the number of symbols with less precision without
% concern for the potential link overheads.

\paragraph{Selective eACKing.} If the client does not expect it needs a
 retransmission, there is not much point in sending an eACK. In NACK schemes,
 where the client detects loss and asks for a retransmission, it can choose to
 selectively eACK only when it would otherwise send a NACK. Note that the client can omit
 regular eACKs without the cache exploding in size because the proxy
 optimistically evicts packets, and these are likely to be received or the
 client would have NACKed.\\


We benchmark our optimized implementation of the power sum quACK~\cite{quack-github}
to demonstrate its practicality for in-line packet processing.
Our microbenchmarks used an m4.xlarge AWS instance with a 4-CPU Intel Xeon E5
processor @ 2.30 GHz and 16 GB memory.

\begin{table}[ht]
  \centering
  % \renewcommand{\arraystretch}{0.000023}
  \begin{tabular}{lrrr}
    \toprule
    & \bf Encode Time & \bf Decode Time\\
    \midrule
    Strawman 1a/1c & $1$ ns/pkt & 0 \\
    Strawman 1b & $51$ ns/pkt & 0 \\
    Strawman 2 & $27$ ns/pkt & $830$ ms \\
    \bf \textcolor{black!50!blue}{Power Sum} & \bf \textcolor{black!50!blue}{33 ns/pkt} & \bf \textcolor{black!50!blue}{2.82 $\mu$s} \\
    \bottomrule
  \end{tabular}
  \caption{
  % Encode and decode times of different quACK constructions.
  The CPU overheads of power sums are comparable to those of the
  strawmen, while being more efficient in space and computation. The
  encode time includes constructing and serializing the quACK(s), given $n$
  identifiers. The decode time includes finding the identifiers of either
  $R$ or $S \setminus R$, given the quACK(s) and $S$.
  Parameters: $n=25$, $t=10$, $b=4$.
  % Average of 1000 trials
  % Encode benchmarks encode 1000 packets
  }
  \label{tab:quack:strawmen-practical}
\end{table}


A power sum quACK that represents $n=25$ outstanding packets
(packets in consideration on that path segment not yet known to be received or lost)
and up to $t=10$ missing
packets with $b=4$-byte identifiers adds $33$\,ns of encoding time per
packet and takes $2.82$\,$\mu$s to decode (\Cref{tab:strawmen-practical}).

\subsection{Identifier bit widths}
\label{sec:quack:psum-microbenchmarks:bit-widths}

Different bit widths have different implications for
which instructions the CPU can use. Modular operations are efficient
for 16- and 32-bit integers, fitting within the 64-bit word size (the number of
bits that can be processed in one instruction) of most modern CPUs. For example,
to multiply two 32-bit integers, we cast them to 64-bit integers, multiply,
then take the modulus.

\Cref{fig:quack-plots} shows the best performance we achieved at different bit
widths. For 16-bit identifiers only, we precomputed power tables that fit in
the L3 cache. For 64-bit identifiers, we implemented Montgomery modular
multiplication~\cite{montgomery1985modular} to avoid an expensive hardware
division for 128-bit integers. In the remainder of the paper, we use $b=4$ as
the preferred tradeoff between space and collision probability.

\subsection{Encoding time}
\label{sec:quack:psum-microbenchmarks:encoding}

The encode time per-packet is directly proportional to
the threshold number of missing packets $t$ (\Cref{fig:construction-time}) at
$33/10 \approx 3$ ns/power sum.
Each power sum can be updated in a constant number of operations based on the
previous power sum, so encoding an identifier requires $t$ modular additions
and multiplications for the $t$ power sums.

\subsection{Decoding time}
\label{sec:quack:psum-microbenchmarks:decoding}

The decode time must be comparable to the time it takes to process a typical
ACK and modify the logic in the transport protocol. Decoding typically
occurs on end hosts, compared to encoding which occurs in the middle of the path.

Finding the solution to the system of power sum polynomial equations boils down
to applying Newton's identities (a linear algorithm) and finding the roots of a
polynomial equation
in a modular field~\cite{eppstein2011straggler}.
Factoring a polynomial is asymptotically fast in theory, but the implementation
is branch-heavy and complicated~\cite{batut2000user}.
We found that plugging in and evaluating which of $n$ candidate roots
evaluated to zero was faster in practice for $n < 40,000$ roots.
This is the method we use to decode the power sum quACK.

The decode time of this method is directly proportional to $n$ (\Cref{fig:n-vs-decoding})
and the number of missing packets $m$ (\Cref{fig:m-vs-decoding}).
Decoding takes $2820/10/25 \approx 11$ ns/candidate/missing.
Both $n$ and $m$ are typically a few hundred at most.

\section{IBLT microbenchmarks}
\label{sec:quack:iblt-microbenchmarks}

Both the IBLT and power sums are derived from set reconciliation
techniques, so when should we prefer one to the other?
We adapt the power sum quACK from \cite{yuan2024sidekick} for the interface
in \Cref{lst:quack-interface}
so that they can be used interchangeably in our microbenchmarks and compare
it to our implementation of the IBLT quACK.
In this section, we compare the overheads in practice.
We run the microbenchmarks on a single core of an AWS m4.xlarge instance.

\subsection{Encoding and decoding time scalability}
\label{sec:quack:iblt-microbenchmarks:scalability}

\begin{figure}[t]
    \centering
    \begin{subfigure}[b]{0.49\linewidth}
        \centering
        \includegraphics[width=\linewidth]{figures/quack_encode.pdf}
        \caption{Encode time.}
        \label{fig:quack:encode}
    \end{subfigure}
    \begin{subfigure}[b]{0.49\linewidth}
        \centering
        \includegraphics[width=\linewidth]{figures/quack_decode.pdf}
        \caption{Decode time.}
        \label{fig:quack:decode}
    \end{subfigure}
    \caption{IBLT vs. power sum eACK microbenchmarks. The number of trials is
     such that the cumulative time is at least $100$ ms. With rateless eACKs,
     the client can encode much larger numbers of symbols $t$ while the proxy
     decodes fewer symbols in the common case.
     % The logscale axes emphasize the overheads at smaller numbers.
     }
    \label{fig:quack}
\end{figure}

Encoding in the IBLT is faster in practice when there are at least
$\approx\!30$ symbols (\Cref{fig:quack:encode}).
Each update in the IBLT uses an expensive square root instruction in the
pseudorandom mapping algorithm, adding constant factor overheads that impact
smaller numbers of symbols.
m
Decoding in the IBLT is much faster in practice for any number of symbols
(\Cref{fig:quack:decode}). Note that the power sum quACK actually uses a decoding
method that is linear in \textit{all} packets received, not just the \textit
{missing} packets, due to the complexity of symmetric polynomial
factorization.

\subsection{Probabilistic correctness}
\label{sec:quack:iblt-microbenchmarks:correctness}

\begin{figure}[t]
    \centering
    \includegraphics[width=0.9\linewidth]{packrat-paper/figures/quack_multiplier.pdf}
    \caption{The CDF of the minimum number of symbols $t$ needed to successfully
    decode an quACK for various numbers of missing packets $m$, 100000 trials.}
    \label{fig:iblt-quack}
\end{figure}

It is unknown how many symbols are required in the IBLT to decode the same
number of $m$ missing packets using power sums in the previous benchmarks.
There is a constant factor overhead ($1.35\times$ on average~\cite
{yang2024practical}).
\Cref{fig:iblt-quack} shows the CDF of the minimum number of symbols to decode
various $m$ as this constant multiplier increases. $m=1$ trivially requires
one symbol, while the multiplier decreases for higher $m$ to achieve the same
success rates.

The quACK sender does not know how many symbols it needs to encode to later
decode a certain $m$. Using power sums, this is exactly $m$ symbols. In
the quACK, $4 \cdot m$ symbols have at least a $\!98.8\%$ success rate for
all $m$ evaluated in \Cref{fig:iblt-quack}. When $m$ is large, the quACK utilizes more
of the link, but maintaining more symbols at the proxy and client means
the Packrat has a greater worst-case tolerance for errors with the same encoding
overheads. The Packrat proxy can reset the connection if it cannot decode a quACK.

\section{Summary}
\label{sec:quack:summary}

The power sum quACK is efficient to decode, adds reasonable link overhead,
and is a cumulative representation of the packets seen by the receiver
(\Cref{tab:strawmen-theoretical}).
Compared to Strawman 2, the power sum quACK can be decoded with simple
algebraic techniques.
Its link overhead is proportional only to the number of
missing packets between consecutive quACKs, up to a configurable threshold. In
comparison, the link overhead of Strawman 1 is necessarily proportional to the
number of received packets.
The power sum quACK is also resilient to mis-identifying a received packet as
dropped, in the case a quACK is lost in transmission.

To make the IBLT data structure suitable for small, network packets, we
carefully consider constant factor overheads in the
number of symbols and the computational efficiency. We find the quACK
is most likely to be useful in settings with high, bursty loss, where
encoding and decoding large numbers of symbols is more efficient than the
quACK. However, the quACK is still more efficient when
loss is small or infrequent.
