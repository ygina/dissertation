\begin{figure}[t]
    % \centering
    % Client payloads
    \begin{subfigure}[b]{0.48\linewidth}
        \begin{protopayload}{\texttt{Init}}
            \begin{lstlisting}[language=Rust]
epoch: u32;
base_conn: [u8; 12];
quack_ty: u8;
num_symbols: u8;
id_offset: u16;
mem_bytes: u32;
            \end{lstlisting}
        \end{protopayload}
        \begin{protopayload}{\texttt{quACK}}
            \begin{lstlisting}[language=Rust]
count: u32;
last_element: u32;
code: Vec<Symbol>;
            \end{lstlisting}
        \end{protopayload}
        \caption{Client payloads.}
        \label{fig:packrat:payloads:client}
    \end{subfigure}
    \hfill
    % Proxy payloads
    \begin{subfigure}[b]{0.48\linewidth}
        \begin{protopayload}{\texttt{InitACK}}
            \begin{lstlisting}[language=Rust]
epoch: u32;
udp_port: u16;
errno: u32;
            \end{lstlisting}
        \end{protopayload}
        \begin{protopayload}{\texttt{Reset}}
            \begin{lstlisting}[language=Rust]
epoch: u32;
errno: u32;
            \end{lstlisting}
        \end{protopayload}
        \begin{protopayload}{\texttt{Retransmit}}
            \begin{lstlisting}[language=Rust]
udp_payload: Vec<u8>;
            \end{lstlisting}
        \end{protopayload}
        \caption{Proxy payloads.}
        \label{fig:packrat:payloads:proxy}
    \end{subfigure}
  % Caption
  \caption{Packrat protocol messages to initialize the connection and generate
   retransmissions. There are two constructions of the quACK: power sum and
   IBLT (\Cref{sec:quack}). The \texttt{Symbol} uses 5 and 4 bytes in each
   type, respectively.}
  \label{fig:packrat:payloads}
\end{figure}
