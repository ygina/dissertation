\chapter{Conclusion}
\label{sec:conclusion}

\section{Summary}
\label{sec:conclusion:summary}

\section{Future work}
\label{sec:conclusion:future}

\subsection{Real-world studies}
\label{sec:conclusion:future:real-studies}

Our emulations model non-congestive loss as random, independent loss,
which may not accurately reflect the real world. While non-congestive loss
does exists, its various behaviors are not well-characterized. This work
would benefit from a measurement study on the patterns of non-congestive loss
to better understand the scope of the problem and its impact.

We base this work on the idea that proxies and link-layer approaches are
deployed for a reason, and understanding the impact that encrypted transport
protocols have in these real-world network paths would better motivate the need
for protocol-agnostic proxies at all. This is also true on paths where
reliable connectivity is not a solved problem. Also,
in-network retransmissions intuitively reduce load in the network and it may be
interesting to explore how these retransmissions benefit not just
applications but the network itself.

\subsection{Deployment and standardization}
\label{sec:conclusion:future:deployment}

The question for any research system is how can it be deployed and useful in
the real world. The simplest deployment today could look like a simple in-line
Packrat proxy between a Wi-Fi router and Internet modem. Public hotspots and
dense urban apartments make good candidates for smaller-scale lossy path
segments.
The client integration could be a browser extension for QUIC or a simple
modified media client.
The Packrat protocol has the advantage that only one endpoint in addition to the
proxy needs to speak the protocol.

\subsection{Set reconciliation in packet-scale settings}
\label{sec:conclusion:future:set-reconciliation}

In this paper, we explored an alternative construction of the quACK that is
optimized for computational efficiency at a small granularity. An extension to
this is finding other applications of the quACK in the setting of network
packet analysis. We may also be able to further improve the link overheads of
the IBLT quACK at millisecond timescales using, e.g., interactive quACKing
between the proxy and endpoint.

% \subsection{Congestion Control Algorithm Design}

% The correspondence between endpoint-driven PACUBIC and ``split CUBIC''
% is good, and both are better than end-to-end CUBIC in
% \Cref{fig:loss-vs-tput}), but not exact. The appropriateness of the
% PACUBIC heuristic, and in general the idea of path-aware congestion
% control, needs to be further explored. We discuss this more in
% \Cref{sec:appendix:pacubic}.

\section{Concluding Remarks}
\label{sec:conclusion:remarks}
...
