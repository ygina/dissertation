\chapter{Conclusion}
\label{sec:conclusion}

\section{Summary}
\label{sec:conclusion:summary}

\section{Future work}
\label{sec:conclusion:future}

% % \subsubsection{Real-world studies}
% Our emulations model non-congestive loss as random, independent loss,
% which may not accurately reflect the real world. While non-congestive loss
% does exists, its various behaviors are not well-characterized. This work
% would benefit from a measurement study on the patterns of non-congestive loss
% to better understand the scope of the problem and its impact.

% We base this work on the idea that proxies and link-layer approaches are
% deployed for a reason, and understanding the impact that encrypted transport
% protocols have in these real-world network paths would better motivate the need
% for protocol-agnostic proxies at all. This is also true on paths where
% reliable connectivity is not a solved problem. Also,
% in-network retransmissions intuitively reduce load in the network and it may be
% interesting to explore how these retransmissions benefit not just
% applications but the network itself.

% % \subsubsection{Deployment and standardization}
% The question for any research system is how can it be deployed and useful in
% the real world. The simplest deployment today could look like a simple in-line
% Packrat proxy between a Wi-Fi router and Internet modem. Public hotspots and
% dense urban apartments make good candidates for smaller-scale lossy path
% segments.
% The client integration could be a browser extension for QUIC or a simple
% modified media client.
% The Packrat protocol has the advantage that only one endpoint in addition to the
% proxy needs to speak the protocol.

% % \subsubsection{Set reconciliation in packet-scale settings}
% In this paper, we explored an alternative construction of the quACK that is
% optimized for computational efficiency at a small granularity. An extension to
% this is finding other applications of the quACK in the setting of network
% packet analysis. We may also be able to further improve the link overheads of
% the IBLT quACK at millisecond timescales using, e.g., interactive quACKing
% between the proxy and endpoint.

\section{Concluding Remarks}
\label{sec:conclusion:remarks}
...
