\chapter{Conclusion}
\label{sec:conclusion}

This dissertation explored a new approach to resolving the tension between
performance and ossification for secure transport protocols. This tension
emerged over the last few decades starting with the growth of wireless networks
and lossy, high-delay paths. TCP performance-enhancing proxies were deployed to
ameliorate the poor performance of ``loss-based'' congestion control algorithms
such as CUBIC in these settings, but PEPs have since become controversial for
preventing existing transport protocols from evolving on the wire. This led to
the deployment of ``secure'' transport protocols such as QUIC that encrypt
their transport headers and forgo all proxy assistance.

In the \textbf{Sidekick} approach to in-network assistance, proxies and
endpoints send information on an \textit{adjacent} connection to the base
connection \textit{about} the encrypted packets that they have received. This
information is called the \textbf{quACK}, applying set reconciliation
techniques to be able to efficiently refer to random identifiers even at
packet-scale settings (\Cref{sec:quack}). When proxies and endpoints send
quACKs using the Sidekick protocol (\Cref{sec:sidekick}) or Packrat protocol
(\Cref{sec:packrat}), we find that secure transport protocols are able to
achieve performance benefits similar to those achieved by TCP using traditional
PEPs. These benefits include earlier retransmissions, path-aware congestion
responses, reduced energy usage at the endpoint, and reduced network
congestion---all without modifying the wire format nor the packets of the
underlying connection.

At times, it seems the world has already moved on from PEPs, designing new
congestion control algorithms and transport protocols to supplant those of the
1990s connection-splitting era. But this dissertation has shown that even the
latest ``model-based'' BBR algorithm remains vulnerable to lossy networks---and
that in some network conditions, at least in emulation, proxies still
significantly improve the long-lived throughput of modern QUIC
(\Cref{sec:splitting}). In-network assistance from proxies continues to offer
valuable performance benefits, but will we have learned our lesson about
ossification? The Sidekick approach is one such proposal for reconciling the
two. I hope this work encourages efforts to better establish the scope of this
problem in real-world deployments, and to design more practical, deployable
methods that reimagine how endpoints and the network can securely collaborate.

\section{Future work}
\label{sec:conclusion:future}

% % \subsubsection{Real-world studies}
% Our emulations model non-congestive loss as random, independent loss,
% which may not accurately reflect the real world. While non-congestive loss
% does exists, its various behaviors are not well-characterized. This work
% would benefit from a measurement study on the patterns of non-congestive loss
% to better understand the scope of the problem and its impact.

% We base this work on the idea that proxies and link-layer approaches are
% deployed for a reason, and understanding the impact that encrypted transport
% protocols have in these real-world network paths would better motivate the need
% for protocol-agnostic proxies at all. This is also true on paths where
% reliable connectivity is not a solved problem. Also,
% in-network retransmissions intuitively reduce load in the network and it may be
% interesting to explore how these retransmissions benefit not just
% applications but the network itself.

% % \subsubsection{Deployment and standardization}
% The question for any research system is how can it be deployed and useful in
% the real world. The simplest deployment today could look like a simple in-line
% Packrat proxy between a Wi-Fi router and Internet modem. Public hotspots and
% dense urban apartments make good candidates for smaller-scale lossy path
% segments.
% The client integration could be a browser extension for QUIC or a simple
% modified media client.
% The Packrat protocol has the advantage that only one endpoint in addition to the
% proxy needs to speak the protocol.

% % \subsubsection{Set reconciliation in packet-scale settings}
% In this paper, we explored an alternative construction of the quACK that is
% optimized for computational efficiency at a small granularity. An extension to
% this is finding other applications of the quACK in the setting of network
% packet analysis. We may also be able to further improve the link overheads of
% the IBLT quACK at millisecond timescales using, e.g., interactive quACKing
% between the proxy and endpoint.

\section{Concluding Remarks}
\label{sec:conclusion:remarks}
...
