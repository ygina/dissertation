\chapter{Conclusion}
\section{Summary}
\section{Future Work}

\subsection{Real-World Studies}

Our emulations model non-congestive loss as random, independent loss,
which may not accurately reflect the real world. While non-congestive loss
does exists, its various behaviors are not well-characterized. This work
would benefit from a measurement study on the patterns of non-congestive loss
to better understand the scope of the problem and its impact.

We base this work on the idea that proxies and link-layer approaches are
deployed for a reason, and understanding the impact that encrypted transport
protocols have in these real-world network paths would better motivate the need
for protocol-agnostic proxies at all. This is also true on paths where
reliable connectivity is not a solved problem. Also,
in-network retransmissions intuitively reduce load in the network and it may be
interesting to explore how these retransmissions benefit not just
applications but the network itself.

\paragraph{Multipath scenarios.}
We have only considered Sidekick proxies along a single path, and not thought
extensively about how quACKs would interact with protocols such as
\mbox{TCPLS}~\cite{rochet2020tcpls} that use multiple paths or streams,
or even multipath QUIC~\cite{de2017multipath}.
To begin thinking about this question, we would have a more complex model of
the network: multiple PEPs along a single path, multiple paths each with varying
numbers of PEPs, and so on. The proxy can include
additional information in the Sidekick-reply packet to indicate which path the
PEP assistance is on, and the sender can infer from the RTT how far along a path each PEP
is relative to others. New Sidekick algorithms that come from this model could
diagnose troublesome paths, or better allocate network traffic in a multipath
connection. Existing algorithms could be applied to individual paths as if they
were single-path connections.

\paragraph{Even more diverse network scenarios.}
The three scenarios we explored all consisted of a lossy Wi-Fi link and a
high-latency WAN link. Not all scenarios will be favorable to the
Sidekick protocol we designed.
If the ``lossy'' section of a network path were on the far path segment from the
sender, the sender would not have any more information about the problematic
link. To accomodate scenarios like this, Sidekick protocols will need
more features. For example, the \emph{proxy} would need some way to receive
quACKs from the data \emph{receiver}, as well as a mechanism to buffer and
retransmit packets~\cite{balakrishnan1995snoop,caini2006pepsal}.

There are likely other scenarios that could benefit from Sidekick protocols as
described, but we did not evaluate them. For example, if we replaced the lossy Wi-Fi
link with a modern wireless link that has a fluctuating physical
capacity~\cite{niu2015survey,burchardt2014vlc,koenig2013wireless},
the sender may be able to more quickly adapt and make
data available for transmission whenever capacity intermittently becomes available.

\subsection{Deployment and Standardization}

The implementation of Robin exists as a research system that has been evaluated
in emulation and a limited set of real-world scenarios. Since Sidekick protocols
require the cooperation of middleboxes and client applications, more work will
be needed to standardize the discovery protocol and wire format of Sidekick messages
described in \Cref{sec:design}, ideally with interest from the IETF.
The standards will need to establish several design choices such as how
identifiers are computed, how quACKs are transmitted, and the exact mechanisms
for security and backwards compatibility.
We may also want to standardize sender behavior for specific base protocols,
though this could be opaque except to the sender.

The deployment of Sidekick protocols can be gradual and backwards-compatible
with parties that are either unaware of or do not want to participate in Sidekick
protocols.
To migrate existing client applications, one needs to modify the code to
discover a PEP and use information in a quACK to inform the base protocol.
To migrate middleboxes, they would need to be modified to listen for
Sidekick-request markers, then accumulate and send quACKs for participating
connections.

The question for any research system is how can it be deployed and useful in
the real world. The simplest deployment today could look like a simple in-line
Packrat proxy between a Wi-Fi router and Internet modem. Public hotspots and
dense urban apartments make good candidates for smaller-scale lossy path
segments.
The client integration could be a browser extension for QUIC or a simple
modified media client.
The Packrat protocol has the advantage that only one endpoint in addition to the
proxy needs to speak the protocol.

\subsection{Set Reconciliation in Packet-Scale Settings}

In this paper, we explored an alternative construction of the quACK that is
optimized for computational efficiency at a small granularity. An extension to
this is finding other applications of the quACK in the setting of network
packet analysis. We may also be able to further improve the link overheads of
the IBLT quACK at millisecond timescales using, e.g., interactive quACKing
between the proxy and endpoint.

% \subsection{Congestion Control Algorithm Design}

% The correspondence between endpoint-driven PACUBIC and ``split CUBIC''
% is good, and both are better than end-to-end CUBIC in
% \Cref{fig:loss-vs-tput}), but not exact. The appropriateness of the
% PACUBIC heuristic, and in general the idea of path-aware congestion
% control, needs to be further explored. We discuss this more in
% \Cref{sec:appendix:pacubic}.

\section{Concluding Remarks}
...