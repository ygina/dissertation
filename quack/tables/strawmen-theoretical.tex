\begin{table*}[ht]
  \centering
  % \renewcommand{\arraystretch}{0.000023}
  \small
  \begin{tabular}{llllll}
    \toprule
    & \bf Per-Packet & \bf & \bf Num QuACK & \bf Payload & \bf Cumu-\\
    & \bf Encode Time & \bf Decode Time & \bf Packets & \bf Size (bytes) & \bf lative?\\
    \midrule
    Strawman 1a & N/A & N/A & $n$ (UDP) & $b$ & No \\
    Strawman 1b & Move sliding & N/A & $n$ (UDP) & $b \cdot window$ & No \\
    & window  & & & & \\
    Strawman 1c & N/A & N/A & $n$ (TCP) & $b$ & No \\
    Strawman 2 & Concatenate & Concatenate and & $1$ (UDP) & $32+4$ (hash & Yes \\
    & and hash & hash $\binom{n}{m}$ subsets & & and count) & \\
    Power Sums & $t$ modular & Plug $n$ candidate & $1$ (UDP) & $4+b+b\cdot t$ ($t$ & Yes \\
    & multiplications & roots into a degree-$m$ & & power sums, last & \\
    & and additions & polynomial & & $t$ value, count) & \\
    \bottomrule
  \end{tabular}
  \caption{Strawmen compared to the power sum quACK representing $n$ packets
  sent by the data sender, $m$ missing packets, and $b$-byte identifiers. The
  power sum quACK uses the threshold $t$. The total data overhead of each
  quACK must consider the packet payload size along with transport headers.
  We evaluate the overheads in practice in \Cref{sec:sidekick:emulation:cpu-overheads}.
  }
  \label{tab:quack:strawmen-theoretical}
\end{table*}
\begin{table}[t]
    \centering
    \begin{tabular}{r l l}
        \toprule
        \bf & Power Sum~\cite{yuan2024sidekick} & IBLT \\
        \midrule
        Encode & $O(t)$ & $O(\log(t))$ \\
        Decode & $O(Nt)$ & $O(m\log(t))$ \\
        $t=$Size & $O(m)$ & $O(m)$ \\
        \bottomrule
    \end{tabular}
    \caption{The computational complexity of the encoding and decoding
    operations of each type of quACK, and the number of symbols. The IBLT
    is theoretically better or the same in all regards, but it
    incurs constant overheads in the size and in hashing.
    $N$ is the number of packets at the proxy, $m < N$ is the number of
     missing packets, and $t$ is the number of symbols.}
    \label{tab:quack-complexity}
\end{table}
