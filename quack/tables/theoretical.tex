\begin{table}[t]
\centering
\begin{tabular}{lccc}
  \toprule
  & \bf Encode Time & \bf Decode Time & \bf QuACK Size \\
  \midrule
  Strawman 1a & $O(1)$ & $O(1)$ & \textcolor{red}{$O(n)$} \\
  Strawman 1b & $O(1)$ & $O(1)$ & \textcolor{red}{$O(n)$} \\
  Strawman 1c & $O(1)$ & $O(1)$ & \textcolor{red}{$O(n)$} \\
  Strawman 2 & $O(1)$ & \textcolor{red}{$O(\binom{n}{t})$} & $O(1)$ \\
  \bf \textcolor{black!50!blue}{Power Sums} & \bf \textcolor{black!50!blue}{$O(t)$} & \bf \textcolor{black!50!blue}{$O(t^2)$} & \bf \textcolor{black!50!blue}{$O(t)$} \\
  \bf \textcolor{black!50!blue}{IBLT} & \bf \textcolor{black!50!blue}{$O(\log t)$} & \bf \textcolor{black!50!blue}{$O(t\log t)$} & \bf \textcolor{black!50!blue}{$O(t)$} \\
  \bottomrule
\end{tabular}
\caption{The algorithmic complexity of the costs of each quACK construction.
The encoding time includes inserting an element into the data structure that
represents the quACK(s). The decoding time
 includes finding the elements of either $S \setminus R$ or $R \subseteq S$,
 given the quACK(s) and $S$.
$n = |S|$ is the total number of sent elements, $t \geq |S \setminus R|$ is an
upper bound on the number of missing elements. \gina{True value of Strawman 2 $O(1)$ encoding?}\\
}
\label{tab:quack:theoretical}
\end{table}
